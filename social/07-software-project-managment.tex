\documentclass[]{../industrial-development}
\graphicspath{{template/}}

\title{Деятельность менеджера программных проектов в компаниях ИТ-индустрии}
\author{}
\date{}

\begin{document}

\begin{frame}
  \titlepage
\end{frame}

\section{Взаимодействие разработчиков и менеджеров}

\subsection{Степень вовлечённости менеджера в процесс разработки проекта}
\subsection{Степень вовлечённости разработчика в процесс управления программным продуктом}
\subsection{Средства для взаимодействия разработчиков и менеджеров}

\begin{frame} \frametitle{Мind mapping} 
\begin{block}{}
Мind mapping - это графический способ изображения информации в процессе мышления в удобной для человеческого восприятия форме – логических и ассоциативных схем.
\end{block}
\begin{itemize}
\item MindMup 2 
\item MindMeister
\item GanttPro
\end{itemize}
\end{frame}

\lecturenotes

Мind mapping - это графический способ изображения информации в процессе мышления в удобной для человеческого восприятия форме – логических и ассоциативных схем. В русском языке понятие майндмэппинг известно как интеллект-карта, ассоциативная карта, карта мыслей, схема мышления – у различных авторов по-разному.

MindMup 2 — один из самых популярных сервисов для построения ассоциативных карт. Чистый и интуитивно понятный интерфейс. Есть возможность совместной работы. В бесплатной версии карты можно сохранять только публично. Приватные карты доступны в платной версии. Сервис удобен для новичков, также подойдет для не слишком сложных, но регулярных задач.

MindMeister — еще один сервис для майндмэппинга. Он является платным, но имеет больший функционал. Иконки, библиотека изображений, история правок, режим презентации, task feed, чат и многое другое. Подходит для планирования разработки приложения или другого сложного продукта.

GanttPro — приложение, которое позволяет планировать проекты и контролировать процесс их выполнения. В основе продукта — популярный принцип диаграммы Гантта. Используются две оси: на вертикальной располагается список задач, горизонтальная отводится календарю. Задачи можно объединять в группы и делить на подзадачи, прогнозировать сроки и отслеживать статусы. Сервис наиболее полезен компаниям, которые разрабатывают сложные продукты с большим объемом работы~\cite{NetologyInstruments}.


\begin{frame} \frametitle{Планирование задач} 
\begin{itemize}
\item Trello
\item MeisterTask
\item Jira
\item Bitrix24
\item Redmine
\end{itemize}
\end{frame}

\lecturenotes

Чтобы планировать и распределять задачи, а также контролировать процесс их выполнения, используются различные программы-органайзеры. 

Trello — популярный сервис для организации задач в виде рабочих досок. Базовая часть сервиса бесплатная, оплачиваются только расширения. Есть возможность добавлять чек-листы и выставлять приоритеты. Может синхронизироваться со Slack, GitHub, Salesforce, Evernote, Google Drive.

MeisterTask — схож по функционалу с Trello. Отличительная особенность — интеграция с упомянутым ранее MindMeister, а заодно — со Slack, GitHub, Zendesk.

Jira —  мощный онлайн сервис, позволяющий командам-разработчикам планировать проекты. Идеально подходит для работы по Scrum. Позволяет планировать спринты и проекты в целом, делегировать задачи и собирать их в бэклог, указывать приоритеты и отслеживать дедлайны.

Bitrix24 — продукт, который позиционируется как сервис автоматизации и оптимизации бизнес-процессов. Включает в себя таск-менеджмент, планирование проектов и времени их выполнения, визуализацию информации на календаре, возможность делиться документами, создавать виртуальные рабочие группы и получать уведомления. 

Redmine - информационная система управления проектами с веб интерфейсом , включающая в себя полный набор средств для совместной работы над проектами. Система позволяет вести одновременно несколько проектов, отслеживать их состояния, управлять шагами проекта, задачами, приоритетами, гибко назначать роли участникам. Продукт бесплатен даже для коммерческого использования и не накладывает никаких ограничений на количество пользователей~\cite{NetologyInstruments}.

\begin{frame} \frametitle{Коммуникация} 
\begin{itemize}
\item Slack
\item Mattermost
\item Skype
\item Google Hangouts
\item Zoom
\end{itemize}
\end{frame}

\lecturenotes

Slack — удобный корпоративный мессенджер, который не нуждается в представлении. Позволяет создавать закрытые и открытые чаты, обмениваться файлами, имеет удобную мобильную версию. Подходит для общения в рамках практически любой команды.

Mattermost — аналог Slack с открытым исходным кодом от GitLab, предоставляющий схожий функционал, но улучшен в вопросах конфиденциальности. Мессенджер можно установить на собственный сервер организации.

Skype — инструмент для аудио- и видеосвязи с коллегами, подрядчиками и клиентами от комании Microsoft, которые живут за границей.

Google Hangouts — менее популярный аналог Skype от компании Google, который отлично справляется с теми же функциями. 

Zoom — еще один сервис для проведения видеоконференций. Позволяет вести запись видео длительностью до 40 минут~\cite{NetologyInstruments}.

\section{Взаимопонимание менеджера и разработчика}
\subsection{Суть проблемы и ее причины}

\begin{frame} \frametitle{Суть проблемы} 
\begin{block}{}
Недопонимание между разработчиками и менеджерами обусловлено различием областей их деятельности
\end{block}

Успех программного проекта во многом зависит от степени сотрудничества внутри команды, поэтому в их взаимоотношениях есть симбиотический аспект
\end{frame}

\lecturenotes

Отношения между разработчиками программного обеспечения и руководителями проектов часто сопряжены с конфликтами. Недопонимание между разработчиками и менеджерами обусловлено различием областей их деятельности. В связи с чем они имеют различный взгляд на производство ПО. Менеджеры зачастую не обладают достаточными техническими навыками для понимания процесса разработки, а программистам не хватает бизнес знаний для понимания управления процессом разработки.

Тем не менее, успех программного проекта во многом зависит от того, как коллектив справляется с конфликтными ситуациями~\cite{Awati}. 

\begin{frame} \frametitle{Положение в организационной иерархии} 
Положение менеджеров в организационной иерархии выше положения программистов, но данная разница является незначительной
\begin{itemize}
\item Менеджеры проекта не имеют специальных технических знаний, необходимых для понимания работы программистов
\item Менеджеры программных проектов зависят от программистов
\end{itemize}
\end{frame}

\lecturenotes

Одной из причин возникновения конфликта является разное положение в организационной иерархии. Не смотря на то, что руководители проектов обычно выше программистов в организационной иерархии, взаимопонимание осложняется двумя факторами:
\begin{itemize}
\item Менеджеры проекта не имеют специальных технических знаний, необходимых для понимания работы программистов.
\item Менеджеры программных проектов значительно зависят от программистов. 
\end{itemize}

Из-за этих факторов у программистов может складываться мнение, что они имеют более высокий или схожий организационный статус (или имеют большее значение), чем руководители проектов. Таким образом, иерархическая близость двух сторон также может быть одной из причин конфликта между ними~\cite{Awati}.

\begin{frame} \frametitle{Различие в специфике работы} 
\begin{itemize}
\item Программисты считают, что основное внимание должно быть уделено созданию высококачественного кода
\item Для руководителей проектов больший приоритет имеет оптимизацию ресурсов и времени, а также удовлетворение требованиям
\end{itemize}
\end{frame}

\lecturenotes

С другой стороны, различия между двумя дисциплинами также являются источником конфликта между двумя сторонами. 
Обе стороны утверждают, что имеют одну и ту же цель - производить качественное программное обеспечение в разумные сроки. Однако они различаются в средствах достижения цели. Программисты полагают, что основное внимание должно быть уделено созданию высококачественного кода. Для руководителей проектов больший приоритет имеет оптимизацию ресурсов и времени, а также удовлетворение требованиям системы. В следствии чего, программисты считают, что им приходится идти на компромисс по качеству (или эстетике) из-за временного давления, а с другой стороны, руководители проектов считают, что программисты не понимают коммерческий императив для продвижения проекта вперед~\cite{Awati}.

\subsection{Способы решения}

\begin{frame} \frametitle{Участие в процессе управления} 
\begin{block}{}
Привлекайте разработчиков на ранней стадии проектной консультации
\end{block}

Когда разработчики понимают цель продукта и задачи клиента, они могут своевременно дать менеджеру совет и свести к минимуму ошибки и недопонимания
\end{frame}

\lecturenotes

Не смотря на то, что оценка времени и ресурсов, необходимых для проекта является трудоемко задачей, определение объема и требований проекта поможет избежать проблем с этим. Прежде чем согласиться на предложенный объем, руководители проектов должны иметь четкое представление о бизнес-целях клиента. Важно, чтобы разработчики были вовлечены в процесс планирования и определения требований, так как именно перед ними будет стоять задача реализации этих бизнес-цели.

Когда обе стороны поняли бизнес-цели проекта, клиент должен определить свои приоритетные задачи. Ответственность менеджера — донести до клиента, что основные цели — это те, что определяют успех проекта~\cite{Codementor}. 

\begin{frame} \frametitle{Область компетентности} 
\begin{block}{}
Уважайте компетентность программистов
\end{block}
Разработчики лучше разбираются в технологиях, поэтому менеджеру следует использовать возможность посоветоваться с разработчиком, чтобы получить хороший результат
\end{frame}

\lecturenotes

Одна из основных задач менеджера является координация команды, которая включает в себя множество обязанностей. Менеджеру приходится контролировать и влиять на каждую часть проекта, что непременно порождает стресс. 

Не смотря на то, что вы должны управлять командой, вы должны уважать компетентность программистов и доверять их опыту и знаниям. Это облегчит вашу жизнь, и позволит программистам стать более продуктивными и уверенными в своей работе. К тому же, в большинстве случаев разработчики знают о технологии больше, чем их руководитель, поэтому менеджеру следует использовать это и не упускать любую возможность посоветоваться с разработчиком, чтобы получить хороший результат~\cite{Codementor}.

Когда нужно внести изменения, менеджеру следует спросить, а не поставить перед фактом. У разработчиков должен быть выбор: «Если возможно...» намного лучше, чем «Мне нужно, чтобы ты...»

\begin{frame} \frametitle{Коммуникация} 
\begin{block}{}
Важно наладить эффективную коммуникацию внутри команды
\end{block}
\begin{itemize}
\item Реализуйте четкий набор правил и рекомендаций по коммуникации
\item Проработайте структуру и план совещаний
\end{itemize}
\end{frame}

\lecturenotes

Правильная коммуникация способна повысить производительность команды. Если вы хотите создать эффективную линию связи с вашим менеджером или программистом, вам необходимо реализовать четкий набор правил и рекомендаций по коммуникации, которые обеспечат прозрачность, краткость и лаконичность. Вся информация должна своевременно передаваться, а также достигать нужного человека~\cite{Knowledgehut}.

Проработайте структуру и план встречи. У собрания должны быть временные рамки, а каждый участник должен понимать, о чем ему следует рассказать и какие встречные вопросы ему могут задать. Совещания должны быть регулярными, но не препятствовать процессу разработки~\cite{Codementor}.

\begin{frame} \frametitle{Командный дух} 
\begin{itemize}
\item Ограждайте разработчиков от внешних отвлекающих факторов
\item Консультируйтесь с разработчиками относительно любых просьб об изменениях от клиента
\item Разработчики должны нести ответственность за проект
\end{itemize}
\end{frame}

\lecturenotes

Менеджер проекта — лидер команды, и его задача — защищать ее. Когда руководство сталкивается с трудными вопросами или невыполнимыми запросами клиентов, старайтесь самостоятельно решить эти вопросы, чтобы разработчики продолжили выполнять свою работу. Не позволяйте клиентам напрягать участников команды крайними сроками или влезать в работу проекта.

Хотя задача проект-менеджеров — защищать свою команду и помогать им справиться с проблемами, эти неприятные аспекты работы могут быть сведены к минимуму, если они действительно будут брать на себя полную ответственность за проект: своевременно обновлять и доносить информацию, сообщать о задержках и изменять условия для обеих сторон. Консультируйтесь с разработчиками относительно любых просьб об изменениях от клиента. Убедитесь, что они следят за временем и рабочие журналы ведутся должным образом, а не заочно перед проверкой~\cite{Codementor}.

\begin{frame} \frametitle{Критика} 

\begin{block}{}
Важно серьезно подходить к рассмотрению любой критики!
\end{block}
При анализе необходимо провести обсуждение, выслушав обоснования каждой стороны, в результате которого будет найден консенсус .
\end{frame}

\lecturenotes

Accept Criticism~\cite[с.~251--252]{Stellman}.

\begin{frame} \frametitle{Оценка сроков} 
\begin{block}{}
Важно корректно оценивать сроки разработки программного обеспечения
\end{block}
\begin{itemize}
\item При оценке задач используйте относительные единицы
\item При переводе относительной оценки в абсолютную закладывайте дополнительное время
\end{itemize}
\end{frame}

\lecturenotes

Неправильная оценка времени также является следствием конфронтации между менеджером и разработчиками. Менеджеры часто находятся под давлением из-за жестких сроков реализации продукта. Кроме этого, зачастую разработку ПО очень трудно оценить точно. Если команда не успевает выполнить задачи к дедлайну из-за неправильной оценки объема или изменения обстоятельств, то нельзя ожидать от команды сверхурочной работы. Переработки могут привести к понижению продуктивности, а затем к выгоранию~\cite{Netology}.

При оценке задач используйте относительные единицы, которые основываются не на количестве времени, а на сравнительном уровне затрачиваемых усилий при решении. При переводе относительной оценки в абсолютную закладывайте запасное время, так как разработчики склонны неправильно оценивать свои возможности~\cite{Codementor}. 

\begin{frame} \frametitle{Советы разработчикам} 
\begin{itemize}
\item Объясняйте доступно
\item Спрашивайте и предлагайте
\item Уважайте друг друга
\end{itemize}
\end{frame}

\lecturenotes

Если хотите, чтобы менеджер понял, почему одна функция занимает больше времени, чем другая, или почему происходят технические задержки, объясните всё с точки зрения ваших общих бизнес-целей. Это также поможет менеджеру донести суть до клиента.

Задавайте вопросы на начальных этапах проекта. Убедитесь, что вы понимаете цель процессов и не бойтесь рассуждать о направлениях и этапах работы. Это поможет согласовать ваши цели с задачами менеджера и клиента. Если видите потенциальные риски или проблемы, говорите об этом как можно скорее и предлагайте альтернативы. Если недовольны коммуникациями или структурой собраний, не жалуйтесь, а предлагайте свои варианты.

Разработчики и менеджеры проектов не всегда согласны друг с другом, но обе стороны понимают, что каждый просто выполняет свою работу~\cite{Codementor}.

\begin{thebibliography}{99}
\bibitem{NetologyInstruments} \href{https://netology.ru/blog/soft-dlya-upravleniya}{Лучший софт для управления проектами и командой: от планирования до отчетности}
\bibitem{Awati} \href{https://eight2late.wordpress.com/2010/03/04/conflicts-over-identity-on-the-relationship-between-software-developers-and-project-managers/}{Conflicts over identity: on the relationship between software developers and project managers}
\bibitem{Codementor} \href{https://www.codementor.io/blog/project-manager-developer-collaboration-36xk9axtps}{Why Project Managers Suck: How to Close the Gap Between Software Engineers and Project Managers}
\bibitem{Netology} \href{https://netology.ru/blog/zametki-upravlyayushchego-o-pererabotkakh}{Заметки управляющего: 
 О переработках}
\bibitem{Knowledgehut} \href{https://www.knowledgehut.com/blog/project-management/can-project-managers-programmers-find-language}{How Can Project Managers \& Programmers Find the Same Language?}

\end{thebibliography}

\end{document}

%%% Local Variables: 
%%% mode: TeX-pdf
%%% TeX-master: t
%%% End: 
