\documentclass{../industrial-development}
\graphicspath{{12-process-of-turning-a-developer-into-a-manager/}}

\title{Процесс превращения разработчика в~менеджера}
\author{Румянцева Мария Сергеевна, \\Феофанова Алла Александровна, \\ПИ-21 МО}
\date{}

\begin{document}

\begin{frame}
  \titlepage
\end{frame}

\section{Разработчик и менеджер}

\begin{frame} \frametitle{Разработчик и IT-менеджер}
	\alert{Разработчик} — программист, который занимается созданием различных продуктов в ИТ: компьютерных игр, мобильных приложений, веб-сайтов и др. 	\\
~\\
	\alert{IT-менеджер} – это специалист, чьей главной задачей является управление проектом в целом: проектирование и расстановка приоритетов, планирование выполнения задач, контроль, коммуникации, а также оперативное решение проблем.
\end{frame}
\lecturenotes
IT-менеджер – это специалист, чьей главной задачей является управление проектом в целом: проектирование и расстановка приоритетов, планирование выполнения задач, контроль, коммуникации, а также оперативное решение проблем.
IT-менеджеры — это сотрудники, отвечающие за качество услуг, предоставляемых пользователям (сотрудникам компании, а иногда — её клиентам и партнёрам) на базе информационных технологий.
IT-менеджер компании – сотрудник, управляющий информационными процессами. Он разбирается не только в технических аспектах IT-среды, но и в вопросах ее взаимодействия с другими сферами: финансовой, кадровой, рыночной.  Задача менеджера внутри фирмы: знать цели развития бизнеса компании, уметь представить бизнес-процессы для их автоматизации, осуществить правильный выбор информационной системы и рассчитать эффект от ее применения.  Его внешние цели – обеспечить клиентам получение качественных ИТ-услуг или организовать продажу информационных продуктов.
Специфика деятельности разработчика (другое название этой профессии — Developer) всецело зависит от выбранного направления. К примеру, разработкой программного обеспечения прикладного характера (игры‚ бухгалтерские программы‚ редакторы‚ мессенджеры, ПО для систем видео- и аудионаблюдения) занимаются прикладные программисты; созданием операционных систем, работой с сетями, написанием интерфейсов к базам данных озадачены системные программисты; воплощением в жизнь проектов веб-дизайнеров, т. е. созданием сайтов, заняты веб-программисты.


\begin{frame} \frametitle{Важные качества разработчика в IT}
	 \begin{itemize}
	 	\item Сильные технические навыки
		\item Аналитический склад ума
	 	\item Готовность к обучению
	 	\item Умение работать в команде
		\item Стрессоустойчивость
		\item Вовлеченность в рабочий процесс
		\item Способность к концентрации
	 \end{itemize}
\end{frame}
\lecturenotes
Сфера информационных технологий и телекоммуникаций отличается стремительным развитием, и для сохранения набранного темпа становятся всё необходимей специалисты, которые помогли бы ей в этом. В то же время, специалистам следует соответствовать определённым требованиям, без них они не смогут успешно справляться со своими профессиональными задачами. Так какие же личные качества помогают работникам в сфере IT и телекоммуникаций стать настоящими профессионалами?
Такие специалисты в первую очередь должны быть склонны к математике, информатике и работе с техникой, в том числе с компьютерами. Им необходимо иметь аналитический склад ума, хорошую память и способность работать с большим количеством информации. Также незаменимыми качествами для всех сотрудников в этой области, независимо от должности, являются ответственность, организованность, стрессоустойчивость, умение самостоятельно обучаться по специализированной литературе.
Стоит отметить, что облик IT-специалиста, работавшего, к примеру, десять лет назад, существенно отличается от современного. Теперь это не молчаливый, сосредоточенный человек, не отрывающийся от компьютера весь рабочий день, а коммуникабельный сотрудник, готовый к работе в команде и прямому диалогу с клиентами. Создавая продукт, работники IT-сферы и телекоммуникации ориентируются на его будущих потребителей, поэтому хорошо знают интересы и потребности своих потенциальных клиентов.
Важно не просто иметь все эти качества, но и грамотно себя преподносить, например, во время трудоустройства. Главная визитная карточка здесь – резюме, в нём стоит обязательно указать свои преимущества и способности. 
У обладателей таких качеств действительно много шансов найти работу по своей специальности, тем более что спрос на работников в сфере информационных технологий и телекоммуникаций заметно превышает предложение. Конечно, для того чтобы быть компетентными и востребованными им понадобится не только психологическая склонность к профессии. Им не обойтись без хорошего образования, профессиональных знаний и умений, постоянного совершенствования своих навыков. А знание английского языка, к тому же, поможет найти работу в уже известной или очень перспективной зарубежной компании.


\subsection{Основания для перевода разработчика в менеджера}
\begin{frame} \frametitle{Основания компании для перевода разработчика в менеджеры}
\begin{itemize}
		\item Стремление <<вырастить>> менеджера внутри компании		
		\item Отсутствие менеджера, при невозможности увеличения штата сотрудников
		\item Дефицит времени на введение в работу человека со стороны	
		\item Понимание специфики организации и отрасли
		\item Знание сильных и слабых сторон коллектива
	\end{itemize}
\end{frame}
\lecturenotes
???

\begin{frame} \frametitle{Необходимые качества IT-менеджера}
\begin{itemize}
		\item Лидерские качества
		\item Умение находить общий язык
		\item Знание процессов разработки и особенностей продукта изнутри
		\item Готовность взять ответственность не только за себя, но и за других
		\item Желание развиваться
		\item Доверие (авторитет) у заказчика
		\item Достижение определенных профессиональных высот

	\end{itemize}
\end{frame}
\lecturenotes

\subsection{Подготовка менеджера проекта}
\begin{frame} \frametitle{Подготовка менеджера проекта}

	 	\begin{table}[H]
\caption{\label{tab:canonsummary} Обучение специалистов}
\begin{center}
\begin{tabular}{|p{0.2\linewidth}|p{0.8\linewidth}|}
\hline
\textbf{Вид} & \textbf{Специфика} \\
\hline
Официаль-ные &  обучение специалистов в ВУЗ-ах и на специальных курсах, завершение которых удостоверяется соответствующим документом. \\
\hline
Полуофи-циальные  & прохождение краткосрочных курсов, посещение популярных лекций и практических занятий. \\
\hline
Неофициаль-ные &  участие в конференциях, семинарах и ознакомление с соответствующей литературой \\
\hline
Обучение в процессе работы & обучение на рабочем месте, а также самообразование. \\
\hline
\end{tabular}
\end{center}
\end{table} 
\end{frame}
\lecturenotes
официальные – обучение специалистов в ВУЗ-ах и на специальных курсах, завершение которых удостоверяется соответствующим документом. К слушателям этой формы подготовки предъявляются обязательные требования, как при начале, так и при завершении обучения, а в ряде случаев, и по ходу самого обучения;
полуофициальные – прохождение насыщенной программы обучения на краткосрочных курсах (продолжительностью от нескольких дней до нескольких месяцев), посещение популярных лекций и практических занятий. К слушателям этой формы подготовки не предъявляются какие-либо обязательные требования, и они не получают специальных удостоверений об их окончании;
неофициальные – участие в конференциях, симпозиумах, региональных семинарах, собраниях профессиональных обществ, а также ознакомление с соответствующей литературой;
обучение в процессе работы – это обучение на рабочем месте при выполнении конкретного проекта, а также самообразование.

\subsection{Отличия характера работы разработчика и IT-менеджера}


\begin{frame} \frametitle{Отличия в характере работы разработчика}

	 	\begin{table}[R]

\begin{center}
\begin{tabular}{|p{0.5\linewidth}|p{0.5\linewidth}|}
\hline
\textbf{Гибкая методология} & \textbf{Традиционная методология} \\
\hline
\begin{itemize}
\item Тесное общение с заказчиком
\item Приоритет на ПО, чем на документацию
\item В процессе разработки количество итераций неизвестно
\item Приветствие изменения требований в конце разработки
\end{itemize}
 & 
\begin{itemize}
\item Минимальное общение с заказчикам
\item Чрезмерное внимание, уделяемое документации
\item Непрерывный процесс разработки с заранее определенными этапами
\item Сложность изменения требований в конце разработки
\end{itemize} \\
\hline
\end{tabular}
\end{center}
\end{table} 
\end{frame}
\lecturenotes


\begin{frame} \frametitle{Характер работы менеджера}
\begin{itemize}
\item Тесное общение с заказчиком на протяжении всего срока выполнения проекта
\item Мотивирование команды и обеспечение ее нужными условиями работы, доверием и поддержкой
\item Проекты находятся под контролем, отслеживаются ресурсы, риски, время
\item Управление расходами денежных средств по проектам 
\item Контроль выполнения проектов
\end{itemize}
\end{frame}
\lecturenotes


\subsection{Роли менеджеров по модели РАЕI}

\begin{frame} \frametitle{Роли менеджеров — РАЕI\footnote[2]{по мнению д.~Адизеса}}
	Идеального руководителя, по мнению доктора Адизеса, не существует. Согласно его модели PAEI, для достижения эффективности руководителю нужны четыре качества: 
 \begin{itemize}
	 \item \alert{P} — производитель результатов (producer) 
	 \item \alert{A} — администратор (administrator) 
	 \item \alert{Е} — предприниматель (entrepreneur) 
	 \item \alert{I} — интегратора (integrator)
 \end{itemize}
Все это не может сочетаться в одном человеке — в реальности менеджеры выполняют хорошо 1-2 функции. При этом одна из функций у правильного менеджера должна доминировать, а всеми остальными он должен владеть на элементарном уровне.  \\

\end{frame}
\lecturenotes
Код (PAEI) 
Первая функция, которую должен выполнять менеджмент в любой организации, – это (P ), или производство результатов, обеспечивающее результативность организации в краткосрочном аспекте. Почему люди обращаются к вашей компании? Для чего вы им нужны? Какие услуги им требуются? Дело производителя (P ) – удовлетворить их потребности. Оценить эту функцию можно, определив количество людей, которые возвращаются, чтобы приобрести ваши конкурентоспособные продукты или услуги.

Вторая функция, (A ), или администрирование, нужна, чтобы следить за порядком в организационных процессах: компания должна делать правильные вещи в правильной
последовательности с правильной интенсивностью. Задача администратора (A ) – обеспечить
эффективность в краткосрочном аспекте.

Далее нам понадобится провидец. Он определяет направление, которого должна придерживаться организация. Такой человек по натуре способен на упреждающие действия в обстановке постоянных изменений, что гарантирует результативность компании в долгосрочной перспективе. Это функция предпринимателя (E ), который сочетает в себе творческий подход и готовность идти на риск. Если организация успешно справляется с выполнением этой функции, ее услуги и/или продукты будут пользоваться спросом у будущих клиентов.

И наконец, менеджмент должен обеспечить интеграцию (I ), то есть создать такую атмосферу и систему ценностей, которые будут стимулировать людей действовать сообща и не дадут никому стать незаменимым, что обеспечит жизнеспособность и эффективность организации в долгосрочной перспективе.

Здоровая организация характеризуется результативностью и эффективностью в ближайшей и долгосрочной перспективе. 
~\cite{How_to_be_a_good_IT-manage


\begin{frame} \frametitle{Производитель}
	 \begin{block}{Ключевой вопрос P:}
		ЧТО делаем?
	\end{block}	
 \begin{itemize}
	 \item Ориентирован на результат
	 \item Самостоятельное выполнение работы приносит наибольшее удовлетворение
	 \item Сосредоточен на текущих потребностях, что мешает оценить ситуацию в полном комплексном масштабе
 \end{itemize}
\end{frame}
\lecturenotes
Производитель (Р) ориентирован, в первую очередь, на результат, создание продуктов и услуг для удовлетворения потребностей покупателей. Для Р выполненная задача собственными руками, не делегируя никому, приносит наибольшее удовольствие. Р ориентированы на текущие потребности и зачастую не способны подняться над ситуацией вверх и посмотреть с высоты, оценить ее в полном комплексном масштабе. 


\begin{frame} \frametitle{Администратор}
	\begin{block}{Ключевой вопрос A:}
		КАК делаем?
	\end{block}	
 \begin{itemize}
	 \item Нацелен на оптимизацию и минимизацию использования имеющихся ресурсов
	 \item Систематизирует и упорядочивает процесс управления
	 \item Устанавливает бюрократический порядок и стандартизацию процедур
	 \item Не терпит нарушения правил и инструкций
	\item Избегает рисков
 \end{itemize}
\end{frame}
\lecturenotes
Администратор (А) — нацелен на оптимизацию и минимизацию использования имеющихся ресурсов. Он систематизирует весь процесс управления организацией, упорядочивает его, устанавливает бюрократический порядок, прописывает функциональные обязанности, вводит стандартизированные процедуры. Обожает созданные им же правила и инструкции, не терпит, когда другие нарушают их, всегда доводит начатое до конца, наряду с этим всеми силами избегает риска.


\begin{frame} \frametitle{Предприниматель}
	\begin{block}{Ключевой вопрос E:}
		ЗАЧЕМ и КОГДА делаем?
	\end{block}	
 \begin{itemize}
	 \item Предпочитает генерировать и внедрять что-то новое
	 \item Стремится к постоянному развитию организации
	 \item Стимулирует коллектив к саморазвитию
	 \item Не все идеи воплощает и доводит до конца
 \end{itemize}
\end{frame}
\lecturenotes
Предприниматель (Е) предпочитает генерировать новое, экспериментировать, внедрять. Такой руководитель не дает организации останавливаться на достигнутом. Благодаря тому, что у него всегда много планов и идей, он стимулирует всех вокруг идти дальше, развиваться. Большое количество идей Е приводит также и к тому, что большинство из них не воплощаются, а дела, к которым такой руководитель «остыл», не доводятся до конца. 

\begin{frame} \frametitle{Интегратор}
	\begin{block}{Ключевой вопрос I:}
		КТО делает?
	\end{block}	
 \begin{itemize}
	 \item Создает общие традиций и корпоративную культуру организации
	 \item Определяет общую для всех стратегию
	 \item Поддерживает единство в организации
	 \item Объединяет людей вокруг себя
         \item Может оказаться излишне мягким руководителем
 \end{itemize}
\end{frame}
\lecturenotes
Интегратор (I) — это лидер, способный создать общие традиции, ценности, корпоративную культуру организации. Он определяет общую для всех стратегическую цель, призывает трудиться всем вместе, тем самым поддерживая в организации единство. \\~\\ С ним очень приятно общаться, он объединяет людей вокруг себя. Иногда такой руководитель может быть излишне мягким.

\begin{frame} \frametitle{Модель PAEI}
{\includegraphics[width=0.95\linewidth]{paei.png}}
\end{frame}
\lecturenotes
P отвечает за результат в краткосрочной перспективе — т.е.чтобы все было сделано здесь и сейчас (в настоящем).
A — отвечает за эффективность в краткосрочной перспективе — т.е.чтобы то, что уже cейчас происходит, происходило максимально технологично, правильно и с минимальными затратами.
E — отвечает за результативность в долгосрочной перспективе — т.е. чтобы делались именно те вещи, которые верны в стратегическом плане.
I — отвечает за эффективность в долгосрочной перспективе — т.е.чтобы организация сохраняла свою целостность в будущем, а люди могли и дальше работать вместе эффективно.
В нижней части окна оценивается скорость принятия решений: от низкой слева до предельной справа. Некоторые люди принимают решения методично и неторопливо.
Правая сторона окна – это фокус. Его континуум ограничен локальной ориентацией внизу и глобальной ориентацией вверху. Схема показывает, что, посмотрев в окно, каждый из четырех типов сфокусируется на своем. Схема позволяет определить перспективу, которая открывается перед разными руководителями.
С левой стороны представлена последняя переменная – процесс принятия решений. Одни руководители решают проблемы неструктурированными методами, другие предпочитают структурированный подход. 


\begin{frame} \frametitle{Примеры профилей менеджеров }
	\begin{enumerate}
\item Идеальный (несуществующий в реальности) менеджер: 
		 \begin{itemize}
                     \item Производитель — 100\%
 		 \item Администратор — 100\%
 		 \item Предприниматель — 100\%
		\item Интегратор — 100\%
		\end{itemize} 
\item Хороший менеджер-организатор:	
		 \begin{itemize}
                         \item Производитель — 20\%
 		 \item Администратор — 90\%
 		 \item Предприниматель — 30\%
		\item Интегратор — 30\%
		\end{itemize} 
\end{enumerate}
\end{frame}
\lecturenotes
Идеальный «книжный» менеджер-единорог на 100\% Производитель, на 100\% Администратор, на 100\% Предприниматель и на 100\% Интегратор. То есть он идеально владеет и справляется со всеми функциями.
Как ведет себя идеальный менеджер? Он создает результаты, он отличный администратор, предприниматель и интегратор. Он внимательно прислушивается к тому, что говорится и о чем не говорится. Он осознает необходимость изменений. Он осторожно, избирательно и систематически вносит инновации в работу. Он не боится нанимать ярких и непростых подчиненных, он ищет людей с потенциалом и способен его разглядеть. Он не суетится и не жалуется, а когда нужно - предлагает конструктивную критику. Он умеет и анализировать, и работать на результат, быть чувствительным, но не слишком эмоциональным. Его подчиненные не боятся признавать ошибки, он продвигает сотрудников с управленческим потенциалом и поощряет разумный творческий подход. Его организация - это единый организм, стремящийся к цели, члены организации поддерживают и принимают друг друга, согласны с решениями начальника.
Это «книжный» менеджер, потому что встретить его вы можете только в учебниках. Кажется, что такое описание основано не на реальной жизни, а является результатом поиска идеала. Ни один человек не может иметь сразу всех качеств, необходимых для эффективного управления, потому что роли РАПИ предполагают взаимно конфликтующие личностные качества.
Если менеджера типа РАПИ не существует, то как избежать «ошибочного» менеджмента? Для успешной работы в команде менеджер обязательно должен обладать (или стремиться к обладанию) 9  важными качествами:

1. Он может выполнять все четыре управленческие роли. Как минимум одна такая роль удается ему отлично, все остальные он выполняет удовлетворительно. В его описании кода РАПИ нет прочерков.

2. Он знает собственные сильные и слабые стороны.

3. Он поддерживает контакт с окружающими. Он прислушивается к критике своей работы с целью лучше понять самого себя. Он понимает, что он - то, что он делает.

4. У него сбалансированный взгляд на самого себя. Он реализует свои сильные и слабые стороны.

5. Он принимает свои сильные и слабые стороны и не пытается быть кем-то другим, по крайней мере, в краткосрочной перспективе.

6. Он может оценить и признать отличную работу других, даже в тех ролях, которые ему не очень удаются.

7. Он принимает мнение других в тех вопросах, где оно может быть более глубоким, чем его собственное.

8. Он умеет разрешать конфликты, которые неизбежно возникают, когда люди с разными потребностями и стилями оказываются в одной управленческой команде, чтобы обеспечить необходимый набор профессиональных качеств.

9. Он создает среду, способствующую обучению.
В основном же менеджеры успешно выполняют одну-две функции. Так складывается стиль их управления.

Все члены команды должны хотя бы в минимальном объеме владеть каждой из функций. То есть не должно быть нулей.

Если менеджер не знаком с одной из функций и не понимает ее суть, то он будет крайне негибким и нетерпимым по отношению к коллеге, исполняющему эту функцию. В такой ситуации полноценной работы команды не будет.

Хорошего Интегратора, но не знакомого с работой и трудностями Администратора, всегда будет раздражать и не устраивать «бюрократия» отдела персонала с их дотошностью, следованием инструкциям, даже бездушностью.

\begin{frame} \frametitle{Примеры профилей менеджеров }
	\begin{enumerate}
\setcounter{enumi}{2}
\item Хороший менеджер-мотиватор:
		 \begin{itemize}
                     \item Производитель — 20\%
 		 \item Администратор — 10\%
 		 \item Предприниматель — 20\%
		\item Интегратор — 90\%
		\end{itemize} 
\item Менеджер-<<Трудоголик>>:	
		 \begin{itemize}
                         \item Производитель — 90\%
 		 \item Администратор — 30\%
 		 \item Предприниматель — 10\%
		\item Интегратор — 10\%
		\end{itemize} 
\end{enumerate}
\end{frame}
\lecturenotes
Хороший менеджер должен владеть всеми четырьмя навыками на элементарном уровне. А у правильного менеджера одна из функций должна доминировать.
По мнению Адизеса, так как ни один человек не может соответствовать типу РАПИ, лучшим решением для организации становится подбор управленческой команды, участники которой дополняют друг друга. Необходимо подбирать людей, способных лучше или хуже играть все роли и работать с другими в управленческой команде. Менеджер должен уметь увидеть в других те качества, которые бы компенсировали его собственные недостатки.

Данная идея, конечно, не является откровением. Многие управленцы соглашаются, что суть их работы - подобрать хорошую команду. Но, к сожалению, такое простое определение управленческой работы часто игнорируется на практике. Многие менеджеры, не уважающие себя и неспособные принять тех, кто отличается они них самих, просто боятся более талантливых сотрудников. Они подобны владельцу скаковых лошадей, который, наполнив свою конюшню пони, надеется выиграть главный приз на скачках.

Многие менеджеры недостаточно уверены в себе, чтобы работать с людьми, чей стиль работы отличается от их собственного. Когда в команде присутствуют люди с различными управленческими стилями, весьма вероятным становится возникновение конфликтов. Поэтому хороший менеджер должен быть способен управлять конфликтами - принимать и уважать разные мнения и вырабатывать единую стратегию. Хороший менеджер должен создать обучающуюся среду, где конфликт воспринимался бы не как угроза, а как возможность учиться и развиваться.

\subsection{Типы программистов}
\begin{frame} \frametitle{Типы программистов\footnote[1]{по мнению Дж.~Ханка~Рейнвотера}}
	\begin{block}{Распространенные типы:}
\begin{itemize}
\item Архитектор
\item Конструктивист 
\item Художник
\item Инженер
\item Ученый
\item Лихач
\end{itemize}
\end{block}
\end{frame}
\lecturenotes
На что похож типичный программист? Можно ли построить шаблон программиста, хотя бы для того, чтобы понять мотивы его поведения? Может быть. В психологической науке существуют так называемые тесты оценки личности. Здесь тот же принцип – достаточно разобрать отличительные черты программистов в отдельности, и вы поймете, что многие из этих характеристик (даже если они на первый взгляд кажутся взаимоисключающими) могут существовать в одном лице. 

Дж. Ханк Рейнвотер разделяет породы на три категории: распространенные, редкие и дворовые. Распространенные породы встречаются чаще всего. 
Редкие породы встречаются не так часто, но иногда с ними все-таки приходится сталкиваться. Дворовые породы, как и следует из их названия, приносят не слишком много пользы, но знать о них нужно, хотя бы в силу того, что они существуют. Если регулярно помогать дворнягам повышать уровень знаний и, соответственно, преодолевать слабости, которые они по определению привносят в процесс кодирования, они могут стать очень достойными исполнителями.
Любой отдельно взятый человек может заключать в себе характеристики сразу нескольких пород. Конечно, разобраться в таких людях труднее, но это того стоит. У программистов на редкость богатое «наполнение». 


\begin{frame} \frametitle{Архитектор}
	 \begin{itemize}
                     \item Концентрация на общей структуре кода
 		\item Объектное мышление
		\item Полное решение бизнес-задач
 		\item Сначала анализирет систему, после чего переходит к кодированию конкретных решений
		\item Зачастую общий и непонятный код
		\end{itemize} 
\end{frame}
\lecturenotes
В основном архитекторы концентрируются на общей структуре кода. Они мыслят объектами. Посвящая себя без остатка решению бизнес-задач, они строят абстракции, проводят анализ систем, после чего переходят к кодированию конкретных решений.
Зачастую в высшей степени разумные замыслы архитектора воплощаются в настолько общем и непонятном коде, что людей, могущих разобраться в нем и продолжить начинание, просто не находится.

\begin{frame} \frametitle{Конструктивист}
	 \begin{itemize}
                     \item Быстрое и безошибочное написание кода (даже на этапе $\alpha$-тестирования)
 		\item Интуитивное написание кода, что приводит к его неясности
		\item От процесса написания кода и его результата получают удовольствие
 		\item Сопроводительную документацию создают без какого-либо желания
		\end{itemize}
\end{frame}
\lecturenotes
Конструктивисты получают удовольствие от процесса написания кода и его результата. Стратегическим планированием они себя утруждают не всегда, но факт в том, что с написанием кода они справляются быстро, причем в большинстве случаев ошибок в нем не обнаруживается даже на этапе альфа-тестирования. Код конструктивисты пишут интуитивно, а потому их логика не всегда понятна.
Стоит попросить конструктивиста составить документацию, он обязательно ответит, что код самодокументируемый.

\begin{frame} \frametitle{Художник}
	 \begin{itemize}
                     \item Концентрация на процессе создания кода
		\item Переносе коммерческих требований на программные конструкции
 		\item Искусное сведение объектов пользовательского интерфейса в одну структуру
 		\item Обнаружение тенденции к правильной и логичной организации
		\item Затяжное написание кода из-за излишнего внимания деталям
		\end{itemize}
\end{frame}
\lecturenotes
Художник, как тип программиста, сконцентрирован на процессе создания кода – переносе коммерческих требований на программные конструкции и искусном сведении объектов пользовательского интерфейса в одну изящную структуру. Работая с компонентами без видимого интерфейса, художники обнаруживают тенденцию к правильной и логичной организации. 
Художник часто затягивает кодирование, уделяя слишком много внимания деталям.

\begin{frame} \frametitle{Инженер}
 \begin{itemize}
                     \item Приобретение все возможные средств сторонних производителей
		\item Написание множества компонентов и сведение их воедино
 		\item Тяга к усложнениям
 		\item Нехватка гибкости и универсальности
		\end{itemize}
\end{frame}
\lecturenotes
	Инженерам свойственно скупать все возможные средства сторонних производителей, писать десятки СОМ-объектов и сводить их воедино, так что они прекрасно работают в версии~1.
Инженерам присущя тяга к усложнению лишь тогда, когда появляется новая версия. В своей работе им не хватает гибкости и универсальности.

\begin{frame} \frametitle{Учёный}
 \begin{itemize}
		\item Решение особо трудных задач кодирования
                     \item Разработка идет в соответствии с фундаментальными принципами компьютерных наук
		\item Постоянные усложение в процессе работы
		\item Чрезмерное увлечение идеальностью своих трудов
		\end{itemize}
\end{frame}
\lecturenotes
	Программисты такого типа очень полезны, когда речь заходит об особо трудных задачах кодирования, их идеям нет цены. Отодвигая художественную составляющую программирования на второй план, они делают все в соответствии с фундаментальными принципами компьютерных наук.
	 Учёные очень любят все усложнять. Полное отсутствие контроля может привести к невыполнению проекта в установленный срок.

\begin{frame} \frametitle{Лихач}
	 \begin{itemize}
		\item Оперативное достижение результата 
		\item Продукт достаточно успешно работает вплоть до первой неперехваченной ошибки
                     \item Небрежное отношение к архитектуре кода
		\item Понимание и исправление ошибок занимает много времени
		 
		\end{itemize}
\end{frame}
\lecturenotes
Лихачи – это те, кто делают всё быстро. Забывая о комментариях, отступах и соглашениях об именовании переменных, они, тем не менее, умудряются достигать результата очень оперативно – и, что самое замечательное, вплоть до первой неперехваченной ошибки их продукты вполне успешно работают. 
	 Небрежное отношение лихачей к архитектуре кода, в итоге очень часто приводит к ошибкам, на понимание и исправление которых уходит много времени.

\begin{frame} \frametitle{Типы программистов\footnote[1]{по мнению Дж.~Ханка~Рейнвотера}}
	\begin{block}{Редкие типы:}
\begin{itemize}
\item Волшебник
\item Минималист
\item Аналогист
\item Трюкач
\end{itemize}
\end{block}
\end{frame}
\lecturenotes

\begin{frame} \frametitle{Волшебник}
	\begin{itemize}
		\item Решение наиболее трудных задач программирования путём непонятным для других
		\item Своевременное выполнение работы
		\item Доступный для понимания и сопровождения код
		\item При полной вседозволенности могут выходить за пределы своих полномочий.
\end{itemize}
\end{frame}
\lecturenotes
Каким-то загадочным образом те, кого я называю волшебниками, регулярно решают самые трудные задачи программирования, причем идут такими путями, которые раньше никому и в голову не приходили. Более того – волшебники делают все это вовремя, и иногда у них получаются вполне доступные для понимания программы, которые даже можно сопровождать. 
Но стоит распустить подобным деятелям руки, и вскоре вы превратитесь из здравомыслящего руководителя работоспособной группы программистов в обычного подмастерье. Кроме того, если вы будете слишком полагаться на волшебника, в один прекрасный день он вас разочарует – в конце концов, постоянно творить чудеса никому еще не удавалось. 


\begin{frame} \frametitle{Минималист}
	\begin{itemize}
		\item Скромный, но очень функциональный объём кода
		\item Чётко выстроенные объекты, уазывающие на своё назначение
		\item Решив поставленную задачу, быстро теряют к ней интерес
		\item При обнаружении проблем показывают нежелание их исправлять
		\item  С сопровождением кода дела обстоят хуже всех
	\end{itemize}
\end{frame}
\lecturenotes
	 Несмотря на удивительно скромный объем кода, производимого минималистами, код обычно оказывается очень функциональным. Каждая процедура умещается в редакторе кода на одном экране. Объекты выстроены четко и недвусмысленно сообщают о своем назначении. 
	Решив поставленную задачу, быстро теряют к ней всякий интерес, и при обнаружении проблем выказывают нежелание их исправлять. Иногда минималисты капризны и очень придирчиво выбирают область приложения своих сил. С сопровождением кода дела у них обстоят хуже всех.

\begin{frame} \frametitle{Аналогист}
	\begin{itemize}
		\item Прекрасно справляется с аналогиями, но не слишком силен в абстракциях
		\item Быстро схватывают суть задачи
		\item Создают удобный и понятный код
		\item Не всегда получается создавать объекты с чётким межуровневым интерфейсом
	\end{itemize}
\end{frame}
\lecturenotes
 Это программист, который не слишком силен в абстракциях, но прекрасно справляется с аналогиями. Они очень быстро схватывают суть задачи и в результате создают удобный (в том числе и для сопровождения) код. У некоторых аналогистов есть любимые аналогии, которые они норовят применить ко всем без исключения проблемам разработки программных продуктов.  \\ 
	Аналогисты не воспринимают абстракцию, создавать объекты с четкими межуровневыми интерфейсами у них получается не всегда. Конкретное мышление иногда мешает успешно справляться с поставленными задачами.

\begin{frame} \frametitle{Трюкач}
	\begin{itemize}
		\item Увлекаются разными технологическими приёмами
		\item Постоянно осваивают новинки, но результат от этого не улучшается
		\item Их функции ограничиваются экспериментами с разными инструментальными средствами
		\item Не могут усвоить конечное назначение программы
	\end{itemize}
\end{frame}
\lecturenotes
Данный тип программистов слишком увлекается разными технологическими трюками. Они постоянно осваивают разные новинки, но результат от этого не улучшается. \\ 
	Трюкачи, при всех их познаниях в технологии, часто не могут усвоить конечное назначение программы. Полагая, что их функции ограничиваются экспериментами с разными инструментальными средствами, они отказываются учитывать те аспекты программирования, благодаря которым не затрачиваеся усилия на сопровождение. 

\begin{frame} \frametitle{Типы программистов\footnote[1]{по мнению Дж.~Ханка~Рейнвотера}}
	\begin{block}{Дворовые типы:}
\begin{itemize}
\item Разгильдяй
\item Тормоз
\item Любитель
\item Эклектик
\end{itemize}
\end{block}
\end{frame}
\lecturenotes
~\cite{How_to_be_a_good_IT-manager}

\begin{frame} \frametitle{Разгильдяй}
	\begin{itemize}
		\item Небрежное написание кода
		\item Качественно выполнять обязанности им мешают проблемы личного плана
		\item Необходимо обучение написанию эффективного продукта
		\item Читать их код утомительно и сложно
	\end{itemize}	 
\end{frame}
\lecturenotes
Некоторые программисты небрежны, и это проявляется в коде, который они создают. Они не обращают внимания на такие мелочи, как правильное написание имен переменных, комментариев и отступов. \\~\\
Зачастую качественно выполнять свои обязанности им мешают проблемы личного плана. Тому, как пишется эффективный код, их нужно учить. Они любят начать с одного стиля, а через процедуру-другую перейти к новому. Читать их код очень утомительно, поэтому его приходится часто переписывать, чтобы успеть сдать проект в срок. \\

\begin{frame} \frametitle{Тормоз}
	\begin{itemize}
		\item Программист, который не знает, с чего начать
		\item Очень нерешителен в своих действиях, боится совершить ошибку
		\item Имеет небольшой послужной список
		\item Медленно принимает решения
	\end{itemize}
\end{frame}
\lecturenotes
Тормоз – это программист, который не знает, с чего начать. Он постоянно ищет спецификацию (или ожидает, пока ему дадут), отчаянно надеясь, что она станет для него отправной точкой. Нерешительность в чем-то хороша, поскольку в некоторых случаях она повышает качество кода. Однако иной раз она свидетельствует о низкой квалификации программиста, который не хочет лишних ошибок на этапе прогона.  \\ 
Предоставьте этим ущербным образец кода, чтобы они могли разобраться, с чего начинать, и выбрать стиль, которого им нужно будет придерживаться. Нерешительность часто характерна для неопытных программистов, и, воспользовавшись некоторыми воспитательными методами, вы можете наставить их на путь истинный. Кроме того, нерешительностью иногда страдают программисты, у которых по тем или иным причинам не слишком впечатляющий послужной список. Ну, скажем, в прошлый раз их результаты разнесли в пух и прах, а теперь они хотят исправиться, но очень боятся наступить на те же грабли. Помогите тормозу регулярно добиваться небольших успехов, и тогда все наладится. 

\begin{frame} \frametitle{Любитель}
	\begin{itemize}
		\item Обладает большим желанием стать хорошим программистом
		\item Недостаточно образования
		\item Самоуверен в собственных действиях
		\item Моежт дать <<новизну>> коду
	\end{itemize}
\end{frame}
\lecturenotes
Любители очень хотят стать настоящими программистами. Тщательно изучив какой-нибудь инструмент написания макрокоманд, они возводят себя в ранг хакеров. Единственная причина, по которой они бросают уютные места в отделах поддержки пользователей и тестирования, заключается в том, что, по их мнению, быть программистом – это очень круто. 
Любителям не хватает образования, но по мере их обучения нужно пристально за ними следить и лишь при условии определенных достижений с их стороны поручать им работу над критически важными приложениями. На самом деле иногда свежий, незашоренный взгляд начинающего программиста очень помогает нам – старым брюзгливым технарям.


\begin{frame} \frametitle{Эклектик}
	\begin{itemize}
		\item Создаёт шаблонные программные продукты
		\item Не слишком удачно сочетает в себе качества инженера, разгильдяя и  талантливого художника 
		\item Создаёт путанницу в коде из-за различных смешений стилей кодирования и подключаемых модулей
		\item Не сопровождает код
	\end{itemize}
\end{frame}
\lecturenotes
Эклектики создают шаблонные программные продукты. Представитель этого типа сочетает в себе качества инженера, разгильдяя и не слишком талантливого художника, причем упомянутые ингредиенты находятся в чудовищной диспропорции. Результат их деятельности представляет собой смешение из стилей кодирования и подключаемых модулей при невероятной путанице в коде. 

\begin{frame} \frametitle{Негативные эталоны в~менеджменте\footnote[1]{по мнению Дж.~Ханка~Рейнвотера}}
	\begin{enumerate}
\item Руководители типа <<Мелочная опека>>:
		 \begin{itemize}
                     \item Всезнайка
 		 \item Диктатор
 		 \item Генерал
		\end{itemize} 
\item Неорганизованные руководители:	
		 \begin{itemize}
                     \item Скарлетт О'Хара
		 \item Временщик
 		\item Новичок
		\end{itemize} 
\item Гений
\item Строитель империй тьмы
\end{enumerate}
\end{frame}
\lecturenotes
Назначение негативных эталонов – показать, как не стоит себя вести. Если среди
перечисленных черт вы обнаружите сходство с собственной моделью поведения,
попытайтесь ее скорректировать. 
Многие руководители, которым, по идее, следует стремиться к приобретению лидерских качеств, на деле демонстрируют в своем поведении смесь стилей. Об этих стилях, которые проявляются с той или иной силой, и пойдет речь. Анализируя порочные стили руководства, Дж. Ханк Рейнвотер не утверждает, что они олицетворяют абсолютное зло. 

Значительно важнее другое: по тем или иным причинам отдельные руководители перенимают отрицательные качества деструктивных стилей руководства. Они со всей очевидностью проявляются в рабочей среде и снижают качество лидера. Знание перечисленных ниже негативных эталонов не означает, что вы должны наставлять на путь истинный всех, кто им следует, – ваша цель в том, чтобы выявлять соответствующие типы людей и уметь с ними общаться. 




\begin{frame} \frametitle{Руководители типа <<Мелочная~опека>>}
\begin{itemize}
		\item Часто такими руководителями становится высокопрофессиональные инженеры
		\item Убежденнось в том, что никто не может выполнить работу лучше него
		\item Осуществление излишнего контроля над окружающими
		\item Невозможность граммотно разделить обязанности среди команды
		\item Страх совершить ошибку
	\end{itemize}
\end{frame}
\lecturenotes
	Упадочным этот стиль считается по одной простой причине – он являет собой полную противоположность качественному руководству, которое основывается на делегировании и проверке. Деятели, увлекающиеся мелочной опекой, сводят с ума не только окружающих, но и самих себя. \\~\\
Очень часто руководителями становятся именно те сотрудники компании, которые проявили себя высокопрофессиональными инженерами. Первопричина мелочной опеки кроется в убежденности руководителя в том, что никто не может выполнить работу лучше него.  \\
Первопричина мелочной опеки кроется в убежденности руководителя в том, что никто не может выполнить работу лучше него. Прибавьте к этому страх допустить ошибку, и вы поймете, почему так много руководителей попадаются в эту ловушку. Если практиковать мелочную опеку достаточно долго, вы рано или поздно достигнете предела продуктивности. Этот предел определяется, с одной стороны, вашей способностью делать работу за других, с другой – терпением ваших сотрудников, осознающих невостребованность своих талантов.




\begin{frame} \frametitle{Руководители типа <<Мелочная~опека>>}
	\alert{Всезнайка}~— деятель, который искренне верит в то, что ему известно все о его работе, о работе компании, о конкретных заданиях программистов.\\~\\
\alert{Диктатор}~пытается навязывать своим подчиненным конкретные методы решения поставленных перед ними задач. Он создает вокруг себя атмосферу ужаса, которая не позволяет сотрудникам в полной мере проявлять свои способности.\\~\\
\alert{Генерал}~имеет значительное сходство с диктаторами, проявляя в своей деятельности еще большую жесткость.
\\
\end{frame}
\lecturenotes
Всезнайка 
Такой деятель искренне верит в то, что ему известно все о его работе, о работе компании, о конкретных заданиях программистов. Он считает себя эрудитом. 
Диктатор 
Девиз этих деятелей – «я прав, потому что я прав». Диктатор пытается навязывать своим подчиненным конкретные методы решения поставленных перед ними задач. Диктатор создает вокруг себя атмосферу животного ужаса, которая не позволяет сотрудникам в полной мере проявлять свои способности, – все с трепетом ожидают очередного указания сверху и пытаются угадать последствия малейших проявлений независимости. 
Генерал
Руководитель этого типа хорошо усвоил уроки школы «командования и управления» в разработке программных средств. Генералы обнаруживают значительное сходство с диктаторами, проявляя в своей деятельности еще большую жесткость.
Кстати сказать, такой стиль руководства характерен для многих крупных программных проектов, разрабатываемых в гигантских корпорациях или в правительственных структурах. Именно в этой среде зародилась методика программной инженерии. 



\begin{frame} \frametitle{Неорганизованные руководители}
	\begin{itemize}
		\item Нехватка организационных навыков и практического мышления
		\item Присуща открытость и доброжелательность в общении
		\item Стремление избавиться от ответственности
		\item Способствование срыву срока сдачи проекта
		\item Внесение беспорядка в работу подчинённых
		\item Боязнь совершить ошибку
		
	\end{itemize}
\end{frame}
\lecturenotes
Неорганизованный руководитель обнаруживает недостаток организационных навыков, которые иногда, к тому же, сочетаются с нехваткой практического мышления. Такие деятели часто кажутся забавными, но стоит только возложить на них ответственность за что-либо, как они сразу стараются от нее избавиться. \\~\\
Последствия деятельности неорганизованных руководителей весьма разнообразны – в частности, они срывают сроки сдачи проектов и вносят в работу подчиненных полный хаос относительно конкретных задач. \\


\begin{frame} \frametitle{Неорганизованные руководители}
	 \alert{Скарлетт О'Хара}~— деятели, которым иногда кажется, что задачи слишком сложны; в иных случаях они не принимают никаких решений по той лишь причине, что боятся совершить ошибку.\\~\\
\alert{Временщик}~ крайне редко уделяет достаточное внимание сотрудникам и проектам. С одной стороны, эти люди хотят преуспеть и обеспечить тем самым профессиональный рост; с другой же – они обнаруживают неспособность к концентрации внимания, что не позволяет им достигать первой цели. 
\\~\\
\alert{Новичок}~— это неопытный руководитель, который вовсе не знает, с чего начать. Он испытывают серьезные проблемы по части расстановки приоритетов, поскольку все задачи кажутся им в равной степени сложными.
\\
\end{frame}
\lecturenotes


\begin{frame} \frametitle{Гений}
\begin{itemize}
		\item Благосклонность высшего руководства из-за их выдающихся способностей
		\item Исключительная способность к концентрации
		\item Главный акцент на технологии, отодвигая кадры на второй план
		\item Неспособность передать знания окружающим и осуществление менторства
		\item Обыкновение разрабатывать программные шаблоны для  подчиненных 
		\item Неготовность идти на компромисс				
	\end{itemize}
\end{frame}
\lecturenotes
Такой программист может весь день сидеть за клавиатурой и «строгать» прекрасный код, но в том, что касается передачи знаний окружающим, он полный ноль. Проводя проектное совещание, гений демонстрирует чудеса на электронном табло, но объяснить слушателям суть понятий, которые он на нем с такой легкостью иллюстрирует, у него не получается. Консенсус он понимает как единодушное согласие со своими предложениями. скажу лишь, что всем гениям свойственно одно и то же качество – исключительная способность к концентрации. 
Основной акцент они обычно делают на технологии, отодвигая кадры на второй план. Именно это обстоятельство мешает им стать полноценными лидерами. Спору нет, руководитель обязан обращать внимание на вопросы технологии, но основным приоритетом в его деятельности должны быть люди, которые работают с технологией. Гении в ранге руководителей имеют обыкновение разрабатывать для применения подчиненными целые программные шаблоны, которые, по их мнению, необходимо задействовать во всех без исключения продуктах. Зачастую такие шаблоны оказываются слишком сложными для применения в небольших проектах, однако гении, несмотря ни на что, настаивают на их применении, объясняя это необходимостью «поддерживать единую кодовую базу». Гении создают собственные шаблоны даже в тех случаях, когда продукты, способные выполнять те же функции и снабженные толковой документацией, присутствуют на рынке в большом количестве. 
Гении рассматривают администрирование как работу «для других», недостойную их внимания. Руководители высшего уровня, как правило, считают гениев прекрасными лидерами, объясняя свою точку зрения их выдающимися способностями. Чем больше наблюдатель удален от повседневных процессов, тем выше он склонен оценивать достижения гениев, не замечая их откровенных недоработок, происходящих от неэффективного распоряжения человеческими ресурсами и пренебрежения к административным процедурам. 



\begin{frame} \frametitle{Строители империй тьмы }
\begin{itemize}
		\item Основная цель~– подбором преданных подчинённых
		\item Отсутствует стремление реализовать цель компании
		\item Угождение собственным интересам ценится выше интересов компании
		\item Неготовность идти на взаимное соглашение
		\item Неприемлимость критики в свой адрес
		\item Склонность к чрезмерному карьеризму
	\end{itemize}
 
\end{frame}
\lecturenotes
Лидеров, относящихся к типу строителей империй, можно с равным успехом называть политиканами. Строители империй руководствуются в основном единственной целью – подбором преданных исполнителей их воли; при этом очень часто они забывают о необходимости претворять в жизнь цели компании. \\~\\
Поначалу они осыпают подчиненных разного рода поблажками и подарками, но цели построения сплоченной команды перед ними не стоит; они озабочены только одним – превращением своих сотрудников в роботов, беспрекословно исполняющих их предписания. Прислуживание они ценят значительно выше, чем производительность.




\begin{frame} \frametitle{Преимущества менеджера с~техническим бэкграундом}
	 \begin{itemize}
                      \item Понимание технической стороны работы
		\item Общение с командой на одном языке
		\item Реальная оценка трудоемкости задач
		\item Способность убедить заказчика в корректности поставленных менеджером сроков и задач
		\item Грамотное распределение задач в команде (особенно в отсутствие teamleader)
	\end{itemize} 	
\end{frame}
\lecturenotes



\begin{frame} \frametitle{Недостатки менеджера с~техническим бэкграундом}
	 \begin{itemize}
                      \item Оценивание сроков задач исходя из своего опыта без учета мнения команды
		\item Недостаточное понимание экономической стороны вопроса
		\item Выполнение чужих обязанностей при попытке помочь команде
		\item Нехватка знаний в терминологии друг друга
		\item Неспособность посмотреть на проект в целом
	\end{itemize} 	
\end{frame}
\lecturenotes
Менеджер сосредотачивается на исполнительской части, видит отдельные задачи\nobreakdash-«кирпичики», но недостаточно ясно оценивает перспективы
Создаваемый в ходе проекта, глазами потребителей, а для хорошего менеджера проекта это важно \nobreakdash- видеть пользовательские характеристики продукта, понимать, как их можно улучшить.
Есть риск отказа от хороших идей, потому что менеджер считает их реализацию слишком сложной, долгой и дорогой. В то время как менеджер-не разработчик мог бы ухватиться за эту идею, и другие участники команды нашли бы эффективный путь ее воплощения в жизнь.


\begin{frame} \frametitle{Преимущества менеджер с~экономическим бэкграундом}
	 \begin{itemize}
                      \item Грамотная оценка стоимости задач
		\item Общение с заказчиком на его языке
		\item Предоставление свободы действий (самостоятельности) в команде
		\item Видение целостного результата проекта
	\end{itemize} 	
\end{frame}
\lecturenotes



\begin{frame} \frametitle{Недостатки менеджера с~экономическим бэкграундом}
	 \begin{itemize}
                      \item Неправильная оценка реальной трудоемкости задачи и постановки сроков
		\item Перекладывание ответственности на плечи teamleader и команды
		\item Неграмотное распределение задач между членами команды
		\item Отсутствие технических знаний не позволяет увидеть риски проекта
	\end{itemize} 	
\end{frame}
\lecturenotes

~\cite{How_to_be_a_good_IT-manager}



\begin{frame} \frametitle{Ошибки начинающего менеджера}
\begin{itemize}	
		\item Совмещение функций менеджера и разработчика	
		\item Стремление выполнить всю работу самостоятельно
	 	\item Узкое мышление
	 	\item Уход от ответственности за работу всей команды
		\item Чрезмерный контроль над работой подчиненных
	 	\item Неправильное распределение обязанностей
	 	\item Постановка нереальных сроков для выполнения задач
 \end{itemize}
\end{frame}
\lecturenotes

~\cite{How_to_be_a_good_IT-manager}


\subsection{Ценность менеджера для компании}
\begin{frame} \frametitle{Ценность менеджера для компании}
 \begin{block}{Успешный менеджер:}
\begin{itemize}
  \item Укладывается в сроки
  \item Учитывает интересы не только заказчика, но и подчиненных 
  \item Способствует личностному росту команды
  \item Обладает большим количеством успешных проектов
  \item Имеет положительные отзывы от клиентов
  \item Поощряет и вознаграждает сотрудников
  \item Может предугадать возможное развитие событий в проекте

  \end{itemize}
 \end{block}
\end{frame}
\lecturenotes


\section{Организация рабочего процесса IT-менеджера}

\subsection{Задачи IT-менеджера}
\begin{frame} \frametitle{Задачи IT-менеджера}
\begin{itemize}	
		\item Управление персоналом
		\item Расчет бюджета информационной среды
	 	\item Анализ удовлетворенности клиентов
	 	\item Разработка и обеспечение качественных IT-услуг
		\item Переговоры с клиентами и поставщиками
	 	\item Поиск и оценка инновационных технологий, предложение их внедрения
	 	\item Определение сроков работы над IT-проектами
	           \item Реализация IT-стратегии компании
		\item Организация взаимодействия IT-специалистов с другими работниками
 \end{itemize}
\end{frame}
\lecturenotes

~\cite{How_to_be_a_good_IT-manager}


\begin{frame} \frametitle{Ежедневные обязанности менеджера}
{\includegraphics[width=1\linewidth]{day.png}}
\end{frame}
\lecturenotes


\begin{frame} \frametitle{Повседневные проблемы менеджера}
{\includegraphics[width=1\linewidth]{problem.png}}
\end{frame}
\lecturenotes

\subsection{Планирование процесса}
\begin{frame} \frametitle{Планирование процесса}
Когда принято решение о разработке нового продукта, то на плечи менеджера ложится задача спланировать весь процесс работы над будущим проектом. Это касается таких вещей, как:
\begin{itemize}	
		\item Как будет выглядеть проект
		\item Сколько времени понадобится на проект
	 	\item Сколько именно средств придется выделить на проект, в каких местах можно сэкономить
	 	\item Какие специалисты будут задействованы в работе над проектом
		\item Как именно следует распределить задачи в команде
	 	\item Сколько будет стоить продукт, кто и как его будет продвигать на рынке

 \end{itemize}
\end{frame}
\lecturenotes

~\cite{How_to_be_a_good_IT-manager}

\subsection{Встреча с клиентом}
\begin{frame} \frametitle{Встреча с клиентом}
Для менеджера, проводящего первую встречу с возможным клиентом, основными целями являются:
\begin{itemize}	
		\item Оценка перспектив и готовности клиента к сотрудничеству
		\item Определение главных потребностей собеседника
	 	\item Выяснение, какие из товаров и услуг, предлагаемых вашей компанией, могут быть ему полезны и интересны
	 	\item Предложение пробников продуктов, показ образцов работ
		\item Получение обратной связи от клиента

 \end{itemize}
\end{frame}
\lecturenotes

~\cite{How_to_be_a_good_IT-manager}

\begin{frame} \frametitle{Ошибки при встрече с клиентом}
Менеджеры, выезжающие на личные встречи с клиентами – даже самые опытные и толковые – иногда ошибаются. Наиболее типичны следующие ошибки:
\begin{enumerate}

		\item Сразу презентовать свой товар (услугу), пропустив этапы вхождения в контакт и узнавания актуальных потребностей клиента. 
		\item Непонимание главной потребности клиента, смещение акцента внимания на менее важные нужды и проблемы, о которых он упомянул.
	 	\item Выражение неудовольствия действиями клиента (опозданием, переносом встречи, отказом и т. д.).
		\item Необдуманное предложение бонусов, скидок и других льготных условий для клиента.
	 
\end{enumerate}
\end{frame}
\lecturenotes

~\cite{How_to_be_a_good_IT-manager}


\begin{thebibliography}{99}
\bibitem{How_to_be_a_good_IT-manager} \href{http://www.pvsm.ru/upravlenie-proektami/36476}
\bibitem{Managers_in_IT} \href{https://habrahabr.ru/post/219741/}
\bibitem{Managers_thinking_style} \href{http://www.hr-asteri.ru/employer/poleznaya_informaciya/stil_myshleniya_i_povedeniya_professionalnogo_rukovoditelya/}
\bibitem{Programmer_vs_manager} \href{https://dou.ua/lenta/articles/programmer-vs-manager/}
\bibitem{Best_qualities_for_IT-manager} \href{http://hr-portal.ru/article/kakie-kachestva-nuzhny-menedzheru-it-proektov}
\bibitem{From_engineer_to_manager_keeping_your_technical_skills} \href{https://hackernoon.com/from-engineer-to-manager-keeping-your-technical-skills-40579cc8ea00}
\bibitem{From_programmer_to_manager} \href{https://m.dotdev.co/what-i-learned-transitioning-from-being-a-programmer-to-an-it-manager-8e58e7b406}
\end{thebibliography}

\end{document}

%%% Local Variables: 
%%% mode: TeX-pdf
%%% TeX-master: t
%%% End: 
