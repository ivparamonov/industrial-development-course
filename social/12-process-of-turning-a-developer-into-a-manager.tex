\documentclass{../industrial-development}
\graphicspath{{template/}}

\title{Процесс превращения разработчика в менеджера}
\author{Харчев Дмитрий Михайлович, ИВТ-21 МО}
\date{}

\begin{document}

\begin{frame}
  \titlepage
\end{frame}

\section{Разработчик и менеджер}

\subsection{Процесс превращения разработчика в менеджера}

\begin{frame} \frametitle{Менеджер в IT}
	Чтобы стать хорошим IT специалистом, нужен лишь компьютер, интернет и желание; хорошим менеджером без практики стать вряд ли получится
\end{frame}
\lecturenotes
Думаю, многие согласятся с тем, что у нас очень много хороших IT специалистов. И даже наше образование, а также политические, экономические и другие факторы не сильно могут помешать стать хорошим IT специалистом при наличии желания. Но вот с менеджерами в сфере IT не всё так хорошо…

Под IT менеджерами я имею в виду тимлидов, руководителей проектов, начальников отделов, скрам мастеров, лайн менеджеров и т.д. С топ менеджментом дела выглядят получше, так как их либо присылают из-за бугра, либо это представители редкого вида управленцев-технарей, которые иногда водятся в дикой IT среде.
~\cite{How_to_be_a_good_IT-manager}


\begin{frame} \frametitle{Качества хорошего IT менеджера}
	\begin{itemize}
		\item хорошее образование
		\item особый склад мышления
		\item опыт работы(практика)
		\item нести ответственность за подчиненных
	\end{itemize}	
\end{frame}
\lecturenotes
Какими свойствами должен обладать хороший менеджер? Во-первых, он должен уметь коммуницировать с различными людьми, а также ими руководить. Во-вторых, быть хорошо подкованным в смежных областях — психологии, экономике, юриспруденции, разбираться в таких вопросах, как мотивация, моральный риск, оценка рисков, спрос-предложение, трудовое законодательство, международное право, интеллектуальная собственность.
Эти предметы номинально преподают в технических университетах, но их количество и качество стремятся к нулю. Таким образом, университетское образование никак не решает вопрос воспитания IT менеджеров.
Так где же брать IT менеджеров? Из смежных областей менеджеров не возьмешь — у них банально не хватит квалификации.
Остается единственный вариант — выращивать менеджеров из разработчиков. А это имеет несколько проблем, одна из которых — нужно менять мышление.

У менеджеров и программистов есть очень важное отличие в мышлении: программист нацелен на процесс — для него является важным создать изящную архитектуру, написать идеальный код с комментариями, придумать хитрую систему плагинов и писать на новых (или любимых) технологиях. Менеджер, в свою очередь, нацелен (должен быть нацелен) на результат — готовый продукт в обозначенные сроки и бюджет. При этом красота кода отходит на второй (я бы даже сказал, на предпоследний) план. Из-за этого несоответствия не редки проблемы между менеджерами и разработчиками.
Проблема в том, что изменить мышление за день или неделю невозможно. По экспертным оценкам, процесс транформации мышления занимает от полугода до четырёх лет.

Вторая проблема в том, что редкий разработчик будет смотреть наперед на 3-4 года и станет планировать своё менеджерское будущее… без внешнего вмешательства. К сожалению, HR больше обеспокоены, как захантить очередного джависта и закупить новую партию печенек, чтобы хотя бы эти не разбежались.
Также нужно помнить, что хорошим менеджером может стать далеко не каждый технарь. Схема «сегодня — лучший разработчик, завтра — хороший менеджер» не работает. Можно потерять хорошего разработчика и получить плохого менеджера. Экономический эффект от этого даже не стоит начинать считать.
Лучший момент для начала трансформации разработчика в менеджера — middle уровень. В это время нужно начинать читать специализированную литературу (а не только литературу по технологиям). Лучше всего начать с Peopleware Тимоти Лестера и «Джоэл о программировании» Джоэла Спольски. 

Еще один момент, которые многие упускают. В любых ситуациях за всё, что происходит в команде разработчиков, ответственность должен нести менеджер (руководитель).
В век скрама и гибких методологий ответственность размазывается среди членов команды, руководителей и заказчиком, то есть отсутствует персональная ответственность. Не нужно говорить, что это очень выгодно компаниям и самим разработчикам.
Пример из жизни: как-то задал вопрос своему менеджеру, кто возьмет на себя ответственность за нереальную оценку, отсутствие квалифицированных кадров и внедренную методологию управления под названием scrumно. На что получил удивленный ответ: все. При персональной ответственности за свою работу ничего подобного бы ни было. Думаю, не нужно говорить, что проект с треском провалился, и никто при этом не понёс персональное наказание.
Понимать и нести персональную ответственность за себя и своих подчиненных — это одно и самых важных свойств хорошего менеджера.
Также несколько раз наблюдал ситуацию, когда менеджеры через некоторое время отказывались от своих должностей и обратно превращались в разработчиков. Причин несколько: рост зарплаты, как правило, незначительный (если вообще есть), а обязанностей и головной боли — на порядок больше.
К тому же ситуация, когда разработчик зарабатывает больше своего менеджера, — не редкость в IT компаниях, хотя в других сферах такое случается крайне редко. Этим также можно объяснить нежелание многих технарей превращаться в менеджеров.
~\cite{How_to_be_a_good_IT-manager}

\begin{frame} \frametitle{Когда менеджер "--- это плохо}
	\begin{block}{Признаки плохого менеджера}
	 \begin{itemize}
	 	\item первое правило плохого менеджера: не знаешь, что делать "--- организуй митинг
	 	\item плохая подготовка(образование)
	 	\item недостаток опыта
	 \end{itemize}
 	\end{block}
\end{frame}
\lecturenotes
Причина того, что многие ИТ-менеджеры не очень компетентны, заключается в том, что у нас нет полноценного образовательного института, который готовил бы менеджеров с ИТ-уклоном. Поэтому в менеджеры часто идут либо очень плохие, либо очень хорошие разработчики, либо те, кто в ИТ никогда не работали, но хорошо говорят на английском. Во всех этих случаях наличие менеджера на проекте, скорее, вредит проекту, чем помогает. 
Почему никто не готовит менеджеров? Давайте займемся арифметикой. Чтобы из айтишника сделать менеджера нужно от 0.5 года (практически идеальный и редко встречающийся в дикой ИТ-природе случай) до четырех лет. Средняя продолжительность работы айтишника в компании – год/полтора. Таким образом, если компания начнет вкладывать деньги в процесс трансформации “айтишник-менеджер”, то с большой долей вероятности трудами этого обучения воспользуются конкуренты. Поэтому компании с удовольствием обучают айтишников на курсах повышения квалификации, но не готовы вкладывать в обучение менеджеров.
Что же делать айтишнику в таком случае? Во-первых, решить, хочет ли он в будущем быть менеджером. Если ответ положительный, то, начиная где-то с уровня мидл, нужно параллельно с книгами по технологиям начинать читать книги по психологии, риск-менеджменту и проджект-менеджменту. 
~\cite{Managers_in_IT}

\begin{frame} \frametitle{Когда менеджер "--- это хорошо}
	\begin{block}{Признаки хорошего менеджера}
		\begin{itemize}
			\item быть стеной между заказчиками и исполнителями
			\item брать на себя ответственность за весь проект
		\end{itemize}
	\end{block}
\end{frame}
\lecturenotes
Как-то одному нашему знакомому, будучи еще довольно молодого возраста, довелось выступить в роли “23-летнего сеньора” (а еще и в роли тим-лида) для одного очень серьезного заказчика (Fortune 100). Проект был не сложный технически, но непростой с точки зрения интеграции и требований к аппаратной среде. Для интеграции была выделена отдельная машина (на стороне заказчика), где всё благополучно запустилось. Но инженерам заказчика этого показалось мало и они решили развернуть систему на персональном ноутбуке, а также на рабочем компьютере. И в какой-то момент система не завелась. Пришлось долго разбираться почему, причем проблемы возникали одна за другой. 
Это могло продолжаться довольно таки долго, если бы в ситуацию не вмешался менеджер. Он запретил инженерам заказчика давать исполнителю задания без его ведома и разворачивать систему на не проверенном железе. После этого вмешательства ситуация устаканилась, разработчика перестали дергать, а проект был благополучно сдан спустя несколько дней.

Стоит запомнить одну простую мысль: программист кодит, менеджер отвечает. Если программист плохо накодил – в любом случае виноват менеджер, ведь он не проследил за ходом выполенения проекта и/или нанял не подходящего сотрудника. 
Да, менеджер получает больше плюшек, но и ответственность у него на порядок выше.
~\cite{Managers_in_IT}

\begin{frame} \frametitle{Когда менеджер "--- это хорошо}
	\begin{block}{Признаки хорошего менеджера}
		\begin{itemize}
			\item гарантировать достижение цели в условиях ограниченного доступа к ресурсам
			\item быть мотиватором и следить за развитием подчиненных
		\end{itemize}
	\end{block}
\end{frame}
\lecturenotes
Причем ему нужно гарантировать достижение целей как заказчика, так и исполнителей. Это удается редко. Ведь менеджер сталкивается с миллионом различных проблем: отсутствием и недостаточным количеством кадров, времени и бюджета, злым или не компетентным заказчиком, отсутствием экспертизы и т.д. 
Но если менеджер может гарантировать результат, то не важно, где, сколько и как он работает. Более того, менеджер, который будет работать по 7-8 часов в день – гарантированно завалит проект. Менеджеру должны платить за его решения и конечный результат. В идеале менеджер должен получать основной доход в виде бонусов за вовремя выполненные проекты, а не за количество просиженных в linkedin или skype часах. 

Если бы разработчики нас спросили, в какой компании лучше всего работать, мы бы без раздумий ответили: в той, где у них будет адекватный менеджер (даже если денег там будут платить вдвое меньше). 
В одной из компаний, программист получил большой кредит доверия от своего менеджера. Доверие заключалось в полном одобрении принятых им (программистом) решений, автономность работы, помощи при организации встреч разработчиков, учебных курсов и поездок на всевозможные конференции. Результат не заставил себя долго ждать: был открыт новый отдел на 5-6 человек; знания, полученные на мероприятиях, были внедрены в рабочий процесс, на проектах начали использовать современные технологии и иструменты. Итог? Все предложения других компаний (включая предложения с более высокой зарплатой), на протяжении года-полтора были отклонены программистом без капли сожаления.
~\cite{Managers_in_IT}

\begin{frame} \frametitle{Стиль мышления и поведения профессионального IT менеджера
}
\begin{block}{Работа профессионального менеджера:}
	\begin{itemize}
		\item результат
		\item мотивация
		\item команда
		\item система
		\item лидерство
	\end{itemize}
\end{block}
\end{frame}
\lecturenotes
В этом секторе находятся менеджеры, ориентированные на цель. Если вы зададите таким менеджерам вопрос: «Что вам больше всего нравится в вашей работе?», они преимущественно отвечают кратко: «Достигать поставленные цели». Большую часть времени эти менеджеры уделяют постановке и контролю выполнения задач. Они очень жесткие руководители, так как требуют от людей выполнения поставленных задач. Если персонал не выполняет поставленные цели и задачи, то они способны быстро принять решение об увольнение данных сотрудников.
Эти руководители мышлением и поведением похожи на спортсменов — любая цель является достижимой, для этого необходимо приложить максимум своих усилий. Такие руководители требовательны к себе и требовательны к своему персоналу. Персонал они воспринимают как ресурс для достижения своих целей. У таких руководителей доминирующим вопросом является: «Что я хочу?»

В этом секторе находятся менеджеры, ориентированны на других людей. Если вы зададите таким менеджерам вопрос: ««Что вам больше всего нравится в вашей работе?», они преимущественно отвечают: «Мне нравится работа с людьми, нравится мотивировать и поддерживать сотрудников, которые работают в нашей организации. Каждый человек индивидуален, и к каждому необходим свой ключик. Мне нравится находить подход к каждому человеку».
Большую часть своего времени они уделяют развитию персонала и мотивации в достижении поставленных целей. Они являются лояльными руководителями. Если персонал не выполняет поставленные цели, они ищут проблемы в мотивах поведения данного сотрудника. Они очень переживают любое увольнение персонала из организации и долго не могут принять решение об увольнении сотрудника. У таких руководителей доминирующим вопросом является: «Как найти подход к каждому человеку, чтобы им помочь достичь поставленных целей?»

В этом секторе находятся менеджеры, которые ориентированы на создание комфортного психологического климата среди сотрудников. Если вы зададите таким менеджерам вопрос: «Что вам больше всего нравится в вашей работе?», они преимущественно отвечают: «Это работа в команде. Это разрешение конфликтов среди сотрудников». Они считают, что залогом успеха организации является команда. Такие менеджеры часто привлекают персонал к принятию решения. Если кто-то из персонала не выполняет поставленные цели и задачи, они прибегают к коллективному принятию решения относительно данных сотрудников. Такие руководители являются сердцем своей команды. Подчиненные отзываются о таком руководителе, как о душе команды. У таких руководителей доминирующим вопросом является: «Как создать команду единомышленников, чтобы достичь поставленных целей?»

В этом секторе находятся менеджеры, которые ориентированы на создание системы продаж и системы возврата денег.

Если вы зададите таким менеджерам вопрос: «Что вам больше всего нравится в вашей работе?», они преимущественно отвечают: «Создание системы продаж и системы возврата денег, которая менее зависима от человеческого фактора». Такие менеджеры обладают аналитическим складом ума, которые стараются перевести весь процесс работы организации в «конвейер», где организация меньше подвластна влиянию человеческим фактором, где каждый работник является элементом, который можно быстро и эффективно заменить. Здесь уместно привести фразу из кинофильма «Матрица»: «Человек — это батарейка».
У таких руководителей доминирующим вопросом является: «Как построить высокоэффективную систему работы предприятия, с помощью которой можно достигать поставленные цели?»
~\cite{Managers_thinking_style}

\begin{frame} \frametitle{Основные задачи IT менеджера}
	\begin{itemize}
		\item быстро решать входящие задачи
		\item планировать ресурсы
		\item постановка целей проекта, разработка метрик для измерения эффективности работы (достигнуты цели или нет)
	\end{itemize}
\end{frame}
\lecturenotes
У менеджера есть достаточно абстрактная задача и некоторые ресурсы, которые надо использовать максимально рационально. Задача может быть хорошо детализирована, но всё равно проект приобретает конечную форму не в процессе проектирования, а в процессе разработки, и всегда всплывают недостающие детали.

~\cite{Best_qualities_for_IT-manager}

\begin{frame} \frametitle{Основные задачи IT менеджера}
	\begin{itemize}

		\item управление участниками проекта, координация
		\item управление бюджетами
		\item оценка и управление рисками
	\end{itemize}
\end{frame}
\lecturenotes
Часто задача ставится в стиле «сделайте черт-те что, вот roadmap». И его наличие — хороший начальный признак: roadmap вообще — редкий зверь. Еще меня очень смущает, когда просят оценку сроков, исходя из концепта на несколько листов. Но об этом можно поговорить отдельно.
В начале, когда у проектного менеджера есть хоть какие-то ресурсы, он начинает распределять работу среди людей. И превращение задачи в конкретный тикет обычно требует уточнений. Те, кто говорит, что можно сразу хорошо распланировать работу, — либо врут, либо им плевать на проект и результат, а к работе они подходят формально (еще бывает, что менеджер слишком туп, чтобы понять, о чем речь). Ситуация, когда клиент просит то, что ему не нужно, — тоже не редкость. В общем, приходится все время общаться и с клиентом, программистами и остальными нужными людьми.
Если перефразировать, то, помимо планирования, задача продуктового менеджера — это реагирование на входящие сигналы и эффективное распределение ресурсов.
~\cite{Best_qualities_for_IT-manager}

\begin{frame} \frametitle{Должен ли менеджер проектов быть разработчиком?}
	\begin{block}{Аргументы «за»:}
		\begin{itemize}
			\item технически подкованный менеджер умеет говорить с разработчиками на их языке
			\item он способен грамотно ставить задачи и понимать реальные сроки их реализации
		\end{itemize}
	\end{block}

\end{frame}
\lecturenotes
~\cite{Best_qualities_for_IT-manager}

\begin{frame} \frametitle{Должен ли менеджер проектов быть разработчиком?}
	\begin{block}{Аргументы «против»:}
		\begin{itemize}
			\item менеджер\nobreakdash-разработчик не всегда способен посмотреть на проект в целом
			\item разработчик не всегда умеет посмотреть на продукт в целом
			\item у менеджера-разработчика может срабатывать фильтр «реализуемо\nobreakdash-нереализуемо»
		\end{itemize}
	\end{block}
\end{frame}
\lecturenotes
Менеджер сосредотачивается на исполнительской части, видит отдельные задачи\nobreakdash-«кирпичики», но недостаточно ясно оценивает перспективы
Создаваемый в ходе проекта, глазами потребителей, а для хорошего менеджера проекта это важно \nobreakdash- видеть пользовательские характеристики продукта, понимать, как их можно улучшить.
Есть риск отказа от хороших идей, потому что менеджер считает их реализацию слишком сложной, долгой и дорогой. В то время как менеджер-не разработчик мог бы ухватиться за эту идею, и другие участники команды нашли бы эффективный путь ее воплощения в жизнь.
~\cite{Best_qualities_for_IT-manager}

\begin{frame} \frametitle{Почему менеджеры должны быть технически компетентными}
Компетентный технический менеджер должен:
\begin{itemize}
	\item уметь понимать скрытую сложность задачи
	\item быть наставником своей команды
	\item быть в курсе новых програм/фреймворков
	\item ваша команда должна доверять вам и уважать вас
\end{itemize}
\end{frame}
\lecturenotes
Менеджер должен уметь понимать скрытую сложность, связанную с задачами, которые должна выполнять ваша команда. Это упрощает общение с членами команды и нетехнических заинтересованными сторонами, такие как отделы продукта или маркетинга. 
Необходимо быть эффективным наставником для своей команды разработчиков и уметь показать пример в технической области.
Хорошему менеджеру необходимо следить за новостями в области разработки ПО: искать новые программы, фреймворки и даже новые языки программирования.
~\cite{From_engineer_to_manager_keeping_your_technical_skills}

\begin{frame} \frametitle{Знания, которые необходимы программисту, чтобы стать менеджером}
\begin{itemize}
	\item освоить методологию ведения проектов
	\item освоить методологии работы с проектами именно в сфере программного обеспечения 
	\item освоить основы менеджмента, маркетинга, финансовой деятельности
	\item «прокачать» коммуникативные навыки
\end{itemize}
\end{frame}
\lecturenotes
Методология ведения проектов, например, PMBOK (Project Management Body of Knowledge) или PRINCE2 (PRojects IN Controlled Environments 2);
Методологии работы с проектами именно в сфере программного обеспечения (например, Agile, Lean);
Коммуникативные навыки исключительно важны при общении с участниками команды, инвесторами, клиентами;
Лидерские качества очень важны, менеджер проекта – не просто «технический» координатор. Он – лидер, способный увлечь своими идеями и мотивировать участников проекта на достижение целей.
~\cite{Best_qualities_for_IT-manager}

\begin{frame} \frametitle{Знания, которые необходимы программисту, чтобы стать менеджером}
	\begin{itemize}
		\item улучшить (или наработать) навыки презентации
		\item освоить навыки составления и анализа проектной документации
		\item освоить основы менеджмента, маркетинга, финансовой деятельности
		\item усилить до максимума лидерские качества
	\end{itemize}
\end{frame}
\lecturenotes
\begin{frame} \frametitle{Кодер или Наставник}
	Необходимо понять, что IT менеджеру можно возвращаться к работе с кодом только тогда, когда у него появляется свободное время
\end{frame}
\lecturenotes
Первое испытание с которым вы вероятнее всего столкнетесь это конфликт между вашей кодерской сущностью и сущностью ментора: Ваша кодерская сущность захочет решать все интересные технические задачи, с которымы столкнется ваша команда. С другой стороны, ваша менторская сущность должна отдавать эти интересные задачи менее опытным членам команды, которые могут поднять свой уровень программирования и улучшить свои навыки разработки.
Проблема может возникнуть в командах, где много junior-разработчиков и малое количество senior-разработчиков, чтобы их наставлять. В этом случае инженер-менеджер должен будет уделять еще больше внимания аспекту наставничества ее роли.
~\cite{From_engineer_to_manager_keeping_your_technical_skills}

\begin{frame}\frametitle{Сделайте свою команду успешной!}
	\begin{itemize}
		\item только когда ваша команда успешна, вы можете посвятить некоторое время кодированию
		\item в качестве менеджера вы не создаете продукт, вы создает команду, которая знает, как создавать продукты
	\end{itemize}
\end{frame}
\lecturenotes
Ваша команда и ваши люди важны в первую очередь, перед каким-либо кодированием. Это необходимое условие: Тогда и только тогда. Если ваша команда успешна, вы можете посвятить некоторое время кодированию.
Как менеджер вы не создаете продукт, вы создаете команду, которая знает, как создавать продукты. Таким образом, ваша команда должна быть вашей основной целью и метрикой, чтобы оценить себя. Если вы возглавите успешную команду счастливых инженеров, успех команды будет говорить сам по себе, и вам придется тратить меньше времени на собрания, пытаясь объяснить или продать свои достижения своим сверстникам и начальству.
Самодостаточная команда, в которой вы не являетесь узким местом решения, даст вам дополнительную пропускную способность, чтобы заняться чем-то другим.
~\cite{From_engineer_to_manager_keeping_your_technical_skills}

\begin{frame}\frametitle{Отличия стиля мышления и характера работы разработчика и менеджера}
	\begin{itemize}
		\item управление временем
		\item обучение
		\item удовольствие от решения задач
		\item влияние
	\end{itemize}
\end{frame}
\lecturenotes
Как разработчик, вы полностью контролируете свое время. У вас есть задание на, вы должны указать, сколько дней вам нужно для его исполнения. По иронии судьбы, у менеджеров меньше возможностей планировать свое время, поскольку они являются заложниками запросов и сроков, за которые они несут ответственность, которые также зависят от сотрудников менеджера, других команд, заказчиков.

Чтобы научиться быть хорошим разработчиком есть множество книг и уроков. Это увлечение, которое вы можете практиковать в свое свободное время, создавая множество личных проектов, начиная от маленьких приложений до покерных ботов. Вы можете научиться с помощью этих проектов, которые очень помогут вашей работе. Но вы не можете практиковать себя менеджером самостоятельно. Очень трудно научиться эффективному управлению через обучение - нет никаких побочных проектов, которые вы можете сделать сами, вы должны учиться на работе.

Некоторые получают больше удовольствия, когда решают задачи с помощью кодирования, но не многие ценят это или волнуются об этом - они больше волнуются о результатах,но не о интеллектуальном мастерстве, которое было использовано в построении данного решения. Как менеджер, вы получаете гораздо больше достижений, о которых вы даже не думали. Это другое чувство - примерно как разница между удовольствием, которое вы получаете, ремонтируя свой собственный автомобиль по сравнению с тем удовольствием, которое вы получаете, когда ваша компания выигрывает награду «лучший работодатель».
~\cite{From_programmer_to_manager}

\begin{frame}\frametitle{Отличия стиля мышления и характера работы разработчика и менеджера}
	\begin{itemize}
		\item креативность
		\item работа с людьми
		\item планирование
	\end{itemize}
\end{frame}
\lecturenotes
Как разработчик вы можете работать над решением творческих технических проблем, которые, скорее всего, оцените только вы. В качестве менеджера вы решаете более крупные проблемы, которые включают в себя технологию как часть решения (иногда это вовсе не так).

Это очевидно и часто клише, но все же стоит сказать. Как разработчик вам нужно сотрудничать с другими, но не обязательно должны быть подотчетны другим. Как менеджер, вы не только должны оттачивать навыки общения, чтобы управлять своей командой, но и важными внешними заинтересованными сторонами (спонсорами проекта, клиентами, руководителями, юридическими группами, маркетингом и т.д.),

Планирование также очень важно. Как разработчик, вы, как правило, планируете разработку в соответствии со сроками. В качестве менеджера вам нужно планировать все: общую дату поставки, зависимость от других проектов, сроки работы, бюджетирование, цикл планирования численности персонала и т.д.
~\cite{From_programmer_to_manager}


\begin{thebibliography}{99}
\bibitem{How_to_be_a_good_IT-manager} \href{http://www.pvsm.ru/upravlenie-proektami/36476}
\bibitem{Managers_in_IT} \href{https://habrahabr.ru/post/219741/}
\bibitem{Managers_thinking_style} \href{http://www.hr-asteri.ru/employer/poleznaya_informaciya/stil_myshleniya_i_povedeniya_professionalnogo_rukovoditelya/}
\bibitem{Programmer_vs_manager} \href{https://dou.ua/lenta/articles/programmer-vs-manager/}
\bibitem{Best_qualities_for_IT-manager} \href{http://hr-portal.ru/article/kakie-kachestva-nuzhny-menedzheru-it-proektov}
\bibitem{From_engineer_to_manager_keeping_your_technical_skills} \href{https://hackernoon.com/from-engineer-to-manager-keeping-your-technical-skills-40579cc8ea00}
\bibitem{From_programmer_to_manager} \href{https://m.dotdev.co/what-i-learned-transitioning-from-being-a-programmer-to-an-it-manager-8e58e7b406}
\end{thebibliography}

\end{document}

%%% Local Variables: 
%%% mode: TeX-pdf
%%% TeX-master: t
%%% End: 
