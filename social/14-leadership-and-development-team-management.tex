\documentclass{../industrial-development}
\graphicspath{{14-leadership-and-development-team-management/}}

\title{Лидерство руководителя и управление командой разработчиков}
\author{Петров Алексей Анатольевич, ИВТ-21 МО}
\date{}

\begin{document}

\begin{frame}
  \titlepage
\end{frame}

\section{Лидерство руководителя и управление командой разработчиков ПО}

\begin{frame} \frametitle{Кто такой лидер?}
 Лидерство - способность оказывать влияние как на~отдельную личность, так и на группы, направляя усилия всех на достижение целей организации.
\end{frame}

\lecturenotes

Лидерство --- способность оказывать влияние как на~отдельную личность, так и на группы, направляя усилия всех на достижение целей организации. В переводе с английского лидер означает «руководитель», «командир», «глава», «вождь», «ведущий». Группа, решающая значимую проблему, всегда выдвигает для ее решения лидера. Без лидера ни одна группа существовать не может. Лидера можно определить как личность, способную объединять людей ради достижения какой-либо цели. Понятие «лидер» приобретает значение лишь вместе с понятием «цель». Действительно, нелепо бы выглядел лидер, не имеющий цели.
Но иметь цель и достичь ее самостоятельно, в одиночку -- недостаточно, чтобы назваться лидером. Неотъемлемым свойством лидера является наличие хотя бы одного последователя. Роль лидера заключается в умении повести людей за собой, обеспечить существование таких связей между людьми в системе, которые способствовали бы решению конкретных задач в рамках единой цели. Т. е. лидер --- это элемент упорядочивания системы людей. лидерство персонал потенциал.


\begin{frame} \frametitle{Качества лидера}
  \begin{enumerate}
  \item Инициативность
  \item Стратегическое мышление
  \item Налаживание контактов
  \item Навигаторские качества
  \item Построение команды
  \item Принятие рисков  
  \end{enumerate}
\end{frame}

\lecturenotes

Инициативность — это качество отражает мощное побуждение к личному профессиональному росту и совершенствованию.
Стратегическое мышление — способность учитывать широкий спектр факторов при принятии решений и оценке деятельности.
Налаживание контактов — способность выпестовать и поддержи­вать систему внешних и внутренних взаимоотношений в целях расширения горизонтов бизнеса и достижения нужных целей.
Навигаторские качества — способность проложить для компа­нии курс, который приведет ее к успеху, и сделать свое видение достоянием всех членов коллектива, заражая их энтузиазмом и энергией для движения в избранном направлении.
Умение содействовать развитию других — стать ментором и на­ставником для членов команды, развивать у них лидерские каче­ства, делегировать полномочия, поощрять к решению дерзких задач.
Построение высокоэффективных команд — подбирать состав уча­стников и максимизировать эффективность команд; создавать команды, ориентированные на достижение организационных целей.
Умение действовать на результат — стремиться преодолевать барьеры и препятствия, не считая их непреодолимыми; постоян­но подстегивать себя и других добиваться значимых результатов.
Принятие рисков — смело брать на себя риски, тщательно про­считанные и взвешенные, ради достижения поставленных целей бизнеса; готовность изменить укоренившееся положение дел и традиционные методы в поисках новых путей к успеху.
Решительный ум — способность принимать жесткие решения и наделение такими полномочиями членов команд; не бояться про­блем и трудностей, преодолевать их немедленно и решительно.
Умение правильно общаться и слушать — демонстрировать пре­восходные навыки устной и письменной речи; прежде чем от­ветить на возражение, не только выслушивать, но и понять точку зрения и побуждения оппонентов.

\section{Понятие лидерства и отличие между руководителем, и лидером.}

\begin{frame} \frametitle{Типы лидерства}
 	  \begin{block}{Формальный лидер}
           Управляет людьми согласно действующим положениям и~инструкциям.
	  \end{block}

	  \begin{block}{Харизматический лидер}
          Получивший власть над людьми естественным путем благодаря своим личностным качествам.
	  \end{block}
\end{frame}

\lecturenotes

По различным признакам — в зависимости от способов утверждения лидерской позиции и легитимации власти, а также в соответствии со степенью инициативности и харизматичности — можно выделить несколько основных типов лидеров.
Формальный лидер — "назначенный сверху", управляющий людьми согласно действующим положениям и инструкциям. Стараясь усилить свое влияние и опасаясь потерять власть, осознанно принижает статус подчиненных. Удержаться "на плаву" способен за счет умения неторопливо преодолевать "марафонские дистанции", за счет собственной компетенции в той или иной сфере деятельности и благодаря прочным связям с "верхами".
Харизматический лидер — получивший власть над людьми естественным путем благодаря своим личностным качествам.
Хаизматические лидеры, в силу своих личных качеств, спо­собны оказывать на своих последователей сильное и экстраор­динарное воздействие. Люди идентифицируют себя с таким лидером и с готовностью идут за ним. Имеет ярко выраженную индивидуальность, собственную систему ценностей, устанавливает свои правила и следует им невзирая ни на что. В основе его лидерства — эмоции и вера людей в его выдающиеся способности. Отождествление с ним дает людям уверенность и ощущение сопричастности. Харизматический лидер наиболее эффективен в кризисные моменты.

\begin{frame} \frametitle{Виды лидерства}
  \begin{itemize}
  \item Традиционный лидер
  \item Командный лидер
  \item Репрезентативный лидер
  \item Каталитический лидер
  \end{itemize}
\end{frame}

\lecturenotes

Кроме типов лидерства, можно также выделить несколько основных видов:

Традиционный лидер — получивший власть по наследству, которая узаконена вековыми или ставшими священными традициями. Обычно это монархи, религиозные лидеры, вожди племен. В основе такого лидерства лежит привычка людей к исторически сложившемуся типу власти.
Командный лидер — приходит к власти "силовым путем", умело настроив общество против своих конкурентов. Взяв бразды правления в свои руки, указывает людям, что нужно делать. Использует свои идеи. Действует авторитарными методами, не зависит от чужих мнений, обычно сам создает новое общественное, социальное или политическое движение. Результативен. Если командный лидер переоценивает свои возможности и свое влияние или демонстрирует явные черты деспотизма, это неизбежно приводит к внешнему конфликту: люди выходят из повиновения и восстают против него.
Репрезентативный лидер — наделенный властью теми, чьи пожелания он должен выполнять. Власть ему, по существу, делегирована теми, кого он представляет. Возможности такого лидера ограничены, потому он сдержан, никогда не идет на риск, стараясь действовать наверняка. Работает на основе явно выраженных пожеланий других людей. Личность, как правило, заурядная, лишенная "искры Божьей".
Каталитический лидер — пришедший к власти благодаря тому, что сумел уловить и выразить невысказанные идеи и чаяния группы людей или общества в целом. Обладает тонкой интуицией, острой восприимчивостью, умением распознавать и четко формулировать наметившиеся тенденции развития общества. Начинает действовать до того, как сложится общественное мнение, таким образом ускоряя процесс социальной эволюции, но не изменяя при этом обусловленности ее направления.

\begin{frame} \frametitle{Различия между лидером и руководителем}
Лидерство и руководство рассматриваются как групповые процессы, связанные с социальной властью в группе.\\
 \begin{itemize}
\item \alert{Лидер} - в~системе неформальных отношений
\item \alert{Руководитель} - в~системе формальных отношений
  \end{itemize}
\end{frame}

\lecturenotes

Различия между лидером и руководителем:
1. Руководитель назначается официально, лидер выдвигается неофициально.
2. Руководство выступает как явление более стабильное, чем лидерство. Лидерство является стихийным процессом в отличие от руководства.
3. Руководителю права и полномочия даны законом. Лидер не обладает подобными правами и полномочиями.
4. Руководитель в процессе влияния на подчиненных имеет значительно больше санкций, чем лидер, он может использовать формальные и неформальные санкции. Лидер имеет возможности использовать только неформальные санкции.
5. Руководитель входит в макросреду, его сфера деятельности шире. Лидер является представителем своей группы, ее членом, выступает как элемент микросреды, сфера деятельности лидера ограничивается рамками данной группы.
6. Руководитель регулирует формальные отношения. Деятельность лидера ограничивается рамками межличностных отношений.
7. Для принятия решений руководитель использует большой объем информации, как внешней, так и внутренней. Лидер владеет только той информацией, которая существует в рамках данной группы. Принятие решений руководителем осуществляется опосредованно, а лидером - непосредственно.
8. Руководитель несет внешнюю персональную ответственность за деятельность группы и ее результаты, в том числе перед законом. Лидер не несет подобной ответственности за работу группы и за все, что в ней происходит (если, конечно, группа в своей деятельности не нарушает закон).
9. Руководитель может обладать авторитетом, а может и не иметь его совсем. Лидер всегда авторитетен, в противном случае он не будет лидером.

\section{Специфика команды разработчиков. Роль лидерства.}

\begin{frame} \frametitle{Становление команды и этапы формирования}
 \begin{block}{Этапы формирования команды}
  \begin{itemize}
  \item Объединение
  \item Разногласия и конфликты
  \item Становление
  \item Отдача
  \end{itemize}
  \end{block}
 Успех любого проекта --- это наличие работы слаженного колектива для решения любых поставленных задач.\\
\end{frame}

\lecturenotes

Группа людей, объединенных для выполнения работы, сама по себе не становится самоуправляемой командой. Становление команды это процесс, включающий в себя последовательное прохождение четырех четко выраженных этапов.\\
Объединение:
Люди, объединенные в рабочую группу, имеют различные мотивы и ожидания. Важно понимать, в чем будет выигрыш каждого участника в случае общего успеха проекта, и использовать это знание для сплочения сотрудников. Кроме того, в новый коллектив каждый привносит свою «социальную схему», которая представляет собой личные взгляды на то, как должна функционировать команда. Участники группы должны преодолеть внутренние противоречия, пройти через конфликты прежде, чем сформируется действительно спаянный коллектив. На этом этапе многое зависит от лидера: он должен сформировать общекомандное видение проекта. Все участники группы должны четко понимать не только что именно они будут делать, но и почему они будут это делать. На этом этапе многое зависит от лидера: он должен сформировать общекомандное видение проекта. Все участники группы должны четко понимать не только что именно они будут делать, но и почему они будут это делать.\\
Разногласия и конфликты:
Каждый участник пытается установить и отстоять свою роль в проекте. На этом этапе возможны соперничество, споры, оборонительная позиция. Неизбежные сложности или неудачи порождают конфликты, «поиск виновных».\\
Становление:
На этапе «Становление» в команде укрепляется доверие, люди начинают замечать в коллегах не только проблемные, но и сильные стороны. На смену битве амбиций приходит продуктивное сотрудничество. Четче становится разделение труда, исчезает дублирование функций. Лидер перестает находиться в состоянии постоянного аврала, работа по построению команды на этом этапе — уже не тушение пожара, а скрупулезный труд по отработке общих норм и правил.\\
Отдача:
Когда наступает этот этап, то, наконец-то, можно приступить к получению дивидендов за потраченные усилия. Команда работает эффективно, высок командный дух, люди хорошо знают друг друга и умеют использовать сильные стороны коллег. Высок уровень доверия. Это лучший период для раскрытия индивидуальных талантов. Люди хотят и могут совершенствоваться, они более всего заинтересованы в профессиональном росте. Растет значение нематериальной мотивации сотрудников, а оценивать и поощрять материально лучше команду в целом.
На этом этапе лидер использует стратегию «делегирования». Руководитель поддерживает на необходимом уровне мотивацию участников команды, следит за качеством их работы. Основное внимание руководителя сосредотачивается на делах из второго квадранта — «точим пилу» (см. параграф «Управляем своей жизнью»). Настоящий лидер работает на опережение. Он внимательно следит за изменениями в команде, окружении, целях и задачах проекта — предвидит и избегает риски или снижает их возможные воздействия на проект.

\begin{frame} \frametitle{Основные роли лидера}
  \begin{enumerate}
  \item Генератор идей
  \item Исследователь ресурсов
  \item Координатор
  \item Мотиватор
  \item Аналитик
  \item Вдохновитель команды
  \item Реализатор
  \item Контролер
  \item Специалист
  \end{enumerate}
\end{frame}

\lecturenotes

Для роботы с колективом у лидера есть несколько ролей для сплочения и сплаченной работы колектива:
Генератор идей
Оригинальный мыслитель, который дает жизнь новым идеям. Независимый сотрудник с развитым воображением, но подобно остальным людям имеет негативные черты характера — может быть чрезмерно чувствителен к критике. Для успеха генератору идей необходимы конструктивные отношения с руководителем или координатором группы.
Исследователь ресурсов
Так же, как и генератор идей, в состоянии привнести новые идеи в группу, но эти идеи будут заимствованы извне, благодаря широким контактам. Несколько бесцеремонный, гибкий и ищет благоприятные возможности. Обычно разговаривает по телефону или находится где-нибудь на встрече. Не дает развиваться групповой инертности. К отрицательным качествам характера относятся лень, самодовольство и, иногда, требуется кризис или давление обстоятельств, чтобы мотивировать его.
Координатор
Обычно формальный лидер группы. Руководит и направляет группу в сторону достижения целей. Может заранее определить, кто из работников хорош для выполнения необходимых задач. Обычно спокойный, уверенный и распорядительный. Однако иногда склонен к излишнему доминированию, и группа становится продолжением его сильного «Я».
Мотиватор
Энергичный и в состоянии внедрять идеи. Видит мир как проект, который требует внедрения. Обычно уверенный, динамичный, эмоциональный и импульсивный. Мотор группы, но может быть раздражительным, несдержанным, нелюбезным.
Аналитик
Оценивает предложения и занимает позицию наблюдателя за продвижением. Не дает группе двигаться неправильным путем. Осмотрительный, бесстрастный, имеет аналитический склад ума. Может казаться равнодушным, незаинтересованным, иногда становится чрезмерно критичным.	Все ли возможности мы использовали?
Вдохновитель команды
Стремится объединять и вносить гармонию в отношения между членами группы. Занимает позицию понимающего чужие проблемы, стремится помочь и сглаживает конфликты. По натуре человек добрый, стремится налаживать неформальные отношения. Однако бывает нерешительным в сложных или кризисных ситуациях.
Реализатор
Может преобразовать стратегический план в конкретные управленческие задачи, которые доступны для решения. Хороший организатор, методичный и прагматичный. Идентифицируется с группой, лояльный и честный сотрудник. Однако может быть негибким, непреклонным.
Контролер
Отлично умеет создавать отчеты о работе группы. Озабочен точным выполнением взятых обязательств и старается не упускать из виду даже мелких деталей. Заставляет придерживаться точного расписания дел, но может становиться излишне тревожным.	Это дело требует нашего пристального внимания.
Специалист
Профессионал, самостоятелен стремится стать экспертом в своей области. Обладает высокой профессиональной/технической экспертизой и знаниями, гордится своей работой. Приносит вклад только в узкой сфере своей профессиональной экспертизы.

\begin{frame} \frametitle{Стратегии руководства }
  \begin{itemize}
  \item Штурман — формирует общее видение целей и~систему ценностей.
  \item Помощник — создает и, когда необходимо, меняет структуры, процессы, условия.
  \item Вдохновитель — выявляет и направляет способности каждого на достижение результатов
  \end{itemize}
\end{frame}

\lecturenotes

Лидер — это, прежде всего состоявшаяся личность. Слабый, зависимый, не самостоятельный человек, который не управляет собой, не стал лидером для самого себя, не может быть лидером для команды. Просто самостоятельные и независимые люди, не достаточно зрелы для того, чтобы думать и действовать взаимозависимо. Поэтому они могут хорошо работать индивидуально, но не могут быть эффективными лидерами или командными игроками. Суть лидерства — взаимодействие и синергия. Думать и действовать в духе «выиграл/выиграл» — это настоящее искусство, позволяющее лидерам открывать новые возможности, то, что до них не существовало. Эффективное лидерство это не столько работа с людьми, сколько работа ради людей. Для того чтобы стать лидером, необходимы стратегии управления:
Штурман — формирует общее видение целей и систему ценностей, определяет курс, учитывая постоянные изменения, которые происходят вокруг и находя новые возможности.
Образец для подражания с точки зрения человеческих качеств. Личность, которая заслуживает полное доверие. «Учитель не тот, кто учит, а тот, у которого учатся».
Помощник — создает и, когда необходимо, меняет структуры, процессы, условия, которые обеспечивают эффективность работы каждого. Лидеры следуют правилам до того момента, пока они не увидят, что правила перестают действовать.
Вдохновитель — выявляет и направляет способности каждого на достижение результатов, а не на процессы и методы. Поощряет свободу, ответственность, инициативу и творчество, признает право на ошибку.

\section{Тимбилдинг}

\begin{frame} \frametitle{Что такое тимбилдинг?}
\begin{block}{}
Тимбилдинг --- созданию благоприятных условий для~работы команды, осуществление мероприятий, нацеленных на сплочение коллектива и его организованности.
\end{block}
\end{frame}

\lecturenotes

Термин team building в переводе с английского переводится как «построение команды». Слово у всех на слуху, но несведущие задаются вопросом: тимбилдинг – что это такое? Понятие применимо к широкому диапазону действий: созданию благоприятных условий для~работы команды, повышению ее эффективности, осуществлению мероприятий, нацеленных на сплочение коллектива и его организованности.
В общем смысле тимбилдинг – это идеи и реализации командных методов работы, применимые к любым командам: спортивным, ученическим, религиозным, рабочим и т.д. Сегодня это одна из более перспективных моделей корпоративного менеджмента, но в практику ее начали активно внедрять в 60-70-е годы ХХ века. В России и СНГ командообразование появилось чуть позже.
Зачем нужен тимбилдинг?
Основа работы любой фирмы – командный дух, общие усилия. Вне зависимости от количества членов, входящих в коллектив, сплотить его помогает тимбилдинг, цели и задачи которого в создании чувства единства, сплоченности и организованности.

\begin{frame} \frametitle{Виды тимбилдинга}
  \begin{block}{Мероприятия}
  \begin{itemize}
  \item Спорт и танцы
  \item Благотворительность
  \end{itemize}
  \end{block}
  \begin{block}{IT сфера}
  \begin{itemize}
  \item Социальные сети
  \item Компьютерные игры
  \end{itemize}
  \end{block}
  \begin{block}{Общение}
  \begin{itemize}
  \item Разговор с психологом
  \item Фанты и т.п.
  \end{itemize}
  \end{block}
\end{frame}

\lecturenotes

Примеры тимбилдинга:

Танцевальный марафон

Хотя IT-шникам более привычны удары пальцами по клавиатуре, танцы являются неотъемлемой частью любой культуры и, наверняка, дремлют у всех в подсознании. Хороший мастер-класс позволяет вовлечь в процесс каждого, например, распределяя движения между участниками. Впрочем, классический вариант танцевальных корпоративов, когда мастер учит аудиторию танцевать самбу, румбу или даже вальс, тоже дает очень хорошие результаты. К тому же, как говорят медики, осваивание новых движений способствует комплексному развитию мозга. А это значит, что ваши сотрудники станут еще креативнее.

Спортивные мероприятия

Говоря о спорте, мы советуем пойти дальше, чем проведение шуточных конкурсов, состязаний и эстафет. Пусть в жизни вашего коллектива будет место для настоящих игр. В некоторых компаниях создаются настоящие команды по игре в баскетбол, водное поло или другие виды спорта. Например, сотрудники IT-отдела КРОКа регулярно играют в футбол и фактически являются любительской командой. Серьезные занятия спортом помогают поддерживать здоровье, позволяют узнать многое о привычках и поведении коллег, а также учат доверять друг другу.

Социальные сети

Свойственную современному поколению молодых людей привычку общаться в социальных сетях можно также использовать для рабочих целей. Например, создание корпоративной соцсети позволяет решить сразу несколько задач. Общение коллег в рабочее время может происходить через удобный интерфейс, где бы они ни находились. За ответы на вопросы в форумах людям можно начислять приятные бонусы и еженедельно награждать участников с самым высоким рейтингом. Там же можно размещать отчеты и формировать базу знаний. Социальная сеть также является хорошим ресурсом для проведения конкурсов и исследований. В ней можно запускать интересные тесты, приложения, направленные на знакомства и исследовать психологическую совместимость сотрудников.

Компьютерные игры

Впрочем, если ваши сотрудники не слишком позитивно воспринимают разнообразные корпоративные события, вы можете просто организовать турнир по компьютерным играм. Турниры можно проводить регулярно, но следить за тем, чтобы тимбилдинг не вытеснил из графика дня саму работу.

Привет от психолога

Кстати о доверии: почему бы не применить в рамках тимбилдинга известные психологические ходы? Например, большой эффект дает так называемый нитчатый тимбилдинг. Такой подход хорошо работает для больших команд: люди бросают друг в друга клубками ниток, и в результате оказываются полностью опутанными ими. Каждое движение соседа ощущается вами, и этот контакт оказывает очень интересный эффект с психологической точки зрения.

Чтобы снять стресс у новых сотрудников, можно применить методику «детской колыбели». Для этого нужно взять плотную ткань, например, покрывало или растяжку с корпоративным логотипом, и предложить каждому по очереди ложиться на нее. Тем временем остальные члены коллектива поднимают ткань за края и начинают медленно раскачивать из стороны в сторону. Задача лежащего — расслабиться и почувствовать себя как в колыбели. Конечно, такую практику лучше всего проводить в небольших группах.

Если же вам нужно развить доверие сотрудников друг другу, можно обратиться к классике и использовать известную психологическую игру с падением со стула спиной назад. Остальные подставляют руки и ловят своего коллегу. Уже после одного-двух таких падений подсознательно мы начинаем доверять людям, и этот эффект незамедлительно сказывается на работе.

Психологические тренинги также хорошо работают как «брейнштормы», когда нужно решить какую-то сложную рабочую задачу или выработать стратегию развития, вовлекая в креативный процесс всех сотрудников.

\begin{frame} \frametitle{Танцы}
\alert{Танцевальный тренинг} — относительно новый формат среди корпоративных программ, направленных на~совместный отдых и командообразование.
\end{frame}

\lecturenotes

Танцевальный тренинг — относительно новый формат среди корпоративных программ, направленных на совместный отдых и командообразование. Танцы и музыка — это только положительные эмоции! Движение — Жизнь! Вот такой девиз танцевальных тренингов! Главное – желание раскрепощаться и пробовать новое.

Танцевальный мастер класс — самостоятельное командообразующее мероприятие. Но может быть проведен и в рамках корпоратива. Он объединяет Ваш коллектив в одно целое, где каждый является незаменимой частью, звеном одной танцевальной постановки! Это возможность проявить себя в движении, эмоциях!

Как проходит танцевальный тренинг
Длительность танцевального тренинга может быть от 45 минут до нескольких часов.

Чаще всего в рамках «корпоратива» мы помогаем участникам изучить базовые элементы и движения, позволяющие им танцевать в выбранном стиле. И мы это называем танцевальным мастер-классом.

Программа такого мастер-класса выглядит следующим образом:

Знакомство —  приветствие (иногда — выступление) танцоров-хореографов — 3-5 минут.
«Разогрев» — разминочные упражнения и ice-breaker’ы, помогающие участникам расслабиться и, если необходимо, поделиться на команды — 5-10 минут.
 Основная часть — мастер-класс для участников — 40-45 минут.
Финал — танец или постановка, фото на память — от 5 до 10 минут.
Если же предполагается более длительный и «серьезный» подход, то мы предлагаем несколько репетиций для тщательной проработки танца или подготовки нескольких танцевальных этюдов. Чтобы  уже на корпоративном мероприятии продемонстрировать свои танцевальные постановки друг другу.

\begin{frame} \frametitle{Спорт}
\alert{Спортивный тимбилдинг} – это построение команды, благодаря спортивным играм.\\
\begin{itemize}
	\item Работники находятся в активном движении
	\item Проводится на свежем воздухе
	\item Благоприятно для состояния работников
\end{itemize}
\end{frame}

\lecturenotes

Спортивный тимбилдинг – это построение команды, благодаря спортивным играм. При этом важно учитывать, что участникам не обязательно иметь специальную физическую подготовку. Сегодня это направление стало достаточно популярным, поскольку помогает сотрудникам сблизиться в нерабочее время и объединиться в настоящую команду.

Спортивный тимбилдинг имеет свои особенности:

Самая главная особенность заключается в том, что он проводится в большинстве случаев на свежем воздухе;
Все работники находятся в активном движении большую часть времени, благодаря чему активно вырабатывается гормон счастья или как его принято называть, эндорфин;
Все мероприятия, в которых задействованы сотрудники, благоприятно отражаются на психоэмоциональном состоянии работников.
Также стоит учитывать, поскольку спортивный тимбилдинг проводится на свежем воздухе, единственное, о чем стоит позаботиться – это выбрать подходящий день. При этом время года совершенно не важно, поскольку многообразие игр позволяет отвлечься и настроиться на результат как в зимнее, так и летнее время.

\begin{frame} \frametitle{Социальные сети}
  \begin{block}{Цель}
Вам и Вашим коллегам предстоит сначала зарегистрироваться в новой корпоративной социальной сети, создать свои группы.
  \end{block}
Ваша компания - это одна большая корпоративная \alert{социальная сеть}.
\end{frame}

\lecturenotes

Зарегистрировавшись в социальной сети и вступив в одну из групп, участники начинают жить отнюдь не виртуальной жизнью. В течение дня каждой группе будет необходимо заполнить фотоальбом группы, создать себе и своей группе аватар, нарисовав его на холсте, придумать название и запостить статус!
Все изменения в группах будут онлайн транслироваться на огромном экране. Конечно же, каждой из групп предстоит проверить свою сноровку и ловкость играя в приложения (командные игры и задания).

\begin{frame} \frametitle{Групповая дискуссия}
Обсуждение, критика, новые взгляды и свежие точки зрения - тот золотой фонд, которым владеет команда и~которого не имеют одиночки.
\end{frame}

\lecturenotes

Действительно, командная деятельность невозможна без активного общения и взаимодействия. Решение общих задач требует от участников единства, сплоченности, согласования действий, координации решений. Командные отношения - это всегда своеобразное сотворчество, когда мысли одного обогащаются за счет их переосмысления другими. Обсуждение, критика, новые взгляды и свежие точки зрения - тот золотой фонд, которым владеет команда и которого не имеют одиночки.

Отношения взаимозависимости и взаимодействия является отличительным признаком команды. Так называемый эффект синергии - это результат эффективного взаимодействия между членами команды на основе общих стремлений и ценностей. Он приводит к тому, что суммарное усилие команды намного превышает сумму усилий ее отдельных членов.

\begin{frame} \frametitle{}
"Люди вместе могут совершить то, чего не в силах сделать в~одиночку; единение умов и рук, сосредоточение их сил может стать почти всемогущим."
\end{frame}

\begin{thebibliography}{99}

%Информацию собирал с нескольких мест на один слайд, поэтому указал списком, а не метками

С. Архипенков Руководство командой разработчиков разработчиков ПО;
Государственный университет  - Высшая школа экономики Факультет Бизнес-информатики Учебное пособие «Лидерство и управление командой»;
\end{thebibliography}

\end{document}

%%% Local Variables: 
%%% mode: TeX-pdf
%%% TeX-master: t
%%% End: 
