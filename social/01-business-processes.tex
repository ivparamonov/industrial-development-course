\documentclass{../industrial-development}
%\documentclass[lecturenotes]{../industrial-development}
\graphicspath{{01-business-processes/}}

\title{Бизнес-процессы компании по разработке программного обеспечения}
\author{Новиков Денис Александрович, \\Мехтиханов Леонид Игоревич, \\ПИ-21 МО}
\date{}

\begin{document}

\begin{frame}
  \titlepage
\end{frame}


\section{Бизнес-процессы ИТ-компании}

\subsection{Понятие бизнес-процесса}


\begin{frame} \frametitle{Понятие бизнес-процесса}
	\begin{block}{}
		\alert{Бизнес-процесс} "--- логически завершённая цепочка взаимосвязанных и повторяющихся видов деятельности, в результате которых создаётся продукт или сопутствующая ему услуга
	\end{block}
	\begin{block}{Свойства}
		\begin{itemize}
			\item Может являться частью более общего бизнес-процесса
			\item Потребителем результатов его деятельности может выступать другой бизнес-процесс
		\end{itemize}
	\end{block}
\end{frame}

\lecturenotes

Подход к управлению организацией как управлению функционально разделёнными отделами в современном мире не является эффективным.  Современный подход управления - это управление бизнес-процессами. Например, стандарт ISO 9001:2000 требует использовать процессный подход к управлению организацией.

Бизнес-процесс "--- устойчивая, целенаправленная совокупность взаимосвязанных видов деятельности (последовательность работ), которая по определённой технологии преобразует входы в выходы по определённым правилам с помощью определённых механизмов~\cite{BStudio}.


\begin{frame} \frametitle{Преимущества процессного подхода}
	В отличие от функционального моделирования, процессный подход нацелен на:
	\begin{itemize}
		\item Эффективное разграничение полномочий и повышение автономности персонала
		\item Снижение зависимости от конкретного исполнителя
		\item Стандартизацию рутинных процессов
		\item Выявление источников сокращения издержек и времени на исполнение бизнес-процессов
	\end{itemize}
\end{frame}

\lecturenotes

Основными недостатками функционального подхода к управлению организацией, вытекающими из отсутствия нацеленности на конечный результат, признаны высокие накладные расходы, длительные сроки выработки управленческих решений, риск потери клиентов.

Процессный подход к управлению игнорирует организационную структуру управления организацией со свойственным ей закреплением функций за отдельными подразделениями. При процессном подходе организация воспринимается руководителями и сотрудниками как деятельность, состоящая из бизнес-процессов, нацеленных на получение конечного результата. Организация воспринимается как сеть бизнес-процессов, представляющая собой совокупность взаимосвязанных и взаимодействующих бизнес-процессов, включающих все функции, выполняемые в подразделениях организации. В то время как функциональная структура бизнеса определяет возможности предприятия, устанавливая, что следует делать, процессная структура (в операционной системе бизнеса) описывает конкретную технологию выполнения поставленных целей и задач, отвечая на вопрос, как это следует делать.

К основным преимуществам процессного подхода можно отнести:
\begin{itemize}
	\item нацеленность на удовлетворение требований клиента;
	\item освобождение руководства от рутины оперативного управления;
	\item возможность выявления узких мест и резервов работы;
	\item создание эталонов последовательности действий персонала;
	\item появление возможности «тиражирования» бизнеса – открытия новых бизнес-площадок на основе формализованных бизнес-процессов;
	\item реализация принципа постоянного совершенствования деятельности.
\end{itemize}
Эти преимущества гарантируют высокую результативность деятельности организации, управление которой имеет выраженный процессно-ориентированный характер~\cite{Ekon123}.


\subsection{Классификация бизнес-процессов}


\begin{frame} \frametitle{Классификация бизнес-процессов}
	\begin{itemize}
		\item Основные процессы
		\item Вспомогательные процессы
		\item Управляющие процессы
	\end{itemize}
\end{frame}


\begin{frame} \frametitle{Основные процессы}
	\begin{block}{}
		Направлены на создание, эксплуатацию и поддержку ПО. Генерируют ценность для клиента и прибыль для компании
	\end{block}
	\begin{block}{Примеры}
		\begin{itemize}
			\item Анализ потребностей клиента
			\item Формирование требований к ПО
			\item Разработка ПО
			\item Поддержка и сопровождение продукта
		\end{itemize}
	\end{block}
	\begin{block}{Результат}
		ПО или услуга удовлетворяющее потребностям заказчика
	\end{block}
\end{frame}

\lecturenotes

Отличительной чертой данной группы является тот факт, что процессы, входящие в неё, принимают непосредственное участие в реализации установленных направлений компании. Именно они определяют доход предприятия и имеют стратегическое значение. Данная разновидность поступательных действий никогда не может быть отдана на аутсорсинг, это приведёт к потере конкурентоспособности. И именно эти процессы каждая компания должна выполнять наилучшим образом в своей сфере. Все они так же могут быть подразделены на несколько групп:
\begin{itemize}
	\item те, которые создают добавленную стоимость;
	\item те, которые создают сам продукт;
	\item генерирующие доходы;
	\item те, которые клиент готов оплачивать.
\end{itemize}
Ещё раз следует подчеркнуть: данная группа обеспечивает доходность предприятия и определяет профиль и направление его деятельности~\cite{Vseproip}.


\begin{frame} \frametitle{Вспомогательные процессы}
	\begin{block}{}
		Обеспечивают ресурсы для основных процессов. Не создают продукт напрямую, но необходимы для его создания
	\end{block}
	\begin{block}{Примеры}
		\begin{itemize}
			\item Поиск и подготовка кадров
			\item Организация технической и иной инфраструктуры
			\item Организация труда, оснащение рабочих мест
		\end{itemize}
	\end{block}
	\begin{block}{Результат}
		Ресурсы, необходимые для функционирования основных процессов (разработка и сопровождение ПО)
	\end{block}
\end{frame}

\lecturenotes

Обеспечивающие бизнес-процессы разительно отличаются от основных, и имеют совершенно другие цели и назначение. Если первая группа отвечает за производство конечного продукта и получение прибыли, то данная категория позволяет поддерживать инфраструктуру всего предприятия.

За эти процессы клиент, безусловно, уже не готов платить деньги, но они являются обязательными в любой компании. В качестве таких механизмов можно привести примеры административно-хозяйственного обеспечения, безопасность, юридическое содействие. Рассчитаны они больше на внутреннего клиента, то есть на сотрудников. Без присутствия таких операций внутренний штат предприятия будет не способен реализовывать процессы предыдущей группы.

Поскольку действия, производимые в рамках данной группы, не имеют стратегического значения и не отвечают за прирост прибыли или направление деятельности компании, у них есть ещё второе название "--- вспомогательные бизнес-процессы. В отличие от основных механизмов работы компании, данные функции предприятия уже могут быть отданы на аутсорсинг, для выполнения их лицами, которые имеют в этом профессиональный опыт~\cite{Vseproip}.


\begin{frame} \frametitle{Управляющие процессы}
	\begin{block}{}
		Обеспечивают координацию и слаженную работу бизнес-процессов и функциональных единиц компании
	\end{block}
	\begin{block}{Примеры}
		\begin{itemize}
		\item Планирование
		\item Коммуникация (мотивация и поддержка сотрудников)
		\item Реализация и контроль выполнения плана действий
		\item Оценка результатов
		\end{itemize}
	\end{block}
	\begin{block}{Результат}
		Организационная структура и управленческие решения, обеспечивающие функционирование компании
	\end{block}
\end{frame}

\lecturenotes

К данной категории можно отнести те действия, которые связаны с руководством функционированием всей системы предприятия. Яркими примерами в рамках этой группы может быть стратегический менеджмент и корпоративное управление.

К ним можно отнести:
\begin{itemize}
	\item стратегическое управление;
	\item контроль и распределение финансов;
	\item маркетинг;
	\item контроль качества;
	\item управление персоналом;
	\item руководство проектами развития.
\end{itemize}

Управляющие бизнес-процессы, как правило, охватывают весь спектр функций управления предприятием на уровне каждой цепочки действий в отдельности и бизнес-системы в целом. Данные механизмы регулирования можно назвать стратегическим, оперативным или текущим планированием. Они направлены на формирование и осуществление управленческих воздействий на все процессы, протекающие на предприятии~\cite{Vseproip}.


\subsection{Бизнес-процессы создания ПО}


\begin{frame} \frametitle{Бизнес-процессы создания ПО: Подготовка}
	\begin{itemize}
		\item Бизнес-процесс подготовки необязателен и в ряде случаев является частью бизнес-процесса анализа
		\item Применяется когда затраты на более строгий анализ слишком высоки, позволяет отказаться от нереализуемых и потенциально невостребованных продуктов на ранних этапах
	\end{itemize}
\end{frame}

\lecturenotes

Подготовительный этап процесса разработки ПО имеет для исполнителя очень простую цель "--- принять решение, стоит ли браться за реализацию программного продукта. С точки зрения заказчика системы целью этого этапа является принятие решения о том, можно ли доверять исполнителю. При успешном развитии событий естественным итогом подготовительного этапа должно стать заключение договора (или иного соглашения, по ситуации) на создание или модификацию системы, требуемой заказчику.

Для достижения поставленных целей заказчику и исполнителю совместно нужно решить ряд вполне определённых задач:
\begin{itemize}
	\item на основе исходной идеи сформулировать цели и задачи будущего проекта;
	\item разработать некоторое исходное видение "--- концепцию проекта;
	\item провести анализ востребованности будущего продукта;
	\item провести предварительную оценку рисков будущего проекта;
	\item на основе концепции и списка предварительных рисков подготовить предварительное техническое решение;
	\item выбрать методологию разработки и подготовить предварительный план работ;
	\item провести предварительную оценку трудозатрат и необходимых ресурсов;
	\item провести анализ реализуемости продукта;
	\item провести независимое рецензирование технического решения;
	\item принять решение о том, стоит ли продолжать работы~\cite{Habr1}.
\end{itemize}


\begin{frame} \frametitle{Бизнес-процессы создания ПО: Подготовка}
	\begin{block}{Предназначение}
		\begin{itemize}
			\item Анализ целей на основе исходной идеи
			\item Анализ востребованности будущего продукта
			\item Анализ рисков
			\item Анализ предварительного технического решения
			\item Анализ трудозатрат и необходимых ресурсов
			\item Анализ реализуемости
		\end{itemize}
	\end{block}
	\begin{block}{Результат}
		Решение, о том, стоит ли продолжать работы
	\end{block}
\end{frame}


\begin{frame} \frametitle{Бизнес-процессы создания ПО: Анализ и проектирование}
	\begin{block}{Предназначение}
		\begin{itemize}
			\item Определение потребностей клиента и путей их удовлетворения
			\item Оценка существующих решений, их соответствия поставленной задаче
			\item Разработка архитектуры решения, определение используемых технологий
			\item Формирование технического задания, спецификации
		\end{itemize}
	\end{block}
	\begin{block}{Результат}
		Формальные требования к разрабатываемому ПО, технический проект (техническое задание)
	\end{block}
\end{frame}

\lecturenotes

При формулировании цели и задач проекта должен быть создан документ, дающий ответы на следующие вопросы:
\begin{itemize}
	\item Как определяется предметная область для данного проекта?
	\item Какие термины используются в предметной области? Какие бизнес-процессы, затрагиваемые проектом, протекают в предметной области?
	\item Какая цель у проекта?
	\item Какой эффект должна оказать разрабатываемая система на бизнес-процессы предметной области?
	\item Какие задачи требуется решить, чтобы достичь поставленной цели?
	\item По каким критериям будет оцениваться качество решения поставленных задач?
	\item Каковы функциональные и нефункциональные показатели качества разрабатываемой системы?
	\item Какие ограничения на сроки выполнения, ресурсы, бюджет накладываются на реализацию проекта?
\end{itemize}

Ответы на эти вопросы должны позволить исполнителям сформулировать концепцию проекта и дать оценку его востребованности и реализуемости. Для заказчика корректность ответов на указанные вопросы важна, потому что эта информация задаёт критерии, по которым заказчик сможет решить, стоит ли доверять работу тому или иному исполнителю.

Как правило, именно заказчик, как носитель идеи проекта, формулирует цель и задачи. Но в моей практике был случай, когда эту работу пришлось выполнять нам, исполнителям. Нам предстоял тендер на разработку одной весьма специфической системы для некой правительственной службы. Такие проекты имеют особенность: службе «сверху» даётся поручение разработать некоторую систему и сопровождать её, но сама служба, как правило, не является пользователем системы. Естественно, руководство службы берёт под козырёк и объявляет конкурс. Но никто в самой службе не знает, для какой цели и как будет использоваться разрабатываемый продукт. И, что хуже всего, никто не стремится это знать. В таких условиях исходное видение системы у заказчика (правительственной службы) отсутствует почти полностью. Ни цели, ни задачи проекта заказчик сформулировать не может. Нам пришлось самим облазить всё оборудование заказчика, разыскать нужную техническую документацию и выйти на конечных пользователей в различных правительственных ведомствах. В конечном итоге, мы успешно выполнил проект, получили прибыль, неплохой опыт и, что немаловажно, рекомендации заказчика. Однако такое положение, всё же, "--- исключение. Цели и задачи проекта должен формулировать именно носитель идеи~\cite{Habr1}.


\begin{frame} \frametitle{Бизнес-процессы создания ПО: Планирование}
	\begin{block}{Предназначение}
		\begin{itemize}
			\item Формирование оптимального плана работ, выделение основных этапов, условий их завершений
			\item Оценка временных и стоимостных затрат при различных подходах, с учётом доступных ресурсов
		\end{itemize}
	\end{block}
	\begin{block}{Результат}
		Целевой план работ, план бюджетирования
	\end{block}
\end{frame}

\lecturenotes

Организация бизнес-планирования требует наличия в организации следующих компонентов:
\begin{itemize}
	\item аналитический блок (методология и методика бизнес-планирования);
	\item программно-технологический блок (технические средства, программное обеспечение);
	\item информационный блок (информация о внутренней и внешней средах);
	\item организационный блок (структура и система управления, система ответственности, функции структурных подразделений, регламент взаимодействия).
\end{itemize}

Бизнес-планирования имеет следующую логику:
\begin{itemize}
	\item формируется группа разработчиков;
	\item определяются центры прибыли/отдельные подразделения;
	\item для каждого направления, центра прибыли, подразделения составляются отдельные бизнес-планы;
	\item единичные планы сводятся воедино, составляется итоговый бизнес-план всей организации~\cite{Centeryf}.
\end{itemize}


\begin{frame} \frametitle{Бизнес-процессы создания ПО: Технико-экономического обоснование}
	\begin{block}{Предназначение}
		\begin{itemize}
			\item Определение эффекта от внедрения, его оценка с учётом предполагаемого срока эксплуатации
			\item Оценка рентабельности инвестиций, например, путём расчёта соотношения эффект / затраты
			\item Планирование бюджетирования
		\end{itemize}
	\end{block}
	\begin{block}{Результат}
		Оценка экономического эффекта, определение необходимости дальнейшей работы над проектом
	\end{block}
\end{frame}

\lecturenotes

Технико-экономическое обоснование (детальное ТЭО) должно дать всю необходимую информацию для принятия решения об инвестировании в проект. 

Оценка затрат, результатов и показателей эффективности осуществления проекта при проведении финансового анализа проводится с использованием моделей потоков реальных денег, потоков доходов и проектируемого баланса

Оценка основных факторов неопределённости и рисков по проекту, его устойчивости и безубыточности проводится на заключительном этапе финансового анализа.

ТЭО должно включать определение всех факторов коммерческого, технического, предпринимательского и других рисков.

Если слабые места обнаружились на предыдущих стадиях и (или) прибыльность проекта вызывает сомнения, необходимо более тщательно проверить наиболее чувствительные параметры проекта (размер рынка, производственная программа, объём продаж, цены на материалы и сырьё, цены на продукцию по проекту, выбор оборудования и др.).

По результатам анализа должны быть оценены издержки, доход и чистая прибыль, другие показатели эффективности и финансовые показатели проекта.

Все принятые допущения, используемые данные и выбранные варианты должны быть описаны и объяснены, а сделанные прогнозные оценки подтверждены соответствующими исследованиями, чтобы сделать проект более понятным инициатору проекта и инвестору при оценке результатов выполненных исследований.


\begin{frame} \frametitle{Бизнес-процессы создания ПО: Разработка}
	\begin{block}{Предназначение}
		\begin{itemize}
			\item Реализация архитектуры проекта
			\item Реализация функционала ПО в соответствии со спецификацией, техническим заданием
			\item Разработка документации
		\end{itemize}
	\end{block}
	\begin{block}{Результат}
		Программное обеспечение
	\end{block}
\end{frame}

\lecturenotes

Данный этап разработки программного обеспечения организован в соответствии с моделями эволюционного типа жизненного цикла ПО. При разработке применяются экспериментирование и анализ, строятся прототипы, как целой системы, так и её частей. Прототипы дают возможность глубже вникнуть в проблему и принять все необходимые проектные решения ещё на ранних этапах проектирования. Такие решения могут затрагивать разные части системы: внутреннюю организацию, пользовательский интерфейс, разграничение доступа и т.д. В результата этапа реализации появляется рабочая версия продукта.

После того как требования и дизайн продукта утверждены, происходит переход к следующей стадии жизненного цикла – непосредственно разработке. Здесь начинается написание программистами кода программы в соответствии с ранее определёнными требованиями.

Системные администраторы настраивают программное окружение, front-end программисты разрабатывают пользовательский интерфейс программы и логику её взаимодействия с сервером.

Кроме того, программисты пишут Unit-тесты для проверки правильности работы кода каждого компонента системы, проводят ревью написанного кода, создают билды и разворачивают готовое ПО в программной среде. Этот цикл повторяется до тех пор, пока все требования не будут реализованы~\cite{Abolut,Qalight}.


\begin{frame} \frametitle{Бизнес-процессы создания ПО: Контроль качества}
	\begin{block}{}
		\alert{Контроль качества} (Quality Control) "--- процесс определения соответствия ПО требованиям заказчика
	\end{block}
	\begin{block}{Предназначение}
		\begin{itemize}
			\item Проверка соответствия спецификации
			\item Проверка соответствия тестовых показателей нормативам, анализ их динамики
			\item Контроль актуальности документации
		\end{itemize}
	\end{block}
	\begin{block}{Результат}
		Информация о фактическом состоянии продукта, мере его соответствия целевым показателям
	\end{block}
\end{frame}

\lecturenotes

В первую очередь надо выяснить по каким метрикам надо определять качество кода и для чего это нам вообще нужно. В программировании нам повезло и, в большинстве случаев, для определении метрики нам достаточно определить важную для нас характеристику:
\begin{itemize}
	\item соответствие правилам;
	\item сложность кода;
	\item дубликаты;
	\item комментирование;
	\item покрытие тестами.
\end{itemize}

Сейчас рассмотрим каждую из них.

Соответствие правилам

Под этот пункт подпадают ситуации когда код компилируется и, в большинстве случаев, делает своё дело, при чем делает это правильно. Это интересная характеристика в большей степени от того, что в компании сначала должны существовать правила написания кода. Можно поступить проще и взять труд других (Java Code Conventions, GCC Coding Conventions, Zends Coding Standard), а можно поработать и дополнить их своими, наиболее подходящими для специфики вашей компании.

Но зачем нам правила написания кода, если код делает своё дело? Чтобы ответить на вопрос выделим несколько типов правил:
\begin{itemize}
	\item синтаксические правила "--- одни из наиболее бесполезных правил (но только первый взгляд), поскольку совсем никоим образом не виляют на исполнение программы. К ним можно отнести стиль именования переменных (camelCase, через подчёркивание), констант (uppercase), методов, стиль написания фигурных скобок и нужны ли они если в блоке только одна строка кода. Этот список можно продолжить. Когда программист пишет код, он его легко читает, потому что он знает свой собственный стиль. Но стоит ему дать код где используется венгерская нотация и скобки с новой строки, ему придётся тратить дополнительное внимание на восприятие нового стиля. Особенно веселит ситуация когда несколько совсем разных стилей используются в одном проекте или даже модуле.
	\item правила поддержки кода "--- правила, которые должны сигнализировать что код слишком сложный и его будет трудно сопровождать. К примеру, индекс сложности (подробнее о нём ниже) метода или класса слишком большой или слишком много строк кода в методе, наличие дубликатов в коде или <<magic numders>>. Думаю суть ясна, все они указывают нам на узкие места которые сложно будет сопровождать. Но нельзя забывать что именно мы можем решить какой индекс сложности для нас большой, а какой приемлемый;
	\item очистка и оптимизация кода "--- самые простые правила в том смысле, что редко кто-то будет утверждать что выражения очень нужны, даже когда они нигде не используются. Сюда можно отнести лишние импорты, переменные и методы которые уже не используются, но по какой-то причине их оставили в наследство.
\end{itemize}

Метрика здесь очевидная: соответствие правилам должно стремится к 100\%, то есть чем меньше нарушений правил тем лучше.

Цикломатическая сложность кода

Характеристика, от которой напрямую зависит сложность поддержки кода. Здесь выделить метрику посложнее чем в предыдущей характеристике. Если по простому, оно зависит от количества вложенных операторов ветвления и циклов. Кому интересно более подробное описания, можно почитать на вики. Чем индекс ниже тем лучше, и тем легче в будущем будет менять структуру кода. Стоит мерить сложность метода, класса, файла. Значение этой метрики надо ограничить некоим предельным числом. К примеру цикломатическая сложность метода не должна превышать 10, иначе нужно упростить или разбить его.

Дубликаты

Важная характеристика, которая отображает насколько легко в будущем (или настоящим) можно будет вносить изменения в код. Метрику можно означить в процентах как соотношение строк дубликатов к всем строкам кода. Чем меньше дубликатов тем легче будет жить с этим кодом.

В фазе тестирования обнаруживаются пропущенные при разработке баги. При обнаружении дефекта, тестировщик составляет отчёт об ошибке, который передаётся разработчикам. Последние его исправляют, после чего тестирование повторяется "--- но на этот раз для того, чтобы убедиться, что проблема была исправлена, и само исправление не стало причиной появления новых дефектов в продукте.

Тестирование повторяется до тех пор, пока не будут достигнуты критерии его окончания~\cite{Habr2}.


\begin{frame} \frametitle{Бизнес-процессы создания ПО: Приёмочные испытания и внедрение}
	\begin{block}{Предназначение}
		\begin{itemize}
			\item Финальная проверка продукта на соответствие спецификации
			\item Проведение приёмочных испытаний ПО
			\item Передача ПО и документации заказчику
		\end{itemize}
	\end{block}
	\begin{block}{Результат}
		Запуск системы в реальных условиях
	\end{block}
\end{frame}

\lecturenotes
	
Внедрение программного обеспечения "--- процесс настройки программного обеспечения под определённые условия использования, а также обучения пользователей работе с программным продуктом.

Внедрение программного продукта состоялось в том случае, если программный продукт выполняет поставленную задачу, а сотрудники компании полностью перешли на работу с новым продуктом~\cite{Habr3}.	


\begin{frame} \frametitle{Бизнес-процессы создания ПО: Сопровождение и эксплуатация}
	\begin{block}{Предназначение}
		\begin{itemize}
			\item Извлечение ценности из продукта заказчиком
			\item Актуализация программного решения в соответствии с изменениями бизнес-задач
			\item Завершение в случае, когда ПО перестаёт удовлетворять требованиям заказчика
		\end{itemize}
	\end{block}
	\begin{block}{Результат}
		Эксплуатация в реальных условиях, выполнение бизнес-функций заказчика
	\end{block}
\end{frame}

\lecturenotes

Сопровождение (поддержка) программного обеспечения "--- процесс улучшения, оптимизации и устранения дефектов программного обеспечения (ПО) после передачи в эксплуатацию. Сопровождение ПО — это одна из фаз жизненного цикла программного обеспечения, следующая за фазой передачи ПО в эксплуатацию. В ходе сопровождения в программу вносятся изменения, с тем, чтобы исправить обнаруженные в процессе использования дефекты и недоработки, а также для добавления новой функциональности, с целью повысить удобство использования (юзабилити) и применимость ПО.

Сопровождение программного обеспечения стандартизовано, имеются национальные стандарты Российской Федерации, идентичные международным (ISO/IEC 12207:2008 System and software engineering "--- Software life cycle processes, ГОСТ Р ИСО/МЭК 12207-2010 <<Национальный стандарт Российской Федерации. Информационная технология. Системная и программная инженерия. Процессы жизненного цикла программных средств>>; ISO/IEC 14764:99 Information tehnology "--- Software maintenance, ГОСТ Р ИСО/МЭК 14764-2002 <<Государственный стандарт Российской Федерации. Информационная технология. Сопровождение программных средств>>; IEEE 1219)~\cite{Wiki1}.


\begin{frame} \frametitle{Бизнес-процессы создания ПО: Завершение эксплуатации}
	\begin{itemize}
		\item Происходит когда продукт не удовлетворяет требованиям заказчика и его доработка невозможна или нерентабельна
		\item Это может произойти по ряду причин: изменение потребностей бизнеса, конъюнктуры рынка, технологические изменения
	\end{itemize}
\end{frame}


\begin{frame} \frametitle{Бизнес-процессы создания ПО: Завершение эксплуатации}
	\begin{block}{Предназначение}
		\begin{itemize}
			\item Анализ опыта с целью дальнейшей оптимизации бизнес-процессов создания ПО
			\item Завершение работы над проектом и остановка сопровождения
		\end{itemize}
	\end{block}
	\begin{block}{Результат}
		Полное прекращение деятельности в рамках проекта
	\end{block}
\end{frame}


\section{Схемы организации бизнеса}

\subsection{Разработка собственного продукта}


\begin{frame} \frametitle{Разработка собственного продукта}
	\begin{block}{}
		\alert{Разработка собственного продукта} "--- подход, при котором компания самостоятельно определяет требования к продукту, порядок и способ их реализации
	\end{block}
	\begin{block}{Сценарии}
		\begin{itemize}
			\item Разработка ПО для внутренних нужд
			\item Разработка ПО для дальнейшей массовой реализации в виде типового решения
		\end{itemize}
	\end{block}
\end{frame}


\begin{frame} \frametitle{Разработка собственного продукта}
	\begin{block}{Особенности}
		\begin{itemize}
			\item Наличие компетенции в предметной области
			\item Динамичность и высокий срок эксплуатации продукта
			\item Преобладание итеративного подхода к разработке
			\item Внутреннее бюджетирование
		\end{itemize}
	\end{block}
\end{frame}

\lecturenotes

Самостоятельная разработка автоматизированных систем управления имеет как свои плюсы, так и минусы. Преимуществами в данном случае являются:
\begin{itemize}
\item дешевизна;
\item близость разработчиков к бизнес-процессам компании. Но всё же, часто этот фактор из положительного становится отрицательным, поскольку система настолько срастается с текущей организационной структурой компании, что утрачивает гибкость;
\item нулевая стоимость дополнительных лицензий.
\end{itemize}

Также существуют значительные ограничения и риски при разработке новой системы:
\begin{itemize}
\item необходимость значительного увеличения численности ИТ-службы компании, так как на период разработки новой КИС существующую ИС кто-то должен поддерживать в работоспособном состоянии;
\item отсутствие проработанного этапа бизнес-анализа. Для появления более-менее квалифицированных системных аналитиков необходимы время и значительные вложения. Без этого этапа ИС создаётся скорее не для выполнения требований пользователей, а становится объектом для реализации различных идей разработчиков;
\item стоимость сопровождения собственной системы гораздо больше, чем расходы на внедрение и сопровождение типовой~\cite{Lektsii}.
\end{itemize}


\begin{frame} \frametitle{Разработка собственного продукта: ПО для внутренних нужд}
	\begin{block}{Бизнес-процессы}
		Полный цикл бизнес-процессов разработки: основных, вспомогательных и управляющих
	\end{block}
	\begin{block}{Прибыль}
		Продукт генерирует прибыль косвенно за счёт оптимизации бизнес-процессов компании или их реализации
	\end{block}
\end{frame}


\begin{frame} \frametitle{Разработка собственного продукта: ПО для внутренних нужд}
	\begin{block}{Условия}
		\begin{itemize}
			\item На рынке отсутствует необходимое ПО
			\item Эксплуатация и/или поддержка существующих решений слишком дорога
			\item Решение носит стратегический характер и риски связанные с прекращением поддержки внешним разработчиком неприемлемы
		\end{itemize}
	\end{block}
\end{frame}


\begin{frame} \frametitle{Разработка собственного продукта: ПО для внутренних нужд}
	\begin{block}{Примеры}
		\begin{itemize}
			\item ПО для объектов критической инфраструктуры: АЭС, космос, военные разработки
			\item Узкоспециализированное ПО для уникального оборудования
			\item ПО для инновационных сфер: робототехника
		\end{itemize}
	\end{block}
\end{frame}


\begin{frame} \frametitle{Разработка собственного продукта: ПО в качестве типового решения}
	\begin{block}{Бизнес-процессы}
		Полноценный цикл бизнес-процессов разработки: основных, вспомогательных и управляющих.\\
		Возрастает нагрузка на менеджмент, появляются дополнительные управляющие бизнес-процессы:
		\begin{itemize}
			\item Стратегическое управление и планирование
			\item Управление финансами
			\item Управление маркетингом
		\end{itemize}
	\end{block}
	\begin{block}{Прибыль}
		Продукт генерирует прибыль напрямую за счёт продажи лицензий и сопутствующих услуг
	\end{block}
\end{frame}


\begin{frame} \frametitle{Разработка собственного продукта: ПО в качестве типового решения}
	\begin{block}{Условия}
		\begin{itemize}
			\item На рынке отсутствует необходимое ПО, либо его качество недостаточно
			\item Наличие видения пути развития ПО и его позиционирования на рынке
			\item Наличие фатальных недостатков в существующих решениях
			\item Наличие ресурсов для продвижения и поддержки ПО при массовых продажах
			\item Эксплуатация и/или поддержка существующих решений слишком дорога
		\end{itemize}
	\end{block}
\end{frame}


\begin{frame} \frametitle{Разработка собственного продукта: ПО в качестве типового решения}
	\begin{block}{Примеры}
		\begin{itemize}
			\item CMS для веб-сайтов: 1C-Битрикс, ReadyScript, NetCat
			\item Платформы для бизнеса: 1C-Предприятие, Галактика, Microsoft Dynamics Axapta
			\item Игровые движки: CryEngine, Unity, Unreal Engine, Dunia
		\end{itemize}
	\end{block}
\end{frame}


\subsection{Разработка продукта на заказ}


\begin{frame} \frametitle{Разработка продукта на заказ}
	\begin{block}{}
		\alert{Разработка продукта на заказ} "--- подход к организации разработки, при котором источником генерации прибыли является ПО, соответствующее требованиям внешнего заказчика
	\end{block}
	\begin{block}{Сценарии}
		\begin{itemize}
			\item Разработка нового ПО
			\item Доработка существующих решений
			\item Аутсорсинг
		\end{itemize}
	\end{block}
\end{frame}


\begin{frame} \frametitle{Разработка продукта на заказ}
	\begin{block}{Особенности}
		\begin{itemize}
			\item Требования формирует внешний заказчик
			\item Контроль осуществляется как исполнителем, так и заказчиком
			\item Отсутствие необходимости в продвижении продукта
			\item Внешнее бюджетирование
		\end{itemize}
	\end{block}
\end{frame}
	

\begin{frame} \frametitle{Разработка продукта на заказ: Создание новых решений}
	\begin{block}{Бизнес-процессы}
		В отличие от разработки собственного продукта, отпадает необходимость анализа его бизнес-концепции, стратегии продвижения на рынке и т.д, что отражается в исключении соответствующих бизнес-процессов
	\end{block}
	\begin{block}{Прибыль}
		Продукт генерирует прибыль за счёт решения бизнес-проблем заказчика
	\end{block}
\end{frame}


\begin{frame} \frametitle{Разработка продукта на заказ: Создание новых решений}
	\begin{block}{Условия}
		\begin{itemize}
			\item На рынке отсутствует необходимое ПО, либо его качества не удовлетворяют заказчика
			\item Нецелесообразность или невозможность доработки имеющихся решений
			\item Эксплуатация и/или поддержка существующих решений слишком дорога
		\end{itemize}
	\end{block}
\end{frame}


\begin{frame} \frametitle{Разработка продукта на заказ: Создание новых решений}
	\begin{block}{Примеры}
		\begin{itemize}
			\item Разработка веб-сайтов и мобильных приложений
			\item Разработка ПО под инфраструктуру заказчика
			\item Разработка уникального ПО с учётом особенностей предметной области: ПО для станков
			\item Разработка специфичных системы бизнес-аналитики
		\end{itemize}
	\end{block}
\end{frame}


\begin{frame} \frametitle{Разработка продукта на заказ: Доработка существующих решений}
	\begin{block}{Бизнес-процессы}
		В отличии от создания новых решений, появляется необходимость анализа совместности требований и возможности существующего решения
	\end{block}
	\begin{block}{Прибыль}
		Продукт генерирует прибыль напрямую за счёт продажи лицензий и сопутствующих услуг
	\end{block}
\end{frame}


\begin{frame} \frametitle{Разработка продукта на заказ: Доработка существующих решений}
	\begin{block}{Условия}
		\begin{itemize}
			\item Наличие опыта работы с подобными решениями
			\item Наличие технической возможности
			\item Экономическая целесообразность доработки
		\end{itemize}
	\end{block}
\end{frame}


\begin{frame} \frametitle{Разработка продукта на заказ: Доработка существующих решений}
	\begin{block}{Примеры}
		\begin{itemize}
			\item Создание и доработка 1С конфигураций
			\item Создание и доработка имеющихся веб-сайтов и приложений
			\item Доработка существующей CRM системы под требования заказчика
		\end{itemize}
	\end{block}
\end{frame}


\begin{frame} \frametitle{Разработка продукта на заказ: Аутсорсинг}
	\begin{block}{Бизнес-процессы}
		Формируют автономный бизнес-процесс "--- <<чёрный ящик>>, направленный на удовлетворение потребностей заказчика, и содержание которого отходит на второй план.
		Это позволяет снизить среднесписочное число сотрудников, поскольку исполнители являются работниками внешней компании
	\end{block}
	\begin{block}{Прибыль}
		Источником прибыли является совокупность процессов и услуг, формирующих автономный бизнес-процесс
	\end{block}
\end{frame}

\lecturenotes

Аутстаффинг представляет собой комплекс мер, направленных на использование персонала вне штата. Явление это достаточно новое для России. Следует отличать аутстаффинг от аутсорсинга. Для аутстаффинга характерны краткосрочные отношения и найм рабочих, которые будут выполнять отдельные операции, в то время как аутсорсинг характеризуется долгосрочностью и высокой квалификацией специалистов, которым передаётся весь бизнес-процесс (бухгалтерия, call-центр и т.д.).


\begin{frame} \frametitle{Разработка продукта на заказ: Аутсорсинг}
	\begin{block}{Условия}
		\begin{itemize}
			\item Носит системный, продолжительный характер
			\item Ориентирован на получение результата, отвлечённо от внутреннего содержания
			\item Ориентирован на снижение затрат на второстепенные регулярные процессы
		\end{itemize}
	\end{block}
\end{frame}


\begin{frame} \frametitle{Разработка продукта на заказ: Аутсорсинг}
	\begin{block}{Примеры}
		\begin{itemize}
			\item Ведение бухгалтерии, юридическое сопровождение
			\item Организация продаж: колл-центры, консультанты
			\item Логистика: доставка товаров, организация поставок
			\item Организация технической поддержки
		\end{itemize}
	\end{block}
\end{frame}


\subsection{Аутстаффинг}


\begin{frame} \frametitle{Аутстаффинг}
	\begin{block}{}
		\alert{Аутстаффинг} "--- передача заказчику сотрудников или команд для выполнения каких-либо задач без их оформления в штат
	\end{block}
	\begin{block}{Цель}
		Решение вопросов, связанных с оптимизацией штатного расписания и оперирования бюджетом компании, а также снижением рисков, связанных с решением трудовых споров
	\end{block}
\end{frame}


\begin{frame} \frametitle{Аутстаффинг}
	\begin{block}{Особенности}
		\begin{itemize}
			\item В качестве продукта "--- разработчики или команда разработчиков, а не конкретное программное решение
			\item Оплата происходит в зависимости от числа работников и степени их участия в проекте (частичная или полная занятость, оплата за часы)
		\end{itemize}
	\end{block}
\end{frame}

\lecturenotes

Аутстаффинг (англ. out "--- <<вне>> + англ. staff "--- <<персонал>>) "--- возможность использования рабочей силы другого предприятия. Компания-заказчик не вступает в правовые отношение с персоналом организации, предоставляющей услуги аутстаффинга. При этом нанятые рабочие обязуются выполнить перечень работ, указанных в договоре. Таким образом, организация, занимающаяся аутстаффингом, предоставляет во временное распоряжение часть персонала предприятия. Место, время и объём необходимых работ в договоре указывает заказчик. То есть работники формально числятся в штате одной организации, а фактически выполняют работу другого предприятия~\cite{Delatdelo}.


\begin{frame} \frametitle{Аутстаффинг}
	\begin{block}{Бизнес-процессы}
		Включают в себя поиск и обучение кадров, а также стратегическое управление компанией. Остальные бизнес-процессы реализуются заказчиком
	\end{block}
	\begin{block}{Прибыль}
		Источником генерации прибыли является компетенция сотрудников и время их работы
	\end{block}
\end{frame}


\begin{frame} \frametitle{Аутстаффинг}
	\begin{block}{Условия}
		\begin{itemize}
			\item Отсутствие желания или возможности формирования команды и поиска кадров
			\item Понимание предметной области, готовность управлять процессом разработки
			\item Необходимость временно и оперативно увеличить штат с минимальными издержками
			\item Необходимость в подключении готовой функциональной команды
		\end{itemize}
	\end{block}
\end{frame}


\begin{frame} \frametitle{Аутстаффинг}
	\begin{block}{Примеры}
		\begin{itemize}
			\item Аренда сотрудников для оперативного выпуска новой версии ПО или исправления критических проблем
			\item Временное привлечение сертифицированных специалистов с полным погружением
			\item Разработка ПО для собственного оборудования на стадии создания
		\end{itemize}
	\end{block}
\end{frame}


\subsection{ИТ-консалтинг}


\begin{frame} \frametitle{ИТ-консалтинг}
	\begin{block}{}
		\alert{ИТ-консалтинг} "--- консультирование, направленное на  информационную поддержку бизнес-процессов, позволяющее получить независимую экспертную оценку проекта и его реализации, используемых технологий
	\end{block}
	\begin{block}{Особенности}
		\begin{itemize}
			\item В качестве продукта "--- экспертное мнение или иной результат анализа
			\item Чаще всего носит разовый, несистемный характер
			\item Узкая специализация исполнителя
		\end{itemize}
	\end{block}
\end{frame}

\lecturenotes

ИТ-консалтинг (англ. IT-consulting) "--- консалтинг в сфере информационных технологий (ИТ). Является одним из многочисленных направлений консалтинга (консалтинговых услуг).

ИТ-консалтинг "--- проектно-ориентированная деятельность, связанная с информационной поддержкой бизнес-процессов, позволяющая дать независимую экспертную оценку эффективности использования информационных технологий.

На сегодняшний день большинство компаний использует ИТ в управлении своим бизнесом. Информационные технологии позволяют делать бизнес более наглядным, более управляемым, более прогнозируемым.

ИТ-консалтинг "--- это услуга, которую предлагают ИТ-компании (как правило в вопросах комплексных проектов), а также независимые эксперты в том или ином направлении IT (обычно в узком спектре, например, защита от DDOS атак).

Услуга по предоставлению ИТ-консалтинга, как правило, включает следующие пункты:
\begin{itemize}
	\item оптимизация затрат на внедрение информационных технологий, ИТ-решений в рамках компании;
	\item повышение эффективности бизнес-процессов компании;
	\item повышение управляемости, прозрачности деятельности организации за счёт создания единой инфраструктуры (ИТ-инфраструктуры);
	\item внедрение систем уровня предприятия (ERP, CRM, Business Intelligence, Groupware-системы, NIS-системы);
	\item ИТ-аудит (оценка уровня автоматизации)~\cite{Wiki2}.
\end{itemize}


\begin{frame} \frametitle{ИТ-консалтинг}
	\begin{block}{Бизнес-процессы}
		Исключаются бизнес-процессы разработки, увеличивается доля бизнес-процессов анализа требований и контроля качества продукта, его конкурентоспособности.\\
		Растёт значение поиска и обучения кадров, стратегическое планирование и позиционирование компании на рынке консалт-услуг
	\end{block}
	\begin{block}{Прибыль}
		Экспертность кадров генерирует прибыль компании
	\end{block}
\end{frame}


\begin{frame} \frametitle{ИТ-консалтинг}
	\begin{block}{Условия}
		\begin{itemize}
			\item Аудит программного решения
			\item Кратковременное привлечение эксперта
			\item Формальная верификация используемых алгоритмов
			\item Анализ бизнес-рисков связанных с эксплуатацией ПО
			\item Анализ и оптимизация бизнес-процессов ИТ-подразделений и компаний
			\item Оптимизация затрат на ИТ-подразделение и инфраструктуру
			\item Формирование решений на основе потребностей заказчика, их оценка
		\end{itemize}
	\end{block}
\end{frame}


\begin{frame} \frametitle{ИТ-консалтинг}
	\begin{block}{Примеры}
		\begin{itemize}
			\item Аудит безопасности ПО
			\item Разработка технических заданий и документации
			\item Анализ структуры бизнес-процессов ИТ-компаний
			\item Оценка эффективности от внедрения
		\end{itemize}
	\end{block}
\end{frame}


\begin{frame} \frametitle{Заключение}
	\begin{itemize}
		\item Процессный подход предназначен для формализации основных процессов бизнеса, с целью анализа и повышениях их эффективности
		\item Это позволяет фокусироваться на процессах, генерирующих непосредственную ценность и прибыль компании, снизив долю рутины, минимизировать риски за счет систематизации рабочего процесса
		\item Использование шаблонов с учетом специфики конкретного бизнеса облегчает выделение бизнес-процессов конкретной компании
		\item В результате мы получаем снижение себестоимости продукта труда, обеспечивая базис конкурентоспособности на рынке
	\end{itemize}
\end{frame}


\begin{thebibliography}{99}
\bibitem{BStudio} \href{http://www.businessstudio.com.ua/bp/bpshort.php}{Краткая теория о бизнес-процессе.}
\bibitem{Ekon123} \href{http://jurnal.org/articles/2015/ekon123.html}{Авдеев~Л.А., Шишкин~С.А. Преимущества процессного подхода к управлению перед функциональным управлением.}
\bibitem{Vseproip} \href{http://vseproip.com/biznes-ip/maluy/klassifikaciya-biznes-processov.html}{Бизнес-процессы и их классификация.}
\bibitem{Habr1} \href{https://habr.com/post/256915/}{Подготовительный этап разработки программного обеспечения.}
\bibitem{Centeryf} \href{http://center-yf.ru/data/Marketologu/Biznes-process-planirovaniya.php}{Бизнес процесс планирования.}
\bibitem{Abolut} \href{http://ab-solut.net/ru/articles/etapi_po/}{Этапы разработки программного обеспечения.}
\bibitem{Qalight} \href{https://qalight.com.ua/baza-znaniy/stadii-tsikla-razrabotki-po/}{Стадии цикла разработки ПО.}
\bibitem{Habr2} \href{https://habr.com/post/205342/}{Что такое качество кода и зачем его измерять.}
\bibitem{Habr3} \href{https://habr.com/company/trinion/blog/242747/}{Внедрение программного продукта. Особенности работы бизнес-консультанта.}
\bibitem{Wiki1} \href{https://ru.wikipedia.org/wiki/Сопровождение_программного_обеспечения}{Сопровождение программного обеспечения.}
\bibitem{Lektsii} \href{https://lektsii.com/2-40143.html}{Преимущества и недостатки самостоятельной разработки информационной системы.}
\bibitem{Delatdelo} \href{https://delatdelo.com/spravochnik/terminy/chto-takoe-autstaffing-personala.html}{Что такое аутстаффинг персонала.}
\bibitem{Wiki2} \href{https://ru.wikipedia.org/wiki/ИТ-консалтинг}{ИТ-консалтинг.}
\end{thebibliography}


\end{document}

%%% Local Variables: 
%%% mode: TeX-pdf
%%% TeX-master: t
%%% End: 
