\documentclass{../industrial-development}
\graphicspath{{01-business-processes/}}

\title{Бизнес-процессы компании по разработке программного обеспечения}
\author{Новиков Денис Александрович, \\Мехтиханов Леонид Игоревич, \\ПИ-21 МО}
\date{}

\begin{document}

\begin{frame}
  \titlepage
\end{frame}


\section{Ключевые задачи, стоящие перед компаниями, и их реализация в бизнес-процессах}

\subsection{Понятие бизнес-процесса}


\begin{frame} \frametitle{Понятие бизнес-процесса}
	\begin{block}{Определение}
		\alert{Бизнес-процесс} "--- логически завершённая цепочка взаимосвязанных и повторяющихся видов деятельности, в результате которых создаётся продукт или сопутствующая услуга.
	\end{block}
	\begin{block}{Свойства}
		\begin{itemize}
			\item Может являться частью более общего бизнес-процесса
			\item Потребителем результатов его деятельности может выступать другой бизнес-процесс
		\end{itemize}
	\end{block}
\end{frame}
\lecturenotes


\begin{frame} \frametitle{Преимущества процессного подхода}
	В отличие от иных форм моделирования, процессный подход нацелен на:
	\begin{itemize}
		\item удовлетворение требований клиента
		\item снижение затрат на оперативное управление
		\item анализ и выявление узких мест и резервов работы
		\item снижение затрат на оперативное управление
		\item создание эталонов, стандартов действий персонала
		\item упрощение масштабирования бизнеса
	\end{itemize}
\end{frame}
\lecturenotes


\subsection{Классификация бизнес-процессов}


\begin{frame} \frametitle{Классификация бизнес-процессов}
	По функциональному назначению бизнес-процессы разработки ПО делятся на:
	\begin{itemize}
		\item \alert{Основные процессы}
		\item \alert{Вспомогательные процессы}
		\item \alert{Управляющие процессы}
	\end{itemize}
\end{frame}
\lecturenotes


\begin{frame} \frametitle{Основные процессы}
	\begin{block}{Определение}
		Направлены на создание, эксплуатацию и поддержку ПО. Генерируют ценность для клиента и прибыль для компании.
	\end{block}
	\begin{block}{Примеры}
		\begin{itemize}
			\item Анализ потребностей клиента
			\item Формирование требований к ПО
			\item Разработка ПО
			\item Поддержка и сопровождение продукта
		\end{itemize}
	\end{block}
	\begin{block}{Результат}
		ПО или услуга удовлетворяющее потребностям заказчика
	\end{block}
\end{frame}
\lecturenotes


\begin{frame} \frametitle{Вспомогательные процессы}
	\begin{block}{Определение}
		Обеспечивают ресурсы для основных процессов. Не создают продукт напрямую, но необходимы для его создания.
	\end{block}
	\begin{block}{Примеры}
		\begin{itemize}
			\item Поиск и подготовка кадров
			\item Организация технической и иной инфраструктуры
			\item Организация труда, оснащение рабочих мест
		\end{itemize}
	\end{block}
	\begin{block}{Результат}
		Ресурсы, необходимые для функционирования основных процессов (разработка и сопровождение ПО)
	\end{block}
\end{frame}
\lecturenotes


\begin{frame} \frametitle{Управляющие процессы}
	\begin{block}{Определение}
		Обеспечивают координацию и слаженную работу бизнес-процессов и функциональных единиц компании.
	\end{block}
	\begin{block}{Примеры}
		\begin{itemize}
		\item Планирование
		\item Коммуникация (мотивация и поддержка сотрудников)
		\item Реализация и контроль выполнения плана действий
		\item Оценка результатов
		\end{itemize}
	\end{block}
	\begin{block}{Результат}
		Организационная структура и управленческие решения, обеспечивающие функционирование компании
	\end{block}
\end{frame}
\lecturenotes


\subsection{Бизнес-процессы создания ПО}


\begin{frame} \frametitle{Бизнес-процессы создания ПО. Подготовка}
	\begin{itemize}
		\item Бизнес-процесс подготовки необязателен и в ряде случаев является частью бизнес-процесса анализа
		\item Применяется когда затраты на более строгий анализ слишком высоки, позволяет отказаться от нереализуемых и потенциально невостребованных продуктов на ранних этапах
	\end{itemize}
\end{frame}
\lecturenotes


\begin{frame} \frametitle{Бизнес-процессы создания ПО. Подготовка}
	\begin{block}{Предназначение}
		\begin{itemize}
			\item Анализ целей на основе исходной идеи
			\item Анализ востребованности будущего продукта
			\item Анализ рисков
			\item Анализ предварительного технического решения
			\item Анализ трудозатрат и необходимых ресурсов
			\item Анализ реализуемости
		\end{itemize}
	\end{block}
	\begin{block}{Результат}
		Решение, о том, стоит ли продолжать работы
	\end{block}
\end{frame}
\lecturenotes


\begin{frame} \frametitle{Бизнес-процессы создания ПО. Анализ и проектирование}
	\begin{block}{Предназначение}
		\begin{itemize}
			\item Определение потребностей клиента и путей их удовлетворения
			\item Оценка существующих решений, их соответствия поставленной задаче
			\item Разработка архитектуры решения, определение используемых технологий
			\item Формирование технического задания, спецификации
		\end{itemize}
	\end{block}
	\begin{block}{Результат}
		Формальные требования к разрабатываемому ПО, технический проект (техническое задание)
	\end{block}
\end{frame}
\lecturenotes


\begin{frame} \frametitle{Бизнес-процессы создания ПО. Планирование}
	\begin{block}{Предназначение}
		\begin{itemize}
			\item Формирование оптимального плана работ, выделение основных этапов, условий их завершений
			\item Оценка временных и стоимостных затрат при различных подходах, с учётом доступных ресурсов
		\end{itemize}
	\end{block}
	\begin{block}{Результат}
		Целевой план работ, план бюджетирования
	\end{block}
\end{frame}
\lecturenotes


\begin{frame} \frametitle{Бизнес-процессы создания ПО. Технико-экономического обоснование}
	\begin{block}{Предназначение}
		\begin{itemize}
			\item Определение эффекта от внедрения, его оценка с учётом предполагаемого срока эксплуатации
			\item Оценка рентабельности инвестиций, например, путём расчёта соотношения эффект / затраты
			\item Планирование бюджетирования
		\end{itemize}
	\end{block}
	\begin{block}{Результат}
		Оценка экономического эффекта, определение необходимости дальнейшей работы над проектом
	\end{block}
\end{frame}
\lecturenotes


\begin{frame} \frametitle{Бизнес-процессы создания ПО. Разработка}
	\begin{block}{Предназначение}
		\begin{itemize}
			\item Реализация архитектуры проекта
			\item Реализация функционала ПО в соответствии со спецификацией, техническим заданием
			\item Разработка документации
		\end{itemize}
	\end{block}
	\begin{block}{Результат}
		Программное обеспечение
	\end{block}
\end{frame}
\lecturenotes


\begin{frame} \frametitle{Бизнес-процессы создания ПО. Контроль качества}
	\begin{block}{Определение}
		\alert{Качество ПО} (Software Quality) "--- степень, в которой программное обеспечение обладает требуемой комбинацией свойств.
		%[1061-1998 IEEE Standard for Software Quality Metrics Methodology]
	\end{block}
	\begin{block}{Определение}
		\alert{Обеспечение качества} (Quality Assurance) "--- совокупность мероприятий, предпринимаемых на разных стадиях жизненного цикла ПО для обеспечения требуемого уровня качества выпускаемого продукта.
	\end{block}
\end{frame}
\lecturenotes


\begin{frame} \frametitle{Бизнес-процессы создания ПО. Контроль качества}
	\begin{block}{Предназначение}
		\begin{itemize}
			\item Проверка корректности реализации, соответствия спецификации и техническому заданию
			\item Проверка соответствия тестовых показателей нормативам, анализ их динамики
			\item Контроль качества программного кода
			\item Контроль актуальности документации
		\end{itemize}
	\end{block}
	\begin{block}{Результат}
		Информация о фактическом состоянии продукта, мере его соответствия целевым показателям
	\end{block}
\end{frame}
\lecturenotes


\begin{frame} \frametitle{Бизнес-процессы создания ПО. Внедрение}
	\begin{block}{Предназначение}
		\begin{itemize}
			\item Финальная проверка продукта на соответствие спецификации
			\item Проведение приёмочных испытаний ПО
			\item Передача ПО и документации заказчику
		\end{itemize}
	\end{block}
	\begin{block}{Результат}
		Запуск системы в реальных условиях
	\end{block}
\end{frame}
\lecturenotes


\begin{frame} \frametitle{Бизнес-процессы создания ПО. Эксплуатация}
	\begin{block}{Предназначение}
		\begin{itemize}
			\item Извлечение ценности из продукта заказчиком
			\item Получение обратной связи, переход к процессу разработки при необходимости
			\item Завершение в случае, когда ПО перестаёт удовлетворять требованиям заказчика
		\end{itemize}
	\end{block}
	\begin{block}{Результат}
		Эксплуатация в реальных условиях, выполнение бизнес-функций заказчика
	\end{block}
\end{frame}
\lecturenotes


\begin{frame} \frametitle{Бизнес-процессы создания ПО. Завершение эксплуатации}
	\begin{itemize}
		\item Происходит когда продукт не удовлетворяет требованиям заказчика и его доработка невозможна или нерентабельна
		\item Это может произойти по ряду причин: изменение потребностей бизнеса, конъюнктуры рынка, технологические изменения
	\end{itemize}
\end{frame}
\lecturenotes


\begin{frame} \frametitle{Бизнес-процессы создания ПО. Завершение эксплуатации}
	\begin{block}{Предназначение}
		\begin{itemize}
			\item Анализ опыта с целью дальнейшей оптимизации бизнес-процессов создания ПО
			\item Завершение работы над проектом и остановка сопровождения
		\end{itemize}
	\end{block}
	\begin{block}{Результат}
		Полное прекращение деятельности в рамках проекта
	\end{block}
\end{frame}
\lecturenotes


\section{Типичные схемы организации бизнеса}

\subsection{Разработка собственного продукта}


\begin{frame} \frametitle{Разработка собственного продукта}
	\begin{block}{Определение}
		\alert{Разработка собственного продукта} "--- подход к разработке, при котором компания самостоятельно определяет требования к продукту, порядок и способ их реализации.
	\end{block}
	\begin{block}{Сценарии}
		\begin{itemize}
			\item Разработка ПО для внутренних нужд
			\item Разработка ПО для дальнейшей массовой реализации в виде типового решения
		\end{itemize}
	\end{block}
\end{frame}
\lecturenotes


\begin{frame} \frametitle{Разработка собственного продукта}
	\begin{block}{Особенности}
		\begin{itemize}
			\item Заказчиком, формирующим требования, является сама компания
			\item Наличие компетенции (экспертности) в предметной области
			\item Продукт динамичен и срок его эксплуатации велик
			\item Большая доля управляющих процессов в цикле разработки
			\item Преобладание итеративного подхода к разработке
			\item Внутреннее бюджетирование
		\end{itemize}
	\end{block}
\end{frame}
\lecturenotes


\begin{frame} \frametitle{Разработка собственного продукта. ПО для внутренних нужд}
	\begin{block}{Бизнес-процессы}
		Полный цикл бизнес-процессов разработки: основных, вспомогательных и управляющих.
	\end{block}
	\begin{block}{Прибыль}
		Продукт генерирует прибыль косвенно за счёт оптимизации бизнес-процессов компании или их реализации.
	\end{block}
\end{frame}
\lecturenotes


\begin{frame} \frametitle{Разработка собственного продукта. ПО для внутренних нужд}
	\begin{block}{Условия}
		\begin{itemize}
			\item На рынке отсутствует необходимое ПО
			\item Эксплуатация и\textbackslash или поддержка существующих решений слишком дорога
			\item Решение носит стратегический характер и риски связанные с прекращением поддержки внешним разработчиком неприемлемы
		\end{itemize}
	\end{block}
\end{frame}
\lecturenotes


\begin{frame} \frametitle{Разработка собственного продукта. ПО для внутренних нужд}
	\begin{block}{Примеры}
		\begin{itemize}
			\item ПО для объектов критической инфраструктуры: АЭС, космос, военные разработки
			\item Узкоспециализированное ПО для уникального оборудования
			\item ПО для инновационных сфер: робототехника
		\end{itemize}
	\end{block}
\end{frame}
\lecturenotes


\begin{frame} \frametitle{Разработка собственного продукта. ПО в качестве типового решения}
	\begin{block}{Бизнес-процессы}
		Полноценный цикл бизнес-процессов разработки: основных, вспомогательных и управляющих.\\
		Возрастает нагрузка на менеджмент, появляются дополнительные управляющие бизнес-процессы:
		\begin{itemize}
			\item стратегическое управление и планирование
			\item управление финансами
			\item управление маркетингом
		\end{itemize}
	\end{block}
	\begin{block}{Прибыль}
		Продукт генерирует прибыль напрямую за счёт продажи лицензий и сопутствующих услуг.
	\end{block}
\end{frame}
\lecturenotes


\begin{frame} \frametitle{Разработка собственного продукта. ПО в качестве типового решения}
	\begin{block}{Условия}
		\begin{itemize}
			\item На рынке отсутствует необходимое ПО, либо его качество недостаточно
			\item Наличие видения пути развития ПО и его позиционирования на рынке
			\item Наличие фатальных недостатков в существующих решениях
			\item Наличие ресурсов для продвижения и поддержки ПО при массовых продажах
			\item Эксплуатация и\textbackslash или поддержка существующих решений слишком дорога
		\end{itemize}
	\end{block}
\end{frame}
\lecturenotes


\begin{frame} \frametitle{Разработка собственного продукта. ПО в качестве типового решения}
	\begin{block}{Примеры}
		\begin{itemize}
			\item CMS для веб-сайтов: 1C-Битрикс, ReadyScript, NetCat
			\item Платформы для бизнеса: 1C-Предприятие, Галактика, Microsoft Dynamics Axapta
			\item Игровые движки: CryEngine, Unity, Unreal Engine, Dunia
		\end{itemize}
	\end{block}
\end{frame}
\lecturenotes


\subsection{Разработка продукта на заказ}


\begin{frame} \frametitle{Разработка продукта на заказ}
	\begin{block}{Определение}
		\alert{Разработка продукта на заказ} "--- подход к организации разработки, при котором источником генерации прибыли является ПО, соответствующее требованиям внешнего заказчика.
	\end{block}
	\begin{block}{Сценарии}
		\begin{itemize}
			\item Разработка нового ПО
			\item Доработка существующих решений
			\item Аутсорсинг
		\end{itemize}
	\end{block}
\end{frame}
\lecturenotes


\begin{frame} \frametitle{Разработка продукта на заказ}
	\begin{block}{Особенности}
		\begin{itemize}
			\item Требования формирует внешний заказчик
			\item Контроль осуществляется как исполнителем, так и заказчиком
			\item Отсутствие необходимости в продвижении продукта
			\item Внешнее бюджетирование
		\end{itemize}
	\end{block}
\end{frame}
\lecturenotes


\begin{frame} \frametitle{Разработка продукта на заказ. Создание новых решений}
	\begin{block}{Бизнес-процессы}
		Весь спектр, за исключением упрощения процесса анализа требований "--- исходные сведения предоставляются заказчиком.\\
		Снижается объём управляющих процессов "--- отпадает необходимость анализе бизнес-концепции продукта, стратегии продвижения продукта на рынке и т.д.
	\end{block}
\end{frame}
\lecturenotes


\begin{frame} \frametitle{Разработка продукта на заказ. Создание новых решений}
	\begin{block}{Условия}
		\begin{itemize}
			\item На рынке отсутствует необходимое ПО, либо его качества не удовлетворяют заказчика
			\item Нецелесообразность или невозможность доработки имеющихся решений
			\item Эксплуатация и\textbackslash или поддержка существующих решений слишком дорога
		\end{itemize}
	\end{block}
\end{frame}
\lecturenotes


\begin{frame} \frametitle{Разработка продукта на заказ. Создание новых решений}
	\begin{block}{Примеры}
		\begin{itemize}
			\item Разработка веб-сайтов и мобильных приложений
			\item Разработка ПО под инфраструктуру заказчика
			\item Разработка уникального ПО с учётом особенностей предметной области: ПО для станков
			\item Разработка специфичных системы бизнес-аналитики
		\end{itemize}
	\end{block}
\end{frame}
\lecturenotes


\begin{frame} \frametitle{Разработка продукта на заказ. Доработка существующих решений}
	\begin{block}{Бизнес-процессы}
		В отличии от создания новых решений, появляется необходимость анализа совместности требований и возможности существующего решения.
	\end{block}
	\begin{block}{Прибыль}
		Продукт генерирует прибыль напрямую за счёт продажи лицензий и сопутствующих услуг.
	\end{block}
\end{frame}
\lecturenotes


\begin{frame} \frametitle{Разработка продукта на заказ. Доработка существующих решений}
	\begin{block}{Условия}
		\begin{itemize}
			\item Наличие опыта работы с подобными решениями
			\item Наличие технической возможности
			\item Экономическая целесообразность доработки
		\end{itemize}
	\end{block}
\end{frame}
\lecturenotes


\begin{frame} \frametitle{Разработка продукта на заказ. Доработка существующих решений}
	\begin{block}{Примеры}
		\begin{itemize}
			\item Создание и доработка 1С конфигураций
			\item Создание и доработка имеющихся веб-сайтов и приложений
			\item Доработка существующей CRM системы под требования заказчика
		\end{itemize}
	\end{block}
\end{frame}
\lecturenotes


\begin{frame} \frametitle{Разработка продукта на заказ. Аутсорсинг}
	\begin{block}{Бизнес-процессы}
		Составляют новый автономный бизнес процесс, на основе группы бизнес-процессов заказчика. Заказчик передаёт определённую потребность бизнеса, не детализируя конкретные бизнес-процессы, направленные на её реализацию.\\
		Позволяет заказчику снизить нагрузку на менеджмент и организацию труда, снизить число вспомогательных и управляющих процессов, концентрируясь на основной деятельности.\\
		Позволяет снизить среднесписочное число сотрудников, поскольку исполнители являются работниками внешней компании.
	\end{block}
\end{frame}
\lecturenotes


\begin{frame} \frametitle{Разработка продукта на заказ. Аутсорсинг}
	\begin{block}{Условия}
		\begin{itemize}
			\item Носит системный, продолжительный характер
			\item Ориентирован на получение результата, отвлечённо от внутреннего содержания
			\item Ориентирован на снижение затрат на второстепенные регулярные процессы
		\end{itemize}
	\end{block}
\end{frame}
\lecturenotes


\begin{frame} \frametitle{Разработка продукта на заказ. Аутсорсинг}
	\begin{block}{Примеры}
		\begin{itemize}
			\item Ведение бухгалтерии, юридическое сопровождение
			\item Организация продаж: колл-центры, консультанты
			\item Логистика: доставка товаров, организация поставок
			\item Организация технической поддержки
		\end{itemize}
	\end{block}
\end{frame}
\lecturenotes


\subsection{Аутстаффинг}


\begin{frame} \frametitle{Аутстаффинг}
	\begin{block}{Определение}
		\alert{Аутстаффинг} "--- передача заказчику сотрудников или команд для выполнения каких-либо задач без их оформления в штат. 
	\end{block}
	\begin{block}{Цель}
		Решение вопросов, связанных с оптимизацией штатного расписания и оперирования бюджетом компании, а также снижением рисков, связанных с решением трудовых споров.
	\end{block}
\end{frame}
\lecturenotes


\begin{frame} \frametitle{Аутстаффинг}
	\begin{block}{Особенности}
		\begin{itemize}
			\item В качестве продукта "--- разработчики или команда разработчиков, а не конкретное программное решение
			\item Оплата происходит в зависимости от числа работников и степени их участия в проекте (частичная или полная занятость, оплата за часы)
		\end{itemize}
	\end{block}
\end{frame}
\lecturenotes


\begin{frame} \frametitle{Аутстаффинг}
	\begin{block}{Бизнес-процессы}
		Исключаются основные, значительно сокращаются вспомогательные "--- остаётся поиск кадров, их обучение.\\
		Снижается объём управляющих процессов "--- отпадает необходимость в анализе и формировании требований к ПО (этим занимается заказчик), остаётся стратегическое управление.
	\end{block}
	\begin{block}{Прибыль}
		Источником генерации прибыли является компетенция сотрудников и время их работы.
	\end{block}
\end{frame}
\lecturenotes


\begin{frame} \frametitle{Аутстаффинг}
	\begin{block}{Условия}
		\begin{itemize}
			\item Отсутствие желания или возможности формирования команды и поиска кадров
			\item Понимание предметной области, готовность управлять процессом разработки
			\item Необходимость временно и оперативно увеличить штат с минимальными издержками
			\item Необходимость в подключении готовой функциональной команды
		\end{itemize}
	\end{block}
\end{frame}
\lecturenotes


\begin{frame} \frametitle{Аутстаффинг}
	\begin{block}{Примеры}
		\begin{itemize}
			\item Аренда сотрудников для оперативного выпуска новой версии ПО или исправления критических проблем
			\item Временное привлечение сертифицированных специалистов с полным погружением
			\item Разработка ПО для собственного оборудования на стадии создания
		\end{itemize}
	\end{block}
\end{frame}
\lecturenotes


\subsection{ИТ-консалтинг}


\begin{frame} \frametitle{ИТ-консалтинг}
	\begin{block}{Определение}
		\alert{ИТ-консалтинг} "--- консультирование, направленное на  информационную поддержку бизнес-процессов, позволяющее получить независимую экспертную оценку проекта и его реализации, используемых технологий.
	\end{block}
	\begin{block}{Особенности}
		\begin{itemize}
			\item В качестве продукта "--- экспертное мнение или иной результат анализа
			\item Чаще всего носит разовый, несистемный характер
			\item Узкая специализация исполнителя
		\end{itemize}
	\end{block}
\end{frame}
\lecturenotes


\begin{frame} \frametitle{ИТ-консалтинг}
	\begin{block}{Бизнес-процессы}
		Исключаются бизнес-процессы разработки, увеличивается доля бизнес-процессов анализа требований и контроля качества продукта, его конкурентоспособности.\\
		Растёт значение поиска и обучения кадров, стратегическое планирование и позиционирование компании на рынке консалт-услуг.
	\end{block}
	\begin{block}{Прибыль}
		Экспертность кадров генерирует прибыль компании.
	\end{block}
\end{frame}
\lecturenotes


\begin{frame} \frametitle{ИТ-консалтинг}
	\begin{block}{Условия}
		\begin{itemize}
			\item Аудит программного решения
			\item Кратковременное привлечение эксперта
			\item Формальная верификация используемых алгоритмов
			\item Анализ бизнес-рисков связанных с эксплуатацией ПО
			\item Анализ и оптимизация бизнес-процессов ИТ-подразделений и компаний
			\item Оптимизация затрат на ИТ-подразделение и инфраструктуру
			\item Формирование решений на основе потребностей заказчика, их оценка
		\end{itemize}
	\end{block}
\end{frame}
\lecturenotes


\begin{frame} \frametitle{ИТ-консалтинг}
	\begin{block}{Примеры}
		\begin{itemize}
			\item Аудит безопасности ПО
			\item Разработка технических заданий и документации
			\item Анализ структуры бизнес-процессов ИТ-компаний
			\item Оценка эффективности от внедрения
		\end{itemize}
	\end{block}
\end{frame}
\lecturenotes


%\begin{frame} \frametitle{Пример слайда}
%  \begin{block}{Важный факт}
%    Промышленная разработка является коллективным процессом\dots
%  \end{block}
%  
%  \begin{itemize}
%  \item Тезис 1\dots
%  \item Тезис 2\dots
%  \item Важный текст на слайдах можно \alert{выделять}\dots
%  \end{itemize}
%\end{frame}
%\lecturenotes
%Текст конспекта, относящийся к слайду с указанием источника~\cite[с.~97--99]{Brooks}.
%На интернет-источники можно ссылаться не по ГОСТу, но с обязательной гиперссылкой~\cite{Fowler}.


%\begin{thebibliography}{99}
%\bibitem{Brooks} Брукс Ф. Мифический человеко-месяц или как создаются программные системы. СПб~: Символ-плюс, 2000.
%\bibitem{Fowler} \href{https://martinfowler.com/articles/injection.html}{Fowler M. Inversion of Control Containers and the Dependency Injection pattern.}
%\end{thebibliography}

\end{document}

%%% Local Variables: 
%%% mode: TeX-pdf
%%% TeX-master: t
%%% End: 
