\documentclass[lecturenotes]{../industrial-development}
\graphicspath{{4-developers-activity/}}

\title{Деятельность разработчика в компаниях ИТ-индустрии}
\author{Соколов Роман, Садикова Анастасия}
\date{}

\begin{document}

\begin{frame}
  \titlepage
\end{frame}


\section{Разработчик в узком и широком смыслах}
\subsection{Определение разработчика в~узком~смысле}
\begin{frame} \frametitle{Определение разработчика в~узком~смысле}
	\begin{block}{Разработчик в узком смысле}
		 Человек, который занимается написанием кода (программированием) и,~иногда, его тестированием
	\end{block}
\end{frame}
\lecturenotes
\\В узком смысле программирование рассматривается как кодирование --- реализация в виде программы одного или нескольких взаимосвязанных алгоритмов (в современных условиях это осуществляется с применением языков программирования). 


\subsection{Определение разработчика в широком смысле}
\begin{frame} \frametitle{Определение разработчика в~широком~смысле}
	\begin{block}{Разработчик в широком смысле}
		 Человек, который занимается разработкой ПО
	\end{block}
\end{frame}
\lecturenotes
В более широком смысле процесс программирования охватывает создание и моделирование (то есть разработку) алгоритмов и анализ потребностей будущих пользователей программного обеспечения. В его число обязанностей может входить: практически все фазы жизненного цикла (но в той степени, в которой это касается разработчика), самостоятельная постановка задач, высокоуровневая картина продукта, широкая кооперация с коллегами, возможно даже взаимодействие с заказчиком.


\section{Типичные задачи разработчика ПО}
\subsection{Обучение}
\begin{frame} \frametitle{Определение обучения в~узком~смысле}
	\begin{block}{Обучение в узком смысле}
	 Процесс изучения новых технологий с целью поддержания минимального уровня квалификации, необходимого для работы над~поставленной задачей
	\end{block}{}
	\vspace{\baselineskip}
	\begin{itemize}
		\item Выполнение задач от наставника
		\item Посещение митапов
		\item Посещение курсов и мероприятий 
		\item Самостоятельное изучение новых областей программирования
	\end{itemize}
	
\end{frame}
\lecturenotes
\\Обучение в ИТ-компании это обязательный процесс для каждого разработчика, поскольку необходимый уровень квалификации для выполнения новых задач динамически меняется.\\
Как правило, будущие разработчики сперва нанимаются на должность стажера, которым выделяется наставник --- человек, непосредственно модерирующий процесс обучения и помогающий разобраться с возникшими (как правило типичными) проблемами.\\
Часто хорошие компании устраивают мероприятия по обмену общими знаниями, где от разработчика требуется как делиться своим опытом с другими сотрудниками, так и самому приобретать новые знания. Данный процесс так же включает в себя и самостоятельное изучение новых областей разработки, методик, языков и т.д., но это остается на совести самого разработчика.  

\begin{frame} \frametitle{Определение обучения в~широком~смысле}
	\begin{block}{Обучение в широком смысле}
		 Процесс поднятия квалификации для последующего, более качественного, создания ПО
	\end{block}{}
\end{frame}
\lecturenotes
\\В отличии от узкого смысла, в широком смысле обучение разработчика проводится не для поддержания минимального уровня квалификации, а для развития в целом и продвижения по карьерной лестнице.\\
Посещение митапов и курсов для разработчиков, в основном, проводятся для рассказа о новых технологиях, освоенных другими разработчиками, их опыта применения и с какими трудностями они столкнулись. Часто в компаниях проводятся сертификации и экзамены с целью повышения текущего уровня знаний разработчика.\\
Благодаря общему взгляду «сверху» на проект, разработчик может заранее понять какие новые технологии возможно применить или потребуется применить в текущем проекте и предварительно изучить их самостоятельно, либо назначить изучение другому разработчику, которому предстоит решение данной задачи.\\
В целом, разработчикам, которые хотят оставаться актуальными в своей профессии, необходимо систематически самостоятельно изучать новые технологии своей среды.


\subsection{Разработка архитектуры ПО}
\begin{frame} \frametitle{Определение разработки архитектуры ПО в~широком~смысле}
  \begin{block}{Разработка архитектуры ПО в широком смысле}
  	Предварительное утверждение будущей логики разрабатываемого ПО
  \end{block}{}	
\end{frame}
\lecturenotes
\\Разработка архитектуры ПО --- процесс подбора наиболее подходящей парадигмы разработки программы для конкретной поставленной задачи. Перед непосредственным написанием кода ПО, ведущему разработчику необходимо создать концепцию общей работы структур данных и логики. Процесс выбора архитектуры заключается в подборе наиболее оптимальной парадигмы, как правило, на основании опыта участвующих в процессе разработчиков.
Основными факторами выбора архитектуры являются риски: недостаточный функционал, переносимость, способность команды справиться с задачей, недостаточная оптимизация, некорректная выполнение задач, низкая модифицируемость и т.д. Как правило, решение принимается на общем обсуждении командой разработчиков.

\begin{frame} \frametitle{Определение разработки архитектуры ПО в~узком~смысле}
	\begin{block}{Разработка архитектуры ПО в узком смысле}
		Разработка архитектуры ПО в узком смысле невозможна, поскольку требует широкого взгляда на весь проект
	\end{block}{}	
\end{frame}
\lecturenotes
\\Выбор архитектуры --- один из самых важных пунктов при разработке ПО, поскольку от этого будет зависеть не только процесс разработки, но и работоспособность после релиза. Данное решение невозможно принять разработчику в узком смысле, поскольку не входит в число строго хорошо поставленных задач тимлидом.

\subsection{Оценка времени}
\begin{frame} \frametitle{Определение оценки времени разработки ПО в~узком~смысле}
	\begin{block}{Оценка времени разработки ПО в узком смысле}
		Процесс, когда разработчик оценивает сложность задачи с~учетом его сил и опыта, и обозначает сколько времени ему понадобится на решение
	\end{block}{}
\end{frame}
\lecturenotes
\\Оценка времен --- обязательный пункт в процессе разработки, в котором разработчик, на основании своего опыта, делает предположение о количестве времени, необходимого для решения поставленной ему задачи.

\begin{frame} \frametitle{Понятие оценки времени разработки ПО в~широком~смысле}
	\begin{block}{Оценка времени разработки ПО в широком смысле}
		Декомпозиция всего процесса разработки на более мелкие задачи, где итоговое время равно сумме времени, необходимого на каждую отдельную задачу
	\end{block}{}
\end{frame}
\lecturenotes
\\Поскольку в любом договоре при разработке ПО требуется указать конкретное время выполнения работы, грамотная оценка необходимого времени является очень важными моментом. Неграмотная оценка может привести к нехватке времени и срыванию сроков, что негативно скажется как на итоговой прибыли, так и на репутации и отношении с клиентом. Поэтому разработчику необходимо ответственно подходить к данному вопросу.\\ 
Процесс оценки времени на разработку проекта состоит из декомпозиции общей сложной задачи на подзадачи, необходимое время на которые уточняются у назначенных на них разработчиков, прибавляя к этому затраты на риски или накладные расходы (например: тестирование задачи, тимлидинг, рефакторинг и тд), а затем суммируются и формируют итоговое время. Однако поскольку оценка времени очень неоднозначна и носит очень субъективный характер, она не всегда может совпадать с предполагаемым временем, необходимым для решения задачи. Поэтому многие опытные разработчики, как правило, закладывают в общее время риски возможных проблем, перемножая итоговое время на коэффициент, выведенный ими личным опытом.

\subsection{Обсуждение задач с командой}
\begin{frame} \frametitle{Определение обсуждения задач с~командой в~узком~смысле}
	\begin{block}{Обсуждение задач с командой в узком смысле}
		Процесс предоставления итогов работы тимлиду над~поставленными задачами
	\end{block}
\end{frame}
\lecturenotes
\\Разработчик ПО периодически (как правило по окончанию спринта) должен предоставить отчет о проделанной работе руководителю проекта или своему тимлиду.\\
Как правило между спринтами организуются встречи, на которых команда разработчиков подводит итоги прошлого спринта, обсуждает результаты, ставит задачи на будущий спринт и проводит их распределение между разработчиками.

\begin{frame} \frametitle{Определение обсуждения задач с~командой в~широком~смысле}
	\begin{block}{Обсуждение задач с командой в широком смысле}
		Процесс подведения итогов работы с целью оценки текущего состояния проекта и оптимального выбора дальнейших действий
	\end{block}
\end{frame}
\lecturenotes
\\Обсуждение задач с командой --- промежуточный процесс при разработке ПО, в котором опытные разработчики рассказывают о состоянии работы на данный момент и выбирают/распределяют дальнейшие действия.\\
Поскольку в процессе разработки большого продукта ПО участвует множество людей, необходимо периодически проводить встречи, на которых оценивается общий прогресс (и/или регресс) по проекту для быстрого выявления багов, неправильной архитектуры и логики, мозговой штурм сложных моментов, обсуждение комментариев заказчика о промежуточных результатах, а также принятия решения о дальнейшей разработке и распределение конкретных задач между разработчиками. 

\subsection{Тимлидинг}
\begin{frame} \frametitle{Определение тимлидинга в~широком~смысле}
	\begin{block}{Тимлидинг в~широком~смысле}
		Процесс управления и контроля разработки ПО
	\end{block}
\end{frame}
\lecturenotes
\\Тимлид – человек осуществляющий контроль и управление разработкой ПО.\\
В число обычных задач тимлида входит:
\begin{itemize}
	\item Оценка и распределение задач между членами команды 
	\item Кодревью – чтение готового кода другого разработчика, после выполнения его задачи, с целью выявления ошибок в логике
	\item Приём выполненных задач в команде
	\item Моторинг проблем и поиск быстрого решения 
	\item Feedback от разработчиков о ходе выполнения задач
	\item Взаимодействие с клиентом 
\end{itemize}
\vspace{\baselineskip}
Задача тимлидинга не рассматривается в узком смысле.

\subsection{Тестирование}
\begin{frame} \frametitle{Определение тестирования в~узком~смысле}
	\begin{block}{Тестирование в~узком~смысле}
		Процесс самостоятельной проверки написанного кода на~соответствие поставленной задаче
	\end{block}
\end{frame}
\lecturenotes
\\Разработчик, после написания кода, должен проверить свою работу с помощью тестирования на предмет соответствия требуемой логики. На данном этапе разработчик может протестировать свой код начиная от визуальной оценки и тестовой выборки до написания модульных тестов.

\begin{frame} \frametitle{Определение тестирования в~широком~смысле}
	\begin{block}{Тестирование в~широком~смысле}
		Запланированный процесс, целью которого является выявление некорректной работы программы или её части для последующего исправления
	\end{block}
\end{frame}
\lecturenotes
\\Тестирование --- процесс проверки корректной работы программы (или её части) относительно её технического задания при разных вводимых данных.\\
Поскольку ПО должно точно следовать техническому заданию, в задачи разработчика входит проверка соответствия работы заданной спецификации. Данной проблемой обычно занимается специализированный сотрудник – тестировщик. Для тестирования используются разные методы: от визуальной проверки, до написания отдельных программ, симулирующих определенное поведение пользователя.

\begin{frame} \frametitle{Определение тестирования в~широком~смысле}
	Виды тестов в широком смысле:
	\begin{itemize}
		\item Нагрузочные
		\item Интеграционное
	\end{itemize}
\end{frame}
\lecturenotes
\\Термин «нагрузочное тестирование» может быть использован в различных значениях в профессиональной среде тестирования ПО. В общем случае он означает практику моделирования ожидаемого использования приложения с помощью эмуляции работы нескольких пользователей одновременно. Таким образом, подобное тестирование больше всего подходит для многопользовательских систем, чаще — использующих клиент-серверную архитектуру (например, веб-серверов). Однако и другие типы систем ПО могут быть протестированы подобным способом. Например, текстовый или графический редактор можно заставить прочесть очень большой документ; а финансовый пакет — сгенерировать отчёт на основе данных за несколько лет. Наиболее адекватно спроектированный нагрузочный тест даёт более точные результаты.\\
Основная цель нагрузочного тестирования заключается в том, чтобы, создав определённую ожидаемую в системе нагрузку (например, посредством виртуальных пользователей) и, обычно, использовав идентичное программное и аппаратное обеспечение, наблюдать за показателями производительности системы. 
Интеграционное тестирование – одна из фаз тестирования программного обеспечения, при которой отдельные программные модули объединяются и тестируются в группе. Обычно интеграционное тестирование проводится после модульного тестирования и предшествует системному тестированию.\\
Интеграционное тестирование в качестве входных данных использует модули, над которыми было проведено модульное тестирование, группирует их в более крупные множества, выполняет тесты, определённые в плане тестирования для этих множеств, и представляет их в качестве выходных данных и входных для последующего системного тестирования.\\ 
Целью интеграционного тестирования является проверка соответствия проектируемых единиц функциональным, приёмным и требованиям надежности. Тестирование этих проектируемых единиц — объединения, множества или группы модулей — выполняется через их интерфейс, с использованием тестирования «чёрного ящика». 

\subsection{Рефакторинг}
\begin{frame} \frametitle{Определение рефакторинга в~узком~смысле}
	\begin{block}{Рефакторинг в~узком~смысле}
		Переделывание части кода программы или её отдельных функций, логики в рамках текущей задачи
	\end{block}
\end{frame}
\lecturenotes
\\При разработке ПО, иногда возникают ситуации, при которых принимается решение переделать часть кода программы --- произвести рефакторинг. В данном процессе разработчику выделяется участок, который необходимо переписать, продумывается исправленная архитектура и заменяется на новый, улучшенный вариант.

\begin{frame} \frametitle{Определение рефакторинга в~широком~смысле}
	\begin{block}{Рефакторинг в~широком~смысле}
		Процесс улучшения читаемости кода, архитектуры, оптимизации, логики структуры разрабатываемого ПО
	\end{block}
\end{frame}
\lecturenotes
\\Цель рефакторинга — сделать код программы более легким для понимания; без этого рефакторинг нельзя считать успешным.\\
Рефакторинг нужно применять постоянно при разработке кода. Основными стимулами его проведения являются следующие задачи:\\
\begin{enumerate}
	\item Необходимо добавить новую функцию, которая недостаточно укладывается в принятое архитектурное решение;
	\item Необходимо исправить ошибку, причины возникновения которой сразу не ясны;
	\item Преодоление трудностей в командной разработке, которые обусловлены сложной логикой программы.
\end{enumerate}

\begin{frame} \frametitle{Основные виды рефакторинга}
	Основные виды рефакторинга:
	\begin{itemize}
		\item Небольшой рефакторинг 
		\item Плановый рефакторинг
	\end{itemize}
\end{frame}
\lecturenotes
\\Есть два основных вида рефакторинга:\\
\begin{itemize}
\item Небольшой рефакторинг – это недорогое по времени улучшения кода в процессе решения задачи поблизости. Пример: при осмотре функций, была найдена одна, которую можно было бы недорого по времени переписать. После согласия тимлида, функция после рефакторинга работает/читается лучше.
\item Плановый рефакторинг планомерное массовое внесение изменений в код программы в целях подготовки к внедрению нового функционала. Примеры: черная пятница, глобальная оптимизация.
\end{itemize}


\subsection{Документирование}
\begin{frame} \frametitle{Определение документирования в~узком~смысле}
	\begin{block}{Документирование в~узком~смысле}
		Процедура, фиксирующая план, процесс и результат разработки программного обеспечения
	\end{block}
\end{frame}
\lecturenotes
\\В процессе разработки ПО, разработчику всегда необходимо тем или иным образом документировать (комментировать) как промежуточные результаты, так и итоговый. При разработке ПО, разработчику часто приходится сталкиваться с чужим кодом, и, чтобы сэкономить время на изучение функционала, используют комментарии --- документацию.

\begin{frame} \frametitle{Определение документирования в~широком~смысле}
	\begin{block}{Документирование в~широком~смысле}
		Формальное описание работы разрабатываемого ПО 
	\end{block}
	\vspace{\baselineskip}
	Примеры видов документации в широком смысле:
	\begin{itemize}
		\item Дизайн-документ и Технический документ
		\item Пользовательская документация
		\item API
	\end{itemize}
\end{frame}
\lecturenotes
\\В широком смысле документирование --- это составление разных глобальных инструкций по проекту. Иногда, для создания документации, нанимают особого сотрудника --- технического писателя, он занимается написанием технического документа и пользовательской документации. Дизайн-документом занимается либо аналитик, либо менеджер, API может сгенерировать тимлид..

\subsection{Сопровождение}
\begin{frame} \frametitle{Определение сопровождения ПО в~широком~смысле}
	\begin{block}{Сопровождение ПО в~широком~смысле}
		Процесс улучшения качества или внедрения нового функционала в ПО после внедрения или релиза
	\end{block}
\end{frame}
\lecturenotes
\\Как правило, после официального релиза ПО, возникает задача исправления или дополнения функционала или логики проекта.\\
Часто процесс сопровождения похож на процесс разработки ПО: обсуждается новый функционал или исправление старого; разрабатывается архитектура, проводится тестирование и установка обновления. 

\begin{frame} \frametitle{Определение сопровождения ПО в~узком~смысле}
	\begin{block}{Сопровождение ПО в~узком~смысле}
		Процесс внесения изменений в программу после её релиза в рамках технической поддержки или плановых обновлений
	\end{block}
\end{frame}
\lecturenotes
\\Разработчику в узком смысле при сопровождении ПО выдают чётко сформулированное задание либо на исправление ошибок, либо на дополнение функционала, что очень похоже на обычную разработку ПО. 

\begin{frame} \frametitle{Типичные задачи сопровождения}
В задачи сопровождения и продвижения программного обеспечения входит:
\begin{itemize}
	\item Адаптация
	\item Расширение функционала
	\item Корректировка документации
	\item Исправление ошибок
	\item Обучение персонала
\end{itemize}
\end{frame}
\lecturenotes
\\В задачи сопровождения и продвижения программного обеспечения входит:\\
\begin{itemize}
	\item Адаптация программного обеспечения к рабочим условиям. Например, настройка удаленного доступа или адаптация под параметры аппаратуры, на которой будет установлена программа.
	\item Расширение функционала. По мере расширения компании появляется необходимость в новых функциях действующего ПО. Правильно их добавить может только специалист, который занимался разработкой.
	\item Корректировка документации. Любое изменение, которое вносится в уже действующую программу, фиксируется в документации на ПО.
	\item Исправление ошибок. Выявление всех неисправностей возможно только после полноценного ввода ПО в эксплуатацию.
	\item Обучение персонала. После внедрения программного обеспечения, мы обучаем сотрудников компании-заказчика работе со всеми функциями новой системы. Это ускоряет работу и помогает использовать ПО по-максимуму. 
\end{itemize}


\end{document}

%%% Local Variables: 
%%% mode: TeX-pdf
%%% TeX-master: t
%%% End: 
