\documentclass{../industrial-development}
\graphicspath{{4-developers-activity/}}

\title{Деятельность разработчика в компаниях ИТ-индустрии}
\author{Соколов Роман, Садикова Анастасия}
\date{}

\begin{document}

\begin{frame}
  \titlepage
\end{frame}


\section{Разработчик в узком и широком смыслах}
\subsection{Определение разработчика в узком смысле}
\begin{frame} \frametitle{Разработчик в узком смысле}
Основной задачей программиста является разработка и отладка компьютерных программ.	
\vspace{\baselineskip}

Основные виды специализации:
 \begin{itemize}
	\item Прикладные программисты
	\item Веб-программисты
	\item Системные программисты
 \end{itemize}
\end{frame}
\lecturenotes
В узком смысле слова, программирование рассматривается как кодирование --- реализация одного или нескольких взаимосвязанных алгоритмов на некотором языке программирования. Конечно, вариантов и видов программ есть великое множество и чтобы лучше объяснить, чем занимаются программисты, укажем их основные специализации:\\
\begin{itemize}
	\item Прикладные программисты — создают программное обеспечение для решения различных задач (редакторы, игры, бухгалтерские программы, CRM-системы и т.д.)
	\item Веб-программисты (чаще всего, это программисты PHP) — создают сайты, программы для управления системами сайтов или интернет-магазинов и т.п.
	\item Системные программисты — разрабатывают операционные системы и оболочки для баз данных, а также решают другие подобные задачи
\end{itemize}


\subsection{Широкое понятие деятельности разработчика}
\begin{frame} \frametitle{Широкое понятие деятельности разработчика}
Разработка программного обеспечения --- это проектирование, написание, тестирование и поддержка компьютерных программ с целью решения задач для множества пользователей; это создание надежных защищенных решений, которые выдержат испытание временем и справятся с некоторыми не известными заранее задачами, лежащими в области, близкой к очевидным исходным задачам.
\end{frame}
\lecturenotes
Разработка программного обеспечения --- это проектирование, написание, тестирование и поддержка компьютерных программ с целью решения задач для множества пользователей; это создание надежных защищенных решений, которые выдержат испытание временем и справятся с некоторыми не известными заранее задачами, лежащими в области, близкой к очевидным исходным задачам.\\
Разработчики ПО не считают своей работой просто написание программ --- они рассуждают с точки зрения удовлетворения потребностей и решения задач. И это важно, потому что не для всякой задачи необходимо писать программу: в некоторых случаях достаточно использовать уже существующую программу или объединить несколько программ. А действуя на упреждение, иногда можно вообще избавиться от необходимости решать данную задачу: разработка хороших программ часто предполагает планирование, которое позволяет предупредить появление некоторых проблем и соответствующих задач в будущем.

\subsubsection{Проектирование и разработка}
\begin{frame} \frametitle{Проектирование и разработка}
Главная задача программиста, вне зависимости от специализации — это создание продукта, а также его дальнейшее тестирование и отладка. При этом программист должен уметь составить грамотное техническое задание (ТЗ) и разбираться в соответствующей терминологии.
\end{frame}
\lecturenotes
Главная задача программиста, вне зависимости от специализации — это создание продукта, а также его дальнейшее тестирование и отладка. При этом программист должен уметь составить грамотное техническое задание (ТЗ) и разбираться в соответствующей терминологии.

\subsubsection{Внедрение и организация корректного взаимодействия с другими программами}
\begin{frame} \frametitle{Внедрение и организация корректного взаимодействия с другими программами}
Программист занимается также внедрением готового продукта, то есть запуском в работу (будь это программа, сайт или новый алгоритм). При этом он старается отследить возможные проблемы, которые могут возникнуть во время работы программы / сайта / алгоритма, предупредить или исправить ошибки, если они все же проявились.
\end{frame}
\lecturenotes
Программист занимается также внедрением готового продукта, то есть запуском в работу (будь это программа, сайт или новый алгоритм). При этом он старается отследить возможные проблемы, которые могут возникнуть во время работы программы / сайта / алгоритма, предупредить или исправить ошибки, если они все же проявились.

\subsubsection{Сопровождение}
\begin{frame} \frametitle{Сопровождение}
Еще одна обязанность программиста — разработка инструкций по работе с программами, а также оформление необходимой технической документации.
\vspace{\baselineskip}

\begin{itemize}
	\item Исправление ошибок
	\item Регламентированная адаптация ПО
	\item Модернизация 
\end{itemize}
\end{frame}
\lecturenotes
В ходе сопровождения в программу вносятся изменения, с тем, чтобы исправить обнаруженные в процессе использования дефекты и недоработки, а также для добавления новой функциональности, с целью повысить удобство использования и применимость ПО.
При проведении сопровождение программного обеспечения могут вноситься следующие изменения:\\
\begin{itemize}
\item Исправление ошибок при помощи программной корректировки;
\item Регламентированная адаптация ПО под условия определенной эксплуатации, при этом учитываются характеристики внешней среды или аппаратурной конфигурации; 
\item Модернизация, которая направлена на расширение функциональных возможностей и улучшение решения отдельно поставленных задач в рамках нового или дополнительного технического задания на программном обеспечении.
\end{itemize}

\section{Типичные задачи проектирования, разработки и сопровождения исходного кода программного обеспечения}
\subsection{Проектирование}
\begin{frame} \frametitle{Проектирование}
  \begin{block}{Проектирование программного обеспечения}
  	Процесс создания проекта программного обеспечения (ПО), а также дисциплина, изучающая методы проектирования. Проектирование ПО является частным случаем проектирования продуктов и процессов
  \end{block}
\end{frame}
\lecturenotes
Проектирование программного обеспечения — процесс создания проекта программного обеспечения (ПО), а также дисциплина, изучающая методы проектирования. Проектирование ПО является частным случаем проектирования продуктов и процессов.\\
Целью проектирования является определение внутренних свойств системы и детализации её внешних (видимых) свойств на основе выданных заказчиком требований к ПО (исходные условия задачи). Эти требования подвергаются анализу. 
Проектирование ПО включает следующие основные виды деятельности: \\
\begin{itemize}
	\item Выбор архитектуры ПО
	\item Выбор метода и стратегии решения
	\item Выбор представления внутренних данных
	\item Разработка основного алгоритма
	\item Выбор представления входных данных
\end{itemize}\\
Первоначально программа рассматривается как чёрный ящик. Ход процесса проектирования и его результаты зависят не только от состава требований, но и выбранной модели процесса, опыта проектировщика. 

\subsubsection{Выбор архитектуры}
\begin{frame} \frametitle{Выбор архитектуры}
  Хорошая архитектура это прежде всего выгодная архитектура, делающая процесс разработки и сопровождения программы более простым и эффективным. Программу с хорошей архитектурой легче расширять и изменять, а также тестировать, отлаживать и понимать.
\end{frame}
\lecturenotes
Сложность, как правило, растет гораздо быстрее размеров программы. И если не позаботиться об этом заранее, то довольно быстро наступает момент, когда ты перестаешь ее контролировать. Правильная архитектура экономит очень много сил, времени и денег. А нередко вообще определяет то, выживет ваш проект или нет. И даже если речь идет всего лишь о «построении табуретки» все равно вначале очень полезно ее спроектировать.\\
Вообще говоря, не существует общепринятого термина «архитектура программного обеспечения». Тем не менее, когда дело касается практики, то для большинства разработчиков и так понятно какой код является хорошим, а какой плохим. Хорошая архитектура это прежде всего выгодная архитектура, делающая процесс разработки и сопровождения программы более простым и эффективным. Программу с хорошей архитектурой легче расширять и изменять, а также тестировать, отлаживать и понимать.

\subsubsection{Оценка}
\begin{frame} \frametitle{Оценка}
Есть всего два способа делать оценки:
  \begin{enumerate}
	\item Опираясь на свою картину мира в прошлом (свой опыт и исторические данные); 
	\item Продлевая свою картину мира в будущее (экспертные оценки и интуиция). 
  \end{enumerate}
\end{frame}
\lecturenotes
Один из самых ненавистных вопросов к разработчикам: «Когда все будет готово?». Корневая причина состоит в недостатке информации и неопределенностью. Предсказать наперед что будет в будущем невозможно, однако конкретные рамки все равно необходимы. Есть всего два способа делать оценки:\\
\begin{enumerate}
	\item Опираясь на свою картину мира в прошлом (свой опыт и исторические данные); 
	\item Продлевая свою картину мира в будущее (экспертные оценки и интуиция). 
\end{enumerate}\\
На начальном этапе, когда человек еще не приучен давать точные оценки, ему надо рассказать, зачем они нужны. Задача руководителя — приучить оценивать свою работу. Не просто давать задачу и смотреть, как человек бежит ее делать, а убедиться, что нужный объем работ проанализирован. При этом стоит не забывать, что закон Мёрфи распространяется даже на самых квалифицированных экспертов.

\subsubsection{Обучение}
\begin{frame} \frametitle{Обучение}
Обучение сотрудников в сфере IT-кампаний – очень важная составляющая процесса разработки программного обеспечения. Обучать необходимо не только новых людей под задачи конкретной кампании, но и периодически повышать квалификацию и уже работающих сотрудников.
\end{frame}
\lecturenotes
Компании в основном предоставляют возможность проходить обучение как на очных, так и дистанционных курсах, тренингах. Цель таких курсов — получение знаний в профессиональной области (программирование, системное администрирование, тестирование, управление качеством, финансовый менеджмент и т.д.).\\
IT-компании активно поддерживают идею смешанного формата обучения, где технологии покрывают часть процесса. Многие IT-предприятия запускают онлайн-курсы и другие формы корпоративного обучения для поддержки профессиональных навыков сотрудников и предоставления новых знаний. Экономика изменяется и влияет на бизнес-среду, которая требует развития новых профессиональных и личностных качеств.\\
Рынок информационных технологий хоть и заполнен кадрами, но испытывает дефицит в действительно качественно подготовленных кандидатах. Когда предприятие нуждается в профессионалах, оно вынуждено искать альтернативные выходы из ситуации. Крупные IT-компании часто решают эту задачу внутри предприятия – вкладывают инвестиции в обучение и развитие собственного персонала. Такой подход повышает и компетенцию сотрудника, и его лояльность.


\subsection{Разработка}
\subsubsection{Типичные задачи разработчика}
\begin{frame} \frametitle{Типичные задачи разработчика}
	(в порядке убывания по временным затратам)
\begin{itemize}
	\item Обсуждение задач с командой
	\item Чтение кода
	\item Отладка кода
	\item Написание кода
\end{itemize}
\end{frame}
\lecturenotes
Обсуждение задач с командой обычно занимает большую часть времени, но его можно систематизировать разными методами (например --- Agile).\\
Чтение кода является очень важным и временно затратным пунктом разработки ПО в команде, поэтому разработчику могут понадобиться не только навыки написания кода, но и его оформления и чтения кода коллег.\\
Отладка: Независимо от обстоятельств код, создаваемый разработчиками программного обеспечения, далеко не всегда работает так, как задумано. В подобных ситуациях необходимо выяснить, почему так происходит. При этом вместо многочасового изучения кода в поисках ошибок гораздо проще и эффективнее будет использовать средство отладки (отладчик).\\
Код --- та часть работы, которая обычно ассоциируется с разработкой ПО как таковой. Важно, чтобы код был в достаточной мере оптимизированным, лаконичным и понятным. Назначаем на подобранные под специфику задания в ТЗ языки специализирующихся на их использовании программистов.

\subsubsection{Тестирование}
\begin{frame} \frametitle{Тестирование}
Тестирование программного обеспечения — проверка соответствия между реальным и ожидаемым поведением программы, осуществляемая на конечном наборе тестов, выбранном определенным образом.
\end{frame}
\lecturenotes
Тестирование программного обеспечения — проверка соответствия между реальным и ожидаемым поведением программы, осуществляемая на конечном наборе тестов, выбранном определенным образом. В более широком смысле, тестирование — это одна из техник контроля качества, включающая в себя активности по планированию работ (Test Management), проектированию тестов (Test Design), выполнению тестирования (Test Execution) и анализу полученных результатов (Test Analysis).\\
Тестирование проводится на каждом этапе разработки ПО, включает множество тестов по плану тестирования, кастомизируемому с учётом специфики проекта на этапе составления технического задания. Результаты тестирования документируются и доступны клиенту в режиме реального времени. Оплата за продукт производится только после прохождения всех видов тестов, в том числе клиентских.

\subsubsection{Рефакторинг}
\begin{frame} \frametitle{Рефакторинг}
Рефакторинг представляет собой процесс такого изменения программной системы, при котором не меняется внешнее поведение кода, но улучшается его внутренняя структура. Это способ систематического приведения кода в порядок, при котором шансы появления новых ошибок минимальны. В сущности, при проведении рефакторинга кода вы улучшаете его дизайн уже после того, как он написан.
\end{frame}
\lecturenotes
Рефакторинг представляет собой процесс такого изменения программной системы, при котором не меняется внешнее поведение кода, но улучшается его внутренняя структура. Это способ систематического приведения кода в порядок, при котором шансы появления новых ошибок минимальны. В сущности, при проведении рефакторинга кода вы улучшаете его дизайн уже после того, как он написан.\\
«Улучшение кода после его написания» --- непривычная фигура речи. В нашем сегодняшнем понимании разработки программного обеспечения мы сначала создаем дизайн системы, а потом пишем код. Сначала создается хороший дизайн, а затем происходит кодирование. Со временем код модифицируется, и целостность системы, соответствие ее структуры изначально созданному дизайну постепенно ухудшаются. Код медленно сползает от проектирования к хакерству.\\
Рефакторинг представляет собой противоположную практику. С ее помощью можно взять плохой проект, даже хаотический, и переделать его в хорошо спроектированный код. Каждый шаг этого процесса прост до чрезвычайности. Перемещается поле из одного класса в другой, изымается часть кода из метода и помещается в отдельный метод, какой-то код перемещается в иерархии в том или другом направлении. Однако суммарный эффект таких небольших изменений может радикально улучшить проект. Это прямо противоположно обычному явлению постепенного распада программы.\\
При проведении рефакторинга оказывается, что соотношение разных этапов работ изменяется. Проектирование непрерывно осуществляется во время разработки, а не выполняется целиком заранее. При реализации системы становится ясно, как можно улучшить ее проект. Происходящее взаимодействие приводит к созданию программы, качество проекта которой остается высоким по мере продолжения разработки.

\subsubsection{Документирование}
\begin{frame} \frametitle{Документирование}
Документирование --- процедура, фиксирующая план, процесс и результат разработки программного обеспечения. Включает в себя всю исходную информацию (ТЗ, макеты), планы работ, затрат, тестирования, список задач исполнителей в каждый момент времени, отчеты о работе и так далее. Документация необходима для быстрого и точного выявления ошибок, прозрачности совместной работы, как обязательная юридическая часть договора.
\end{frame}
\lecturenotes
Глобальный вопрос – зачем вообще заказчику нужна документация на конкретное решение? Не та «для галочки», которую с них спросит вышестоящее руководство, или которую требует нормативное регулирование отрасли. С ней все понятно, она просто нужна. Речь про документацию, которая реально применяется в дальнейшем и имеет прикладную ценность.\\
Универсального ответа на этот вопрос нет. В зависимости от обстоятельств, честный ответ будет варьироваться от «не нужна!» до «а как без нее?». Однако есть определенная зависимость. Чем продолжительнее предполагаемый период жизни и развития решения, тем важнее в нем роль документации. \\
Потому, что тем дольше понадобится:\\
Заказчику:\\
\begin{itemize}
\item Адаптировать решение к изменяющимся реалиям в части функционала и покрытия задач. Причем быстро.
\item Расширять область автоматизации, в том числе, за счет интеграции с другими системами.
\item Учить сменяющийся (иногда – часто и много) персонал.
\item Актуализировать свою собственную нормативную базу, связанную с применением решения (регламенты, положения об отделах, технологические инструкции, должностные инструкции).
\item Вести учет, планирование и оптимизацию затрат на решение. Как минимум, чтобы знать, за что в нем уже оплачено и не попадаться ревизорам на повторной оплате того же самого с вытекающими обвинениями в двойном финансировании или отмывании.
\item Обеспечивать необходимый уровень безопасности предприятия – в том числе, посредством исключения зависимости от единственного поставщика/разработчика критично важного решения.
\end{itemize}\\
Исполнителю: \\
\begin{itemize}
\item Передавать компетенции по разработке решения (за 2 года сменится половина команды «в теме», дальше – больше).
\item Передавать компетенции по обеспечению качества и сопровождению решения – тестировщики и персонал поддержки меняются еще быстрее.
\item Актуализировать информационное обеспечение для менеджера продукта.
\end{itemize}\\
Если заказчик не справится, его ждут весьма ощутимые финансовые и временнЫе потери с перспективой разбираться, разрабатывать и внедрять решение заново. Если не справится исполнитель, то он просто перестанет быть исполнителем. Сразу или спустя некоторое время.

\subsection{Сопровождение}
\begin{frame} \frametitle{Сопровождение}
Сопровождение ПО – это доработка программ после их внедрения. Вам не нужно нанимать системного администратора, который будет обеспечивать функциональность системы --- разработчики «Горизонтов роста» будут делать это лучше и быстрее, потому что знают ПО изнутри. В задачи сопровождения и продвижения программного обеспечения входит:
\begin{itemize}
\item Адаптация
\item Расширение функционала
\item Корректировка документации
\item Исправление ошибок
\item Обучение персонала
\end{itemize}
\end{frame}
\lecturenotes
Сопровождение ПО – это доработка программ после их внедрения. Вам не нужно нанимать системного администратора, который будет обеспечивать функциональность системы --- разработчики «Горизонтов роста» будут делать это лучше и быстрее, потому что знают ПО изнутри. В задачи сопровождения и продвижения программного обеспечения входит:\\
\begin{itemize}
	\item Адаптация программного обеспечения к рабочим условиям. Например, настройка удаленного доступа или адаптация под параметры аппаратуры, на которой будет установлена программа.
	\item Расширение функционала. По мере расширения компании появляется необходимость в новых функциях действующего ПО. Правильно их добавить может только специалист, который занимался разработкой.
	\item Корректировка документации. Любое изменение, которое вносится в уже действующую программу, фиксируется в документации на ПО.
	\item Исправление ошибок. Выявление всех неисправностей возможно только после полноценного ввода ПО в эксплуатацию.
	\item Обучение персонала. После внедрения программного обеспечения, мы обучаем сотрудников компании-заказчика работе со всеми функциями новой системы. Это ускоряет работу и помогает использовать ПО по-максимуму. 
\end{itemize}


\end{document}

%%% Local Variables: 
%%% mode: TeX-pdf
%%% TeX-master: t
%%% End: 
