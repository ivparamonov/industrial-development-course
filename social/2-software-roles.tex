\documentclass{../industrial-development}
\graphicspath{{2-software-roles/}}

\title{Роли сотрудников в бизнес-процессах компании по разработке программного обеспечения}
\author{Гончаров Александр, Галкин Влад ИТ-21 МО}
\date{}

\begin{document}
	
	\begin{frame}
		\titlepage
	\end{frame}
	
	\begin{frame}{План лекции}
		\tableofcontents
	\end{frame}
	
	
	\section{Разработчик }
	
	\begin{frame} \frametitle{Разработчик}
		\begin{block}{}
			\alert {}Разработчик – работник, от которого ждут способности перевести алгоритмы и технические спецификации в код, исполняемы на компьютере. Программист должен знать синтаксис языка, а также должен быть способен не только следовать указаниям при обучении новым технологиям и надстройкам, но и быть способен анализировать их, понимать, и при возможности, улучшать.
			
		\end{block}
		
	\end{frame}
	
	\begin{frame} \frametitle{Разработчик: задачи}
		\begin{itemize}
			\item Использование синтаксиса языка
			\item Анализ и понимание новых технологий
			\item Тенхническая консультация
			\item Разработка инфраструктуры
		\end{itemize}
	\end{frame}
	
	\begin{frame} \frametitle{Разработчик: вовлеченность}
		Разработчик ответственнен за создание кода проекта, и его поддержку. Играет центральную роль в разработке проекта.
	\end{frame}
	
	\begin{frame} \frametitle{Разработчик: особенности}
		Написание кода для создания определенных компонентов продукта, и их улучшение. 	
	\end{frame}

	\lecturenotes
		Разработчик - центральная фигура в процессе разработки программного обеспечения. Основная роль, без которой не обходится ни один процесс разработки так как разработчик пишет непосредственно сам код, прописывает логику, другими словами - создает продукт. Он продумывает структуру компонентов, их взаимодействие между собой в рамках определенной функциональности, надежность кода к ошибкам и другим нестандартным случаям, базовая проверка функциональности. Также разработчику необходимо ориентироваться в новых технологиях, регулярно изучать их, потому что современная разработка продукта происходит только на последних версиях актуальных и популярных технологиях, для удобства, ускорения и экономии ресурсов разработки, так как все технологии направленны именно на это.
		Также задача разработчика оказывать техническую консультацию с другими ролями по выбору или какому-то другому вопросу по определённой технологии, так как разработчик должен быть больше всех осведомлен в технологиях, потому что ближе всех других с ними работает.
	
	\section{Тестировщик }
	
	\begin{frame} \frametitle{Тестировщик}
		\begin{block}{}
			\alert {Тестировщик} – роль работника, ответственного за обеспечение определенного уровня качества для финального клиента путём помощи команде разработки в поиске и определении проблем в процессе.
		\end{block}
		
	\end{frame}
	
	\begin{frame} \frametitle{Тестировщик: задачи}
		\begin{itemize}
			\item Создание тестовых случаев и скриптов
			\item Отчетность о тестах
			\item Планирование тестов
			\item Случайное тестирование
			\item Выполнение скриптов и наблюдение за результатами 
			\item Повышение надежности кода и продукта
		\end{itemize}
	\end{frame}
	
	\begin{frame} \frametitle{Тестировщик: вовлеченность}
		Тестировщик ответственнен за надежность и отказоустойчивость проекта. Играют важную роль в проекте, так как от него зависит надежность будущего функционала.
	\end{frame}
	
	\begin{frame} \frametitle{Тестировщик: особенности}
		Он покрывает тестами функционал, которые тестируют надежность кода как отдельных компонентов так и всей системы в целом. 
	\end{frame}
	
	\lecturenotes
		Тестировщик должен сопровождать проект по разработке программного продукта с самых ранних этапов. Это поможет сократить расходы и добиться более высокого качества на выходе. Задачи тестировщика проверить продукт на устойчивость его к разным действием пользователей, для того чтобы предотвратить поломку проекта в случае нестандартного действия пользователя и в то же время проверить что весь функционал работает корректно в любых действиях пользователя, или других внешних раздражителей, таких, как слабый интернет, внезапное отключения устройства и так далее. Также тестировщик должен спланировать график проведения тестов максимально эффективно для проекта и чтобы это было возможным сделать команде по тестированию. Все дефекты и ошибки, которые находит тестировщик, должны быть им задокументированы, чтобы программисты могли их устранить. Тестировщик отвечает за надежность и устойчивость разрабатываемой логики. Коротко: задача тестировщика уменьшить количество ошибок путем проверок крайних входящих данных.
	
	\section{Инженер по автоматизации тестирования}
	
	\begin{frame} \frametitle{Инженер по автоматизации тестирования}
		\begin{block}{}
			\alert {Инженер по автоматизации тестирования} - роль работника, ответственного за создание автоматических тестов, которые запускаются регулярно при каких-то действиях разработчиков позволяя разработчикам избежать ошибок, которые проверяются в тесте.
		\end{block}
		
	\end{frame}
	
	\begin{frame} \frametitle{Инженер по автоматизации тестирования: задачи}
		\begin{itemize}
			\item Создание автоматических тестов
			\item Планирование эффективного запуска автоматических тестов
			\item Повышение автоматизации рутинных задачь для других участников проекта		
		\end{itemize}
	\end{frame}
	
	\lecturenotes
		Инженер по автоматизации тестирования играет довольно важную роль, так как его действия экономят время, а значит и ресурсы на рутинных задачах при разработке продукта. Автоматические тесты имеют ряд преимуществ перед ручными, а именно они могут быть написаны один раз и выполняться неограниченное количество, в свою очередь во время ручного тестирования необходимы усилия разработчика на каждый тест. Автоматические тесты возможно запускать после каждого изменения кода без дополнительных действий со стороны разработчика, ему будет необходимо лишь проверить результаты тестирования. 
		
		На инженере по автоматизации тестирвоания лежит большая ответственность за тесты, так как необходимо учитывать все варианты использования того, что тестируется, так как эти тесты будут выполняться много раз и у разных разработчиков, что не должно привести к воспроизведению какого-либо кейса у разработчика, пропущенного по вине автотестировщика.
		
	\section{Дизайнер }
	
	\begin{frame} \frametitle{Дизайнер}
		\begin{block}{}
			\alert {Дизайнер} - работник, отвечающий за создание интерфейса для пользователей, а также создание макетов по которым разработчики будут создавать продукт.
		\end{block}
		
	\end{frame}
	
	\begin{frame} \frametitle{Дизайнер: задачи}
		\begin{itemize}
			\item Разработка новых прикладных решений для иконок, шрифтов и т.д.
			\item Создание макета приложений для разработчиков
			\item Проектирование пользовательских интерфейсов
			\item Определение организованности элементов интерфейса
			\item Группировка элементов интерфейса
			\item Выравнивание элементов интерфейса
			\item Создание единого стиля элементов интерфейса
			\item Разграничение информационных блоков
		\end{itemize}
	\end{frame}
	
	\begin{frame} \frametitle{Дизайнер: вовлеченность}
		Дизайнер решает как будет выглядить внешний вид продукта. Составляет макеты, по которым разработчики будут разрабатывать интерфейс продукта.
	\end{frame}
	
	\begin{frame} \frametitle{Дизайнер: особенности}
		Необходимо понимание "user friendly" интерфейсов, чтобы пользователям было комфортно работать с приложением.
	\end{frame}

	\lecturenotes
		Дизайнер - довольно важная роль, так как он отвечает за внешний вид продукта и его удобство для пользователей. На дизайнера возложенна высокая отвественность за продукт, так как интерфейс является связующим звеном между клиентом и логикой и если интерефейс будет плохого качества, то пользователю будет работать с продуктом неудобно, не смотря на то, какой бы хорошей не была логика. 
		
		Основная его работа заключается непосредственно в разработке макетов сайтов (порталы, интернет-магазины, лендинги), дизайн баннеров, иконки, интерфейсы мобильных и браузерных приложений, так же он должен знать стандарты проектирования. 
		
		Дизайнер выстраивает взаимное расположение элементов, схема перемещения по экранам, анимации — все это он должен продумать. Если приложение большое и сложное, то дизайнер может быть и не один. Или, например, один из дизайнеров может только рисовать интерфейс, а другой — глобально продумывать всю концепцию взаимодействия с пользователем. Иногда они совместно с программистами несут ответственность за часть ошибок в приложении: бывает, что реализовать придуманный дизайн довольно сложно, а при возрастании сложности увеличивается и вероятность совершения ошибок, таким образом, обратная связь в виде отзывов пользователей о неудобности использования продукта относится к изначально плохо продуманной идее дизайнера. 
	
	\section{Технический писатель }
	
	\begin{frame} \frametitle{Технический писатель}
		\begin{block}{}
			\alert {Технический писатель} - специалист, занимающийся документированием в рамках решения технических задач, в частности разработки программного обеспечения
		\end{block}
	\end{frame}
	
	\begin{frame} \frametitle{Технический писатель: задачи}
		\begin{itemize}
			\item Оформление документов в соответствии со стандартом (принятым в организации, международным или, например, ГОСТОМ)
			\item Поддерживание документов в актуальном состоянии.
			\item Анализ новых IT-продуктов и программ, а также их тестирование. В конечном счете, подобные занятия помогут специалисту лучше разобраться в новом продукте, а значит написать понятную и подробную инструкцию.
			\item Сбор информации о продукте.
		\end{itemize}
	\end{frame}
	\begin{frame} \frametitle{Технический писатель: вовлеченность}
		Технический писатель ответственен за составление документации в рамках разработки различных программ.
	\end{frame}
	
	\begin{frame} \frametitle{Технический писатель: особенности}
		Составление грамотной документации на понятном языке как для разработчика, так и для конечного пользователя.
	\end{frame}
	\begin{frame} \frametitle{Технический писатель: вовлеченность}
		Технический писатель ответственен за составление документации в рамках разработки различных программ.
	\end{frame}
	
	\begin{frame} \frametitle{Технический писатель: особенности}
		 Составление грамотной документации на понятном языке как для разработчика, так и для конечного пользователя.
	\end{frame}
	
	\lecturenotes
		Основная задача технического писателя — написание документа, который бы удовлетворял определённым требованиям. Требования могут определяться как нормативными актами, существующими в отрасли применения продукта, так и различными целями, которые организация или разработчик ставит перед собой. Например, сокращение расходов по сопровождению продукта путём разработки точного и понятного описания или обеспечение документированием процесса разработки для последующего подтверждения соответствия системе качества. Как правило, технические писатели компетентны как в области языкознания, так и в технической области. Квалифицированный технический писатель умеет создавать, редактировать, иллюстрировать и адаптировать технический материал лаконично и понятно. 
		
		Ответственность такой роли довольно высока, документация составленная неверно, может ввести в заблуждение как пользователя так и разработчика. Технический писатель должен уметь грамотно излагать мысли и правильно толковать "сложные вещи простыми словами". Специалисту этого вида несомненно придется постоянно самостоятельно изучать новые предметные области, а затем обрабатывать полученную информацию.
		
	\section{Бизнес-аналитик}
	
	\begin{frame} \frametitle{Бизнес-аналитик}
		\begin{block}{}
			\alert {}В классическом понимании, {бизнес-аналитик} — это человек, который анализирует бизнес-потребности организации, а также формулирует пути и схемы усовершенствования бизнес-процессов, осуществляет стратегическое планирование. 
		\end{block}
		
	\end{frame}
	
	\begin{frame} \frametitle{Бизнес-аналитик: задачи}
		\begin{itemize}
			\item Усовершенствование продуктов компании
			\item Работа с клиентом
			\item Контроль над качеством продукта и соответствием требованиям клиента
			\item Формулирование высокоуровневых требований к программному продукту
			\item Составление его структуры и связей между элементами
			\item Определение технологий и/или используемых программных решений
			\item Анализ способа взаимодействия между пользователем и программой
		\end{itemize}
	\end{frame}
	
	\begin{frame} \frametitle{Бизнес-аналитик: вовлеченность}
		Бизнес-аналитик отвечает за правильное направление разработки проекта, задает параметры и ожидаемый функционал проекта. Является главным связующим звеном между сотрудниками из разных отделов.
	\end{frame}
	
	\begin{frame} \frametitle{Бизнес-аналитик: особенности}
		Роль, требуемая преимущественно взаимодействия с людьми из разных сфер с пониманием общей структуры проекта и вектора его движения. 
	\end{frame}
	
	\lecturenotes
		 Бизнес-аналитик является важной фигурой при создании продукта, он должен уметь выявлять проблемы бизнеса заказчика и найти максимально эффективное решение, а с полученными знаниями определить в какую сторону будет двигаться вектор разработки и какую функциональность иметь. Важным критерием роли является баланс между пониманием потребностей пользователя и возможностями разработчиков. Бизнес-аналитик работает с требованиями на всех этапах жизненного цикла разработки ПО и постоянно выступает посредником между заказчиком и командой программистов. 
		 
		 Основными этапами в работе такого специалиста можно назвать: 
		 -сбор, формализацию и согласование требований с заказчиками; 
		 -описание и моделирование бизнес-процессов; 
		 -анализ эффективности и выработка предложений по оптимизации процессов; 
		 -подготовка сравнительного анализа деятельности компании; 
		 -подготовка презентаций для руководства и заказчиков. 
		 Кратко данную роль можно охарактеризовать так: бизнес-аналитики помогают разным сторонам понимать друг друга, и в результате получают реализацию, которая удовлетворит всех. 
	
	\section{Администратор} 
	
	\begin{frame} \frametitle{Системный администратор}
		\begin{block}{}
			\alert {Системный администратор} - сотрудник, должностные обязанности которого подразумевают обеспечение штатной работы парка компьютерной техники, сети и программного обеспечения.
		\end{block}
	\end{frame}
	
	\begin{frame} \frametitle{Системный администратор: задачи}
		\begin{itemize}
			\item Подготовка и сохранение резервных копий данных, их периодическая проверка и уничтожение
			\item Установка и конфигурирование необходимых обновлений для операционной системы и используемых программ
			\item Установка и конфигурирование нового аппаратного и программного обеспечения
			\item Создание и поддержание в актуальном состоянии пользовательских учётных записей
			\item Ответственность за информационную безопасность в компании
			\item Устранение неполадок в системе
			\item Планирование и проведение работ по расширению сетевой структуры предприятия
		\end{itemize}
	\end{frame}
	\begin{frame} \frametitle{Системный администратор: вовлеченность}
		Поддержка корректной работы компьютерной техники и программного обеспечения, а также ответственность за информационную безопасность организации.
	\end{frame}
	
	\begin{frame} \frametitle{Системный администратор: особенности}
		Знать не только особенности настройки различных ОС, но и уметь программировать хотя бы на базовом уровне.
	\end{frame}

	\lecturenotes
		В сферу деятельности системного администратора входит обеспечение рабочего состояния компьютерного оборудования, проектирование, администрирование и модернизация локальной сети, поддержка центрального сервера. Сюда относится ответственность за бесперебойную работу компьютеров у сотрудников компании, установка прав доступа к различным ресурсам внутренней и внешней сети. Деятельность системного администратора сосредоточена в обеспечении информационной безопасности компании, то есть предупреждение сбоя любого компонента системы, ликвидация последствий сбоя без ущерба для работы организации. 
		
		Поскольку у этого специалиста довольно широкий перечень обязанностей, а сфера его влияния распространяется на деятельность всей компании, ответственность на этой должности довольно высока. Он ответственен за нарушения работы техники и серверов, опоздания с регистрацией пользователей и несвоевременное информирование руководителя о любых нарушениях пользования локальными ресурсами.
	
\end{document}
%%% Local Variables: 
%%% mode: TeX-pdf
%%% TeX-master: t
%%% End:
