\documentclass{../industrial-development}
\graphicspath{{13-technical-support/}}

\title{Лекция №\,13 по теме «Организация~и~автоматизация технической поддержки программных продуктов»}
\author{ }
\date{}

\begin{document}

\begin{frame}
  \titlepage
\end{frame}

\begin{frame}{План лекции}
  \tableofcontents
\end{frame}  

%%%%%%%%%%%%%%%%%%%%%%%%%%%%%%%%%%%%%%%%%%страница 1
\section{Процесс сопровождения}
\subsection{Определение процесса сопровождения}


\begin{frame} \frametitle{Предпосылки к появлению традиционных методов}

	\begin{definition}
		\alert{Под сопровождением программного обеспечения} 
		 понимают процесс улучшения, оптимизации~и~устранения дефектов программного обеспечения (ПО) после передачи в эксплуатацию
	\end{definition}
	
	Основная задача --- изменить~и~улучшить существующий программный продукт, сохраняя его целостность и~функциональную пригодность
	
\end{frame}

\lecturenotes
Процесс сопровождения является одной из фаз жизненного цикла программного обеспечения, следующей за передачей ПО в эксплуатацию, и завершается выводом его из эксплуатации. В ходе сопровождения в программу вносятся изменения, с тем, чтобы исправить обнаруженные в процессе использования дефекты и недоработки, для добавления новой функциональности, повышения удобства использования (юзабилити) и роста уровня использования ПО. По стандарту ISO/IEC 12207, этот процесс входит в 5 основных процессов жизненного цикла (ЖЦ) ПО: приобретение, поставка, разработка, эксплуатация, сопровождение. 


%%%%%%%%%%%%%%%%%%%%%%%%%%%%%%%%%%%%%%%%%%страница 2
\subsection{Задачи сопровождения}
\begin{frame} \frametitle{Задачи сопровождения}
	\begin{itemize}
		\item устранение сбоев, исправление ошибок
		\item улучшение дизайна 
		\item расширение функциональных возможностей 
		\item модернизация
		\item адаптация для возможности работы на другой аппаратной платформе
	\end{itemize}
\end{frame}

\lecturenotes


%%%%%%%%%%%%%%%%%%%%%%%%%%%%%%%%%%%%%%%%%%страница 4
\subsection{Типы заявок предложений о модификации}
\begin{frame} \frametitle{Типы заявок предложений о модификации}
	\begin{definition}
		\alert{Корректирующая} --- это изменение программного продукта для коррекции обнаруженных проблем\newline
		\alert{Адаптивная} --- изменение программного продукта после поставки для обеспечения его использования в условиях изменения его самого или окружающей среды
	\end{definition}
\end{frame}


%%%%%%%%%%%%%%%%%%%%%%%%%%%%%%%%%%%%%%%%%%страница 5
\begin{frame} \frametitle{Типы заявок предложений о модификации}
	\begin{definition}
		\alert{Cовершенствующая} --- изменение программного продукта после поставки для улучшения производительности или удобства эксплуатации\newline
		\alert{Профилактическая} ---  это изменение программного продукта после поставки для выявления и исправления скрытых дефектов в ПО до того, как они станут явными ошибками
	\end{definition}
\end{frame}

\lecturenotes
Следует также отметить, что профилактическое и совершенствующее сопровождение относятся к проактивному подходу к сопровождению, при котором инициатива исходит от обслуживающего персонала, а корректирующее и адаптивное — к реактивному подходу, инициатива которого находится у пользователей. 
Проактивному сопровождению необходимо уделять достаточно внимания, поскольку именно оно в наибольшей степени способствует повышению удовлетворенности пользователей и эффективному развитию программной системы. 

%%%%%%%%%%%%%%%%%%%%%%%%%%%%%%%%%%%%%%%%%%страница 6
\subsection{Концепции сопровождения}
\begin{frame} \frametitle{Концепции сопровождения}
	\begin{enumerate}
	\item Область сопровождения программного средства
		\begin{enumerate} 
		\item Типы выполняемого сопровождения
		\item Сопровождаемый уровень документов
		\item Реакция (чувствительность) на сопровождение
		\item Обеспечиваемый уровень обучения персонала 
		\item Обеспечение поставки продукта
		\item Организация справочной службы («горячей линии») 
		\end{enumerate}
	\end{enumerate}
\end{frame}

\lecturenotes

%%%%%%%%%%%%%%%%%%%%%%%%%%%%%%%%%%%%%%%%%%страница 7
\begin{frame} \frametitle{Концепции сопровождения}
	\begin{enumerate}[2]
	\item Практическое применение (адаптация) данного процесса
	\end{enumerate}
	\begin{enumerate}[3]
	\item Определение организаций (лиц), ответственных за сопровождение
	\end{enumerate}
	\begin{enumerate}[4]
	\item Оценка стоимости сопровождения: 
		\begin{enumerate}
		\item Проезд до места расположения пользователя 
		\item Обучение как сопроводителей, так и пользователей
		\item СПИ (среда программной инженерии) и СТПС (среда тестирования программного средства) и их ежегодное сопровождение
		\item Персонал (зарплата и премии) 
		\end{enumerate}
	\end{enumerate}
\end{frame}

\lecturenotes
Должен быть сформирован соответствующий план сопровождения. Этот план должен подготавливаться одновременно с разработкой программной системы. План должен определять, как пользователи будут размещать свои запросы на модификацию (изменения) или сообщать об ошибках, сбоях и проблемах. 



%%%%%%%%%%%%%%%%%%%%%%%%%%%%%%%%%%%%%%%%%%страница 9
\begin{frame} \frametitle{Организационные работы и работы по сопровождению}
	\begin{enumerate}[1] \item Роли и обязанности сопроводителя до поставки программного продукта:\end{enumerate}
	\begin{itemize}
	\item Реализация процесса 
	\item Определение инфраструктуры процесса 
	\item Установление процесса обучения 
	\item Установление процесса сопровождения 
	\end{itemize}
\end{frame}

\lecturenotes

%%%%%%%%%%%%%%%%%%%%%%%%%%%%%%%%%%%%%%%%%%страница 10
\begin{frame} \frametitle{Организационные работы и работы по сопровождению}
	\begin{enumerate}[2] \item Роли и обязанности сопроводителя после поставки программного продукта:  \end{enumerate}
	\begin{itemize}
		\item Реализация процесса
		\item Анализы проблем и модификаций (изменений)  
		\item Реализация (внесение) модификаций (изменений) 
		\item Рассмотрение и принятие модификаций (изменений) 
		\item Перенос программного средства в новую среду
	\end{itemize}
\end{frame}

\lecturenotes

%%%%%%%%%%%%%%%%%%%%%%%%%%%%%%%%%%%%%%%%%%страница 11
\begin{frame} \frametitle{Организационные работы и работы по сопровождению}
	\begin{enumerate}[2] \item Роли и обязанности сопроводителя после поставки программного продукта:  \end{enumerate}
	\begin{itemize}
		\item Снятие программного средства с эксплуатации 
		\item Решение проблем (включая справочную службу) 
		\item При необходимости — обучение персонала (сопроводителя и пользователя) 
		\item Усовершенствование процесса
	\end{itemize}
		\begin{enumerate}[3]\item Роль пользователя: \end{enumerate}
	\begin{itemize}
		\item Приемочные испытания  
		\item Взаимосвязи (интерфейсы) с другими организациями
	\end{itemize}
\end{frame}

\lecturenotes


%%%%%%%%%%%%%%%%%%%%%%%%%%%%%%%%%%%%%%%%%%страница 12
\subsection{Ресурсы}
\begin{frame} \frametitle{Ресурсы}
	\begin{enumerate} \item Персонал:  
	\begin{itemize}
		\item Состав персонала для конкретного проекта 
	\end{itemize}
		\item Программные средства 
		\item Технические средства
		\item Оборудование (аппаратура): 
	\begin{itemize}
		\item Определение требований к оборудованию (аппаратуре) системы 
	\end{itemize}
	\end{enumerate}
\end{frame}

\lecturenotes
· состав персонала для конкретного проекта;
Структура, отвечающая за сопровождение, должна проводить общую деятельность по бизнес-планированию, касающуюся бюджетирования, финансового менеджмента и управления человеческими ресурсами в области сопровождения. 
· определение программных средств, необходимых для поддержки эксплуатации системы (с учетом системных требований и требований к СПИ, СТПС и инструментальным средствам); 
· определение требований к оборудованию (аппаратуре) системы (помимо технических средств вычислительной техники); 

%%%%%%%%%%%%%%%%%%%%%%%%%%%%%%%%%%%%%%%%%%страница 13
\begin{frame} \frametitle{Ресурсы}
	\begin{enumerate}[5] \item Документы: \end{enumerate}
	\begin{itemize}
		\item План обеспечения качества
		\item План управления проектом 
		\item План управления конфигурацией 
		\item Документы разработки 
		\item Руководства по сопровождению 
		\item План проведения верификации 
	\end{itemize}
	
\end{frame}

\lecturenotes

%%%%%%%%%%%%%%%%%%%%%%%%%%%%%%%%%%%%%%%%%%страница 14
\begin{frame} \frametitle{Ресурсы}
	\begin{enumerate}[5] \item Документы: \end{enumerate}
	\begin{itemize} 
		\item План проведения аттестации (валидации) 
		\item План тестирования, процедуры тестирования и отчеты о тестировании 
		\item План обучения 
		\item Руководство пользователя
	\end{itemize}
	\begin{enumerate}[6] \item Данные   \end{enumerate}
	\begin{enumerate}[7] \item Другие требования к ресурсам (при необходимости) \end{enumerate}
	
\end{frame}

\lecturenotes


%%%%%%%%%%%%%%%%%%%%%%%%%%%%%%%%%%%%%%%%%%страница 22
\section{Техническая поддержка}

\subsection{Понятие технической поддержки}

\begin{frame} \frametitle{Техническая поддержка}
	\begin{definition} 
		\alert {Техническая поддержка} -- это сервисная структура, разрешающая проблемы пользователей с компьютерами, аппаратным и программным обеспечением. Важная функциональная составляющая ITIL (библиотека инфраструктуры информационных технологий), позволяющая выявить проблемные участки инфраструктуры ИТ, оценить эффективность работы подразделения ИТ
	\end{definition}
\end{frame}
\lecturenotes

\subsection{Многоуровневый тип технической поддержки}
\begin{frame} \frametitle{Многоуровневый тип технической поддержки}
    \centerline{\includegraphics[width=\textwidth]{structure.jpg}}
\end{frame}

\lecturenotes

\begin{frame} \frametitle{Этап заказчика}

Техническая поддержка обычно предоставляется 
	\begin{itemize}
        \item по телефону
        \item по электронной почте
        \item с использованием специальных программ(helpdesk) (1С ИТИЛИУМ, eStreamDesk, Mojo, DIRECTUM, HP, IBM)
	\end{itemize}
\end{frame}

\lecturenotes

\begin{frame} \frametitle{1-ая линия поддержки}

    Операторы (1-я линия поддержки, Call-center) — регистрирует обращение, при возможности помогает пользователю самостоятельно, либо эскалирует (передаёт и контролирует выполнение) заявку на вторую линию поддержки.
    \newline
    \newline
    Они в отличие от остальных специалистов ИТ, ориентированы именно на общение с простыми пользователями и призваны решать несложные (по уровню технической компетенции) обращения пользователей, а также реагировать на доступные их пониманию сообщения систем мониторинга

\end{frame}
\lecturenotes

\begin{frame} \frametitle{2-ая линия поддержки}

    Вторая линия поддержки — получает заявки от первой линии, работает по ним, при необходимости привлекая к~решению проблемы специалистов из смежных отделов (системные администраторы, поддержка POS-терминалов, поддержка специального ПО и~т.~д.)
    \newline
    \newline
    В обязанности специалистов второго уровня входят: 
	\begin{itemize}
        \item контакт и оказание помощи персоналу первой линии
        \item фиксация и последующий анализ инцидента
        \item решение проблемы
        \item передача данных по решенной проблеме на первый уровень
	\end{itemize}
\end{frame}
\lecturenotes
Специалисты второго уровня могут, например, пройти дополнительное обучение по поддержке распространенных операционных систем (например, Microsoft Windows) или аппаратных средств~и, следовательно, иметь возможность решать более сложные задачи, затрагивающие распространенные технологические решения.
    Вторая линия поддержки — получает заявки от первой линии, работает по ним, при необходимости привлекая к~решению проблемы специалистов из смежных отделов (системные администраторы, поддержка POS-терминалов, поддержка специального ПО, поддержка специального оборудования (Дилинг) и~т.~д.). Помогает избавиться от случаев загруженности специалиста несложными, но объемными задачами.

\begin{frame} \frametitle{3-я линия поддержки}

    Третья линия поддержки - обеспечивается как правило производителями оборудования и разработчиками системного программного обеспечения, а~также разработчиками специализированного прикладного программного обеспечения на основании договоров технического обслуживания и сопровождения программного обеспечения. Заявки на третью линию оформляются со второй линии в~случае если специалистам второй линии недостаточно компетенции для решения проблемы  

\end{frame}
\lecturenotes
В компаниях, которые разрабатывают собственное программное обеспечение, довольно распространена практика выделения групп поддержки уровня 3, отвечающих за конкретные приложения или службы

\begin{frame} \frametitle{Достоинства}
	\begin{itemize} 
		\item Заказчикам предоставлено «единое окно» для взаимодействия с ИТ-поддержкой, независимо от характера обращения
		\item Специалисты с общими техническими навыками, необходимыми для работы в поддержке уровня 1 и 2, легко доступны на рынке труда. Одновременно это упрощает вывод на аутсорсинг одного или обоих этих уровней, что также часто встречается
		\item Специализированные технические ресурсы могут быть ограждены от прямого контакта с пользователями. Это гарантирует, что заявки к ним попадают только после корректного анализа
	\end{itemize}
\end{frame}

\begin{frame} \frametitle{Недостатки}
	\begin{itemize} 
		\item Многоуровневая поддержка создает несколько очередей
		\item Лишние временные затраты: заявка проходит несколько уровней поддержки, на каждом из которых выполняются попытки её решения
		\item «Отфутболивание» задач: заявка может возвращаться на предыдущий уровень при неверной маршрутизации или для дополнительной информации
	\end{itemize}
\end{frame}

\lecturenotes Хотя поддержка первого уровня стремится быть реактивной и в реальном времени, любая заявка, которая не может быть решена на этом уровне, сразу же попадает в очередь. Её сущность меняется, превращаясь из текущей задачи в запись в бэклоге.


\subsection{DIRECTUM}
\begin{frame} \frametitle{DIRECTUM}

    Helpdesk на базе системы DIRECTUM – автоматизация работы службы поддержки организации на базе системы электронного документооборота DIRECUM
    \newline
    \newline
    Helpdesk на базе DIRECTUM позволяет вести базу компьютерной техники в представлении системных блок, мониторов, принтеров с закреплением их за определенными сотрудниками организации и ведением истории изменений по ним, вплоть до утилизации соответствующей техники

\end{frame}
\lecturenotes

\begin{frame} \frametitle{Описание решения и~общая~схема~работы}
Подача заявок сотрудниками осуществляется либо с помощью обложки в проводнике DIRECTUM либо с помощью внутреннего корпоративного сайта, на котором размещаются соответствующие ссылки на доступные действия
\end{frame}
\lecturenotes

\begin{frame} \frametitle{Главное окно Help Desk}
\centerline{\includegraphics[width=\textwidth]{pic1.png}}
\end{frame}
\lecturenotes

\begin{frame} \frametitle{Подача заявки}
При подаче заявки сотрудником организации сначала указывается тип проблемы из списка типовых проблем и в следующем окне задается краткое описание проблемы. 
При необходимости можно приложить ссылки к заявке на любые объекты системы DIRECTUM, например, документы или записи справочников. Таким образом, все заявки сотрудников группируются по типам проблем.
\end{frame}
\lecturenotes

\begin{frame} \frametitle{Список текущих проблем}
\centerline{\includegraphics[width=\textwidth]{pic2.png}}
\end{frame}
\lecturenotes

\begin{frame} \frametitle{Просмотр описания проблемы}
\centerline{\includegraphics[height=7.5cm]{pic3.png}}
\end{frame}
\lecturenotes

\begin{frame} \frametitle{Карточка заявки}
\centerline{\includegraphics[height=7.5cm]{pic4.png}}
\end{frame}
\lecturenotes
Карточка заявки в представлении для сотрудников организации даёт возможность написать дополнительную информацию, приложить дополнительно объекты системы, отменить заявку, например, если она уже потеряла актуальность, посмотреть состояние заявки и ознакомится с описанием решения проблемы.

\begin{frame} 
\centerline{\includegraphics[width=\textwidth]{pic5.png}}
Заявки имеют различные стили отображения в зависимости от своего состояния. Завершенные заявки помечаются как архивные записи с помощью стандартных средств DIRECTUM и не отображаются на экране, не нагружая систему, но их можно отобразить по необходимости.
\end{frame}
\lecturenotes

\begin{frame} \frametitle{Обработка заявки}
Сотрудники службы поддержки осуществляют обработку заявок, поступающих от сотрудников организации и оперативно решают возникшие трудности, получив информацию в виде заявки с отображением информации о сотруднике с помощью всплывающих подсказок DIRECTUM и на основе информации о закреплённой за сотрудником техники.
\end{frame}
\lecturenotes

\begin{frame} \frametitle{Обработка заявки}
\centerline{\includegraphics[height=7.5cm]{pic6.png}}
\end{frame}
\lecturenotes

\begin{frame} \frametitle{Состояния заявок}
	Доступны следующие состояния заявок:
	\begin{itemize}
		\item В очереди: назначается по умолчанию при создании заявки
		\item В работе: устанавливается при появлении назначении исполнителя
		\item Приостановлено: заявка отложена по какой-либо причине
		\item Завершено: заявка выполнена
		\item Отменено: заявка отменена по какой-либо причине
	\end{itemize}
\end{frame}
\lecturenotes

\begin{frame} \frametitle{Информация по комплектующим}
В представлении исполнителя можно посмотреть детальную конфигурацию системного блока, историю изменений и кем были произведены изменения. Информация по основным комплектующим выбирается из соответствующих справочников, т.к. эта информация является статической и меняется очень редко, если что-то сломалось. 
\newline
	Доступны следующие состояния техники:
	\begin{itemize}
		\item В резерве
		\item Установлен
		\item Не рабочий
		\item Утилизирован
	\end{itemize}
	
\end{frame}
\lecturenotes

\begin{frame}
\centerline{\includegraphics[height=7.5cm]{pic7.png}}
\end{frame}
\lecturenotes

\begin{frame} \frametitle{Список техники сотрудника}
С помощью использования иерархии по работникам организации отображается список техники закреплённой за определенным сотрудником.
\centerline{\includegraphics[width=\textwidth]{pic9.png}}
\end{frame}
\lecturenotes

\begin{frame} \frametitle{Достоинства DIRECTUM}
Достоинства Helpdesk на базе DIRECTUM:
	\begin{itemize}
		\item работа всех сотрудников организации в едином информационном пространстве
		\item адаптируемость к нуждам организации с помощью инструментов разработки DIRECTUM
		\item поддержка одной системы DIRECTUM вместо поддержки нескольких систем: других программ типа Help Desk и/или учета техники
		\item статистика обращений по количеству и типу заявок
		\item заявки не теряются и находятся под контролем сотрудников и начальника службы поддержки
		\item возможность приложить к заявке ссылки на объекты DIRECTUM (документы или записи справочников)
		\item простота использования
	\end{itemize}
\end{frame}
\lecturenotes

\subsection{Итилиум}
\begin{frame} \frametitle{1С Итилиум}

    Итилиум -- это российский Service Desk для автоматизации службы обработки заявок и процессов управления IT, изначально разрабатывался как массовая система. Его применяют с платформами «1С:Предприятие» 8.2 и 8.3

\end{frame}
\lecturenotes

\begin{frame} \frametitle{Интерфейс}
\centerline{\includegraphics[width=\textwidth]{pic10.png}}
\end{frame}
\lecturenotes


\begin{frame} \frametitle{Основные возможности}
	\begin{itemize}
    \item Управление уровнем услуг
    \item Управление инцидентами
    \item Управление запросами на обслуживание
    \item Управление проблемами
    \item Управление изменениями
    \item Управление работами
    \item Управление конфигурациями и активами
    \item Управление релизами
	\end{itemize}
\end{frame}
\lecturenotes

\begin{frame} \frametitle{Пример обращения}
\centerline{\includegraphics[width=\textwidth]{pic12.png}}
\end{frame}
\lecturenotes

\begin{frame} \frametitle{Возможности управления инцидентами}
	\begin{itemize}
    \item Регистрация инцидентов, контроль сроков решения инцидентов.
    \item Поддержка схем эскалаций (передача ответственности, уведомления).
    \item Управление нарядами.
    \item Поддержка базы знаний по решению инцидентов.
	\end{itemize}
\end{frame}
\lecturenotes

\begin{frame} \frametitle{Возможности управления проблемами}
	\begin{itemize}
    \item Выявление и регистрация проблем.
    \item Ведение перечня «известных ошибок».
    \item Управление конфигурациями и изменениями:
    \item Учет конфигурационных элементов и изменений.
    \item Связь с инцидентами, проблемами, нарядами.
    \item Хранение документооборота по ИТ активам (конфигурационным элементам), изменениям. 
	\end{itemize}
\end{frame}
\lecturenotes

\begin{frame} \frametitle{Динамика обращений}
\centerline{\includegraphics[width=\textwidth]{pic11.png}}
\end{frame}
\lecturenotes

\begin{frame} \frametitle{Реестр обращений}
\centerline{\includegraphics[width=\textwidth]{pic14.png}}
\end{frame}
\lecturenotes

\begin{frame} \frametitle{Отчет по качеству выполнения обращений}
\centerline{\includegraphics[width=\textwidth]{pic13.png}}
\end{frame}
\lecturenotes

\begin{frame} \frametitle{Достоинства Итилиум}
	\begin{itemize}
		\item Широкий функционал
		\item Гибкость системы благодаря открытому коду
		\item Методика внедрения, методические материалы, обучающие семинары входят в стоимость внедрения
		\item Самостоятельное внедрение или полная настройка специалистами 
		\item Система метрик и показателей, позволяющая строить статистику
		\item Совместимость с 1С, логика и интерфейс знакомы 1С-специалистам
		\item Методологическая русскоязычная поддержка
		\item Возможность использования мобильных устройств
	\end{itemize}
\end{frame}
\lecturenotes


\subsection{Классификация}
\begin{frame} \frametitle{Классификация по типу оплаты}
	\begin{itemize} 
		\item {Время и материал}: этот тип поддержки распространен в технологической отрасли. Также известная как ИТ-поддержка <<break-fix>>, оплата материалов и плата за обслуживание технического персонала возлагается на клиента по заранее согласованной ставке
	\end{itemize}
\end{frame}

\begin{frame} \frametitle{Классификация по типу оплаты}
	\begin{itemize} 
        \item {Управляемые услуги}: обычно это предоставляется крупным клиентам, а не отдельным потребителям. Перечень четко определенных услуг и показателей эффективности предоставляется клиенту на постоянной основе по фиксированной ставке, которая согласовывается по контракту. Предоставляемые услуги могут быть круглосуточным мониторингом серверов, круглосуточной службой поддержки и т.~п. Это может включать посещения на месте, когда проблемы не могут быть решены удаленно
	\end{itemize}
\end{frame}

\begin{frame} \frametitle{Классификация по типу оплаты}
	\begin{itemize} 
        \item {Блокировка часов}: это предоплаченная система поддержки, в которой клиент платит определенное количество времени, которое может использоваться в~месяц или в год. Это позволяет клиентам гибко использовать часы без проблем с бумажной работой или несколькими счетами
	\end{itemize}
\end{frame}

\lecturenotes

\begin{frame} \frametitle{Основные функции систем автоматизации службы поддержки}
	\begin{itemize} 
		\item реализация полного жизненного цикла обработки заявок пользователя: регистрация, назначение и эскалация, закрытие
		\item предоставление web-доступа пользователям для самостоятельной регистрации заявок
		\item ведение Базы конфигурации, в которой хранится вся информация об элементах ИТ-инфраструктуры
		\item ведение Базы знаний
	\end{itemize}
\end{frame}
\lecturenotes

\begin{frame} \frametitle{Основные требования к системам автоматизации службы поддержки}
	\begin{itemize} 
		\item единая точка регистрации заявок пользователей, возможность регистрации заявок сотрудником службы технической поддержки от имени пользователя
		\item поддержка средств контроля сроков выполнения заявок на основании сервисных контрактов с потребителями ИТ-услуг, внутренних контрактов службы ИТ и контрактов с внешними поставщиками ИТ услуг
		\item ручные и автоматические назначения исполнителей
		\item поддержка вложенных процессов по механизму дерево/ветки
		\item делегирование заявки между линиями технической поддержки
		\item поддержка интеграции с внешними системами управления
	\end{itemize}
\end{frame}
\lecturenotes


\end{document}

%%% Local Variables: 
%%% mode: TeX-pdf
%%% TeX-master: t
%%% End: 
